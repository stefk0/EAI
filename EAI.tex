\documentclass[a4paper, 10pt]{report}
%\usepackage[paperwidth=9cm, paperheight=12cm, top=0.5cm, bottom=0.5cm, left=0.0cm, right=0.5cm]{geometry}
%\special{papersize=9cm,12cm}


%%%%%%%%%%%%%
%% MARGINS %%
%%%%%%%%%%%%%

\setlength{\marginparsep}{0.5cm}
\setlength{\oddsidemargin}{0.3cm}
\setlength{\hoffset}{0cm}
\setlength{\marginparwidth}{110pt}

\let\oldmarginpar\marginpar

\renewcommand\marginpar[1]{\-\oldmarginpar[\raggedleft\scriptsize #1]%
{\raggedright\scriptsize #1}}
%\renewcommand\marginpar[1]{\oldmarginpar{\scriptsize #1}}

%%%%%%%%%%%%%%

%\usepackage{ucs}

% \usepackage{natbib}
% \usepackage{bibentry}

\usepackage[bulgarian]{babel}
\usepackage[utf8]{inputenc}
\usepackage[colorlinks=true, linkcolor=blue,pdfstartview=FitV,
citecolor=green, urlcolor=blue]{hyperref}
\usepackage{pifont}
\usepackage{amssymb}
\usepackage{amsmath}
\usepackage{mathrsfs}
\usepackage{latexsym}
\usepackage{amsthm}
\usepackage{paralist}
\usepackage{enumerate}
\usepackage{makeidx}
\usepackage{layout}
\usepackage{framed}
\usepackage{bussproofs}
\usepackage{algorithm}
\floatname{algorithm}{Алгоритъм}
%\usepackage{algorithmic}
\usepackage[noend]{algpseudocode}
%\usepackage{algpseudocode}
\usepackage{float}

%%%%%%%%%%%%%%% TIKZ Package %%%%%%%%%%%%%%%%%%%%%%%
\usepackage{tikz}
\usepackage{pgf}
\usetikzlibrary{arrows,automata}
\usetikzlibrary{positioning}
\usetikzlibrary{backgrounds}
%%%%%%%%%%%%%%%%%%%%%%%%%%%%%%%%%%%%%%%%%%%%%%%%%%%%
\usepackage{caption}
\usepackage{subcaption}

\theoremstyle{definition}
\newtheorem{thm}{Теорема}[chapter]
\newtheorem{crl}{Следствие}[chapter]
\newtheorem{cor}{Следствие}[chapter]
\newtheorem{lemma}{Лема}[chapter]
\newtheorem{prop}{Твърдение}[chapter]
\newtheorem{dfn}{Определение}[chapter]
\newtheorem{problem}{Задача}[chapter]
\newtheorem{example}{Пример}[chapter]
\newtheorem{question}{Въпрос}[chapter]
\newtheorem*{remark}{Забележка}
\renewenvironment{proof}{\noindent{\bf Док.}\hspace*{1em}}{\qed\par}
\newenvironment{hint}{\noindent{\bf Упътване.}\hspace*{1em}}{\qed\par}
\newenvironment{solution}{\noindent{\bf Решение.}\hspace*{1em}}{\qed\par}

\newcommand{\A}{\mathcal{A}}
\newcommand{\B}{\mathcal{B}}
\renewcommand{\C}{\mathcal{C}}
\newcommand{\M}{\mathcal{M}}
\renewcommand{\L}{\mathcal{L}}
\newcommand{\D}{\mathcal{D}}
\newcommand{\R}{\mathbb{R}}
\newcommand{\Z}{\mathbb{Z}}
\newcommand{\N}{\mathcal{N}}
\newcommand{\Q}{\mathbb{Q}}
\newcommand{\Ls}{\mathscr{L}}
\newcommand{\Fs}{\mathscr{F}}
\newcommand{\Rs}{\mathscr{R}}
\newcommand{\Ps}{\mathscr{P}}
\newcommand{\As}{\mathscr{A}}
\newcommand{\Bs}{\mathscr{B}}
\newcommand{\Es}{\mathscr{E}}
\newcommand{\Is}{\mathscr{I}}
\newcommand{\Ss}{\mathscr{S}}
\newcommand{\xn}{x_{1},\dots,x_{n}}

\newcommand{\Nat}{\mathbb{N}}
\newcommand{\Int}{\mathbb{Z}}
\newcommand{\Real}{\mathbb{R}}

\newcommand{\xs}{overline{x}}

\newcommand{\ys}{overline{y}}

\newcommand{\zs}{overline{z}}
\newcommand{\ov}[1]{\overline{#1}}
\newcommand{\abs}[1]{\lvert{#1}\rvert}
\newcommand{\pair}[1]{\langle{#1}\rangle}
\newcommand{\writedown}{\ding{45}\ }

\newcommand{\FA}{\langle{Q,\Sigma,s,\delta,F}\rangle}
\newcommand{\FAn}[1]{\langle{Q_#1,\Sigma,s_#1,\delta_#1,F_#1}\rangle}
\newcommand{\NFA}{\langle{Q,\Sigma,s,\Delta,F}\rangle}
\newcommand{\NFAn}[1]{\langle{Q_#1,\Sigma,s_#1,\Delta_#1,F_#1}\rangle}
\newcommand{\PDA}{\langle{Q,\Sigma,\Gamma,\#,s,\Delta,F}\rangle}
\newcommand{\PDAn}[1]{\langle{Q_#1,\Sigma,\Gamma,\#,s_#1,\Delta_#1,F_#1}\rangle}
\newcommand{\CFG}{\langle{V,\Sigma,R,S}\rangle}
\newcommand{\TM}{\langle{Q,\Sigma,\Gamma,\delta,s,\blank,F}\rangle}

\renewcommand{\iff}{\ \leftrightarrow\ }
\newcommand{\df}{\stackrel{\text{деф}}{=}}
\newcommand{\dff}{\stackrel{\text{деф}}{\iff}}


\newcommand{\Th}[1]{{\em Теорема~\ref{th:#1}}}
\newcommand{\Lem}[1]{{\em Лема~\ref{lem:#1}}}
\newcommand{\Cor}[1]{{\em Следствие~\ref{cor:#1}}}
\newcommand{\Prob}[1]{{\em Задача~\ref{pr:#1}}}
\newcommand{\Prop}[1]{{\em Твърдение~\ref{pr:#1}}}

%\newcommand*{\blank}{\allowbreak\textvisiblespace\allowbreak} % visible space
\newcommand*{\blank}{\sqcup}

\title{Езици, автомати, изчислимост}
\author{Стефан Вътев\thanks{ел. поща: \href{mailto:stefanv@fmi.uni-sofia.bg}{stefanv@fmi.uni-sofia.bg}}}
%, \LaTeX\ файловете са \href{https://github.com/stefk0/EAI}{тук}}}
%, Факултет по математика и информатика, Софийски университет ,,Св. Климент Охридски''}}

\makeindex
\begin{document}
\maketitle
% \layout

\tableofcontents

\chapter{Увод}

\section{Съждително смятане}
\label{sect:propositional}
\marginpar{На англ. Propositional calculus}

Съждителното смятане наподобява аритметичното смятане, като вместо аритметичните операции $+,-,\cdot,/$, 
имаме съждителни операции като $\neg, \wedge, \vee$.
Например, $(p\vee q) \wedge \neg  r$ е съждителна формула.
Освен това, докато аритметичните променливи приемат стойности числа, то
съждителните променливи приемат само стойности {\bf истина (1)} или {\bf неистина (0)}.

{\bf Съждителна формула} наричаме съвкупността от съждителни променливи $p,q,r,\dots$, свързани със знаците за логически операции
$\neg, \vee, \wedge, \rightarrow, \leftrightarrow$ и скоби, определящи реда на операциите.

\subsection*{Съждителни операции}

\begin{itemize}
\item
  Отрицание $\neg$
\item 
  Дизюнкция $\vee$
\item
  Конюнкция $\wedge$
\item
  Импликация $\rightarrow$
\item
  Еквивалентност $\iff$
\end{itemize}

Ще използваме таблица за истинност за да определим стойностите на основните съждителни операции
при всички възможни набори на стойностите на променливите.

\[
\begin{array}{|c|c|c|c|c|c|c|c|c|}
  \hline
  p & q & \neg p & p \vee q & p \wedge q & p \rightarrow q & \neg p \vee q & p \iff q & (\neg{p}\wedge q)\ \vee\ (p\wedge \neg q) \\
  \hline
  0 & 0 & 1 & 0 & 0 & 1 & 1 & 1 & 1\\
  \hline
  0 & 1 & 1 & 1 & 0 & 1 & 1 & 0 & 0\\
  \hline
  1 & 0 & 0 & 1 & 0 & 0 & 0 & 0 & 0\\
  \hline
  1 & 1 & 0 & 1 & 1 & 1 & 1 & 1 & 1\\
  \hline
\end{array}
\]


{\bf Съждително верен} (валиден) е този логически израз, който има верностна стойност {\bf 1} при всички възможни набори на
стойностите на съждителните променливи в израза, т.е. стълбът на израза в таблицата за истинност трябва да съдържа само 
стойности {\bf 1}. 

Два съждителни израза $\varphi$ и $\psi$ са {\bf еквивалентни}, което означаваме $\varphi \equiv \psi$, ако са съставени от 
едни и същи съждителни променливи и двата израза имат едни и същи верностни стойности при всички комбинации от верностни 
стойности на променливите. С други думи, колоните на двата израза в съответните им таблици за истинност трябва да съвпадат.
Така например, от горната таблица се вижда, че 
$p\to q \equiv \neg p \vee q$ и $p \iff q \equiv (\neg{p}\wedge q)\ \vee\ (p\wedge \neg q)$.

\subsection*{Съждителни закони}

\begin{enumerate}[I)]
  \item
    {\bf Комутативен закон}
    \[p\vee q \equiv q\vee p\] 
    \[p \wedge q \equiv q \wedge p\]
  \item
    {\bf Асоциативен закон}
    \[(p\vee q)\vee r \equiv p\vee(q\vee r)\]
    \[(p\ \wedge\ q)\ \wedge\ r \equiv p\ \wedge\ (q\ \wedge\ r)\]
  \item
    {\bf Дистрибутивен закон}
    \[p\ \wedge\ (q \vee r) \equiv (p\ \wedge q)\vee (p\ \wedge\ r)\]
    \[p\vee (q\ \wedge\ r) \equiv (p\vee q)\ \wedge\ (p\vee r)\]
  \item
    {\bf Закони на де Морган}
    \[\neg(p \wedge q) \equiv (\neg p \vee \neg q)\]
    \[\neg(p\vee q) \equiv (\neg p \wedge \neg q)\]
  \item
    {\bf Закон за контрапозицията}
    \[p\rightarrow q \equiv \neg q \rightarrow \neg p\]
  \item
    {\bf Обобщен закон за контрапозицията}
    \[(p \wedge q)\rightarrow r \equiv (p \wedge \neg r) \rightarrow \neg q\]
  \item
    {\bf Закон за изключеното трето}
    \[p\vee \neg p \equiv {\mathbf 1}\]
  \item
    {\bf Закон за силогизма (транзитивност)}
    \[[(p\rightarrow q)\ \wedge\ (q\rightarrow r)] \rightarrow (p\rightarrow r) \equiv {\mathbf 1}\]
\end{enumerate}

Лесно се проверява с таблиците за истинност, че законите са валидни.


\section{Предикатно смятане}

\section{Доказателства на твърдения}

\subsection*{Допускане на противното}

Да приемем, че искаме да докажем, че свойството $P(x)$
е вярно за всяко естествено число.
Един начин да направим това е следният:
\begin{itemize}
\item 
  Допускаме, че съществува елемент $n$, за който $\neg P(n)$.
\item
  Използвайки, че $\neg P(n)$ правим извод, от който следва факт, за който знаем, че винаги е лъжа.
  Това означава, че доказваме следното твърдение
  \[\exists x \neg P(x) \rightarrow \mathbf{0}.\]
\item
  Тогава можем да заключим, че $\forall x P(x)$, защото имаме следния извод:
  \begin{prooftree}
    \AxiomC{$\exists x \neg P(x) \rightarrow \mathbf{0}$}
    \UnaryInfC{$\mathbf{1} \rightarrow \neg \exists x \neg P(x)$}
    \UnaryInfC{$\neg \exists x \neg P(x)$}
    \UnaryInfC{$\forall x P(x)$}
  \end{prooftree}
\end{itemize}

Ще илюстрираме този метод като решим няколко прости задачи.

\begin{problem}
  \label{prob:even-number-square}
  За всяко $a \in \Int$, ако $a^2$ е четно, то $a$ е четно.
\end{problem}
\begin{proof}
  Ние искаме да докажем твърдението $P$, където:
  \[P \equiv (\forall a\in\Z)[a^2\mbox{ е четно}\ \rightarrow\ a\mbox{ е четно}].\]
  \marginpar{$\neg (\forall x)(A(x) \rightarrow B(x))$ е еквивалентно на $(\exists x)(A(x) \wedge \neg B(x))$}
  Да допуснем противното, т.е. изпълнено е $\neg P$. Лесно се вижда, че
  \[\neg P \iff (\exists a\in\Z)[a^2\mbox{ е четно}\ \wedge\ a\mbox{ не е четно}].\]
  Да вземем едно такова нечетно $a$, за което $a^2$ е четно.
  Това означава, че $a = 2k+1$, за някое $k \in \Z$,
  и \[a^2 = (2k+1)^2 = 4k^2 + 4k + 1,\]
  което очевидно е нечетно число.
  Но ние допуснахме, че $a^2$ е четно.
  Така достигаме до противоречие, следователно нашето допускане е грешно 
  и 
  \[(\forall a\in\Z)[a^2\mbox{ е четно}\ \rightarrow\ a\mbox{ е четно}].\]
\end{proof}

\begin{problem}
  Докажете, $\sqrt{2}$ {\bf не} е рационално число.
\end{problem}
\begin{proof}
  Да допуснем, че $\sqrt{2}$ е рационално число. Тогава  съществуват $a,b \in \Z$, такива че
  \[\sqrt{2} = \frac{a}{b}.\]
  Без ограничение, можем да приемем, че $a$ и $b$ са естествени числа,
  които нямат общи делители, т.е. не можем да съкратим дробта $\frac{a}{b}$.
  Получаваме, че \[2b^2 = a^2.\]
  Тогава $a^2$ е четно число и от Задача \ref{prob:even-number-square}, $a$ е четно число.
  Нека $a = 2k$. Получаваме, че
  \[2b^2 = 4k^2,\]
  от което следва, че
  \[b^2 = 2k^2.\]
  Това означава, че $b$ също е четно число, $b = 2n$, за някое $n \in \Z$.
  Следователно, $a$ и $b$ са четни числа и имат общ делител $2$,
  което е противоречие с нашето допускане, че $a$ и $b$ нямат общи делители.
  Така достигаме до противоречие.
  Накрая заключаваме, че $\sqrt{2}$ не е рационално число.
\end{proof}


\subsection*{Индукция върху естествените числа}

\marginpar{Да напомним, че естествените числа са $\Nat = \{0,1,2,\dots\}$}
Доказателството с индукция по $\Nat$ представлява следната схема:
\begin{prooftree}
  \AxiomC{$P(0)$}
  \AxiomC{$(\forall x\in\Nat)[P(x)\rightarrow P(x+1)]$}
  \BinaryInfC{$(\forall x\in\Nat) P(x)$}
\end{prooftree}

Това означава, че ако искаме да докажем, че свойството $P(x)$ е вярно за всяко естествено число $x$,
то трябва да докажем първо, че е изпълнено $P(0)$ и след това, за произволно естествено число $x$, ако $P(x)$ вярно, то също така е вярно $P(x+1)$.

\begin{problem}
  \label{prob:number-prod-prime}  
  Всяко естествено число $n \geq 2$ може да се запише като произведение на прости числа.
\end{problem}
\begin{proof}
  Доказателството протича с индукция по $n \geq 2$.
  \begin{enumerate}[a)]
  \item 
    За $n = 2$  е ясно.
  \item
    Ако $n+1$ е просто число, то всичко е ясно.
    Ако $n+1$ е съставно, то \[n + 1 = n_1\cdot n_2.\]
    Тогава $n_1 = p^{n_1}_1\cdots p^{n_k}_k$ и $n_2 = q^{m_1}_1\cdots q^{m_r}_r$,
    където $p_1,\dots,p_k$ и $q_1,\dots,q_r$ са прости числа.
    Тогава е ясно, че $n+1$ също е произведение на прости числа.
  \end{enumerate}
\end{proof}

\begin{problem}
  Докажете, че за всяко $n$, 
  \[\sum^n_{i=0} 2^i = 2^{n+1} - 1.\]
\end{problem}
\begin{proof}
  Доказателството протича с индукция по $n$.
  \begin{itemize}
  \item 
    За $n = 0$, $\sum^0_{i=0}2^i = 1 = 2^{1} - 1$.
  \item
    Нека твърдението е вярно за $n$.
    Ще докажем, че твърдението е вярно за $n+1$.
    \begin{align*}
      \sum^{n+1}_{i=0} 2^i & = \sum^{n}_{i=0}2^i + 2^{n+1}\\
      & = 2^{n+1} - 1 + 2^{n+1} & (\text{от И.П.})\\
      & = 2.2^{n+1} - 1 \\
      & = 2^{(n+1)+1} - 1.
    \end{align*}
  \end{itemize}
\end{proof}

%\subsection*{Пълна индукция върху естествените числа}

\section{Множества, релации, функции}

\subsection*{Основни операции върху множества}

Ще разгледаме няколко операции върху произволни множества $A$ и $B$.
\begin{itemize}
\item
  {\bf Сечение}
  \[A\cap B = \{x\ \mid\ x\in A\ \wedge\ x\in B\}.\]
  % Казано по-формално, $A\cap B$ е множеството, за което е изпълнена формулата
  % \[(\forall x)[x \in A\cap B \iff (x\in A\ \wedge\ x \in B)].\]
  % Примери:
  % \begin{itemize}
  % \item
  %   $A \cap A = A$, за всяко множество $A$.
  % \item
  %   $A \cap \emptyset = \emptyset$, за всяко множество $A$.
  % \item
  %   $\{1,\emptyset,\{\emptyset\}\} \cap \{\emptyset\} = \{\emptyset\}$.
  %   \item
  %     $\{1,2,\{1,2\}\} \cap \{1,\{1\}\} = \{1\}$.
  %   \end{itemize}
  \item
    {\bf Обединение}
    \[A\cup B = \{x\ \mid x\in A\ \vee\ x\in B\}.\]
    % $A\cup B$ е множеството, за което е изпълнена формулата
    % \[(\forall x)[x \in A\cup B \iff (x\in A\ \vee\ x \in B)].\]
    % Примери:
    % \begin{itemize}
    % \item
    %   $A \cup A = A$, за всяко множество $A$.
    % \item 
    %   $A \cup \emptyset = A$, за всяко множество $A$.
    % \item
    %   $\{1,2,\emptyset\} \cup \{1,2,\{\emptyset\}\} = \{1,2,\emptyset,\{\emptyset\}\}$.
    % \item
    %   $\{1,2,\{1,2\}\} \cup \{1,\{1\}\} = \{1,2,\{1\},\{1,2\}\}$.
    % \end{itemize}
  \item
    {\bf Разлика}
    \[A\setminus B = \{x\ \mid\ x\in A\ \wedge\ x\not\in B\}.\]
    % $A\setminus B$ е множеството, за което е изпълнена формулата
    % \[(\forall x)[x \in A\setminus B \iff (x\in A\ \wedge\ x \not\in B)].\]
    % Примери:
    % \begin{itemize}
    % \item
    %   $A \setminus A = \emptyset$, за всяко множество $A$.
    % \item 
    %   $A \setminus \emptyset = A$, за всяко множество $A$.
    % \item 
    %   $\emptyset \setminus A = \emptyset$, за всяко множество $A$.
    % \item
    %   $\{1,2,\emptyset\} \setminus \{1,2,\{\emptyset\}\} = \{\emptyset\}$.
    % \item
    %   $\{1,2,\{1,2\}\} \setminus \{1,\{1\}\} = \{2,\{1,2\}\}$.
    % \end{itemize}
  % \item
  %   {\bf Симетрична разлика}
  %   \[A\triangle B = (A\backslash B)\cup (B\backslash A).\]
  %   % $A\triangle B$ е множеството, за което е изпълнена формулата
  %   % \[(\forall x)[x \in A\triangle B \iff [(x\in A\ \wedge\ x \not\in B) \vee (x \in B\ \wedge\ x\not\in A)]].\]
  %   Примери:
  %   \begin{itemize}
  %   \item 
  %     $A \triangle \emptyset = A$, за всяко множество $A$.
  %   \item
  %     $A \triangle A = \emptyset$, за всяко множество $A$.
  %   \item
  %     $A\triangle B = B \triangle A$, за всеки две множества $A$ и $B$.
  %   \item
  %     $\{1,2,\emptyset\} \triangle \{1,2,\{\emptyset\}\} = \{\emptyset\} \cup \{\{\emptyset\}\} = \{\emptyset,\{\emptyset\}\}$.
  %   \item
  %     $\{1,2,\{1,2\}\} \triangle \{1,\{1\}\} = \{2,\{1,2\}\} \cup \{\{1\}\} = \{2,\{1\},\{1,2\}\}$.
  %   \end{itemize}
  \item
    {\bf Степенно множество}
    \[\Ps(A) = \{x\mid x\subseteq A\}.\]
    % $\Ps(A)$ е множеството, за което е изпълнена формулата
    % \[(\forall x)[x \in \Ps(A) \iff (\forall y)[y\in x\rightarrow y \in A]].\]
    Примери:
    \begin{itemize}
    \item 
      $\Ps(\emptyset) = \{\emptyset\}$.
    \item
      $\Ps(\{\emptyset\}) = \{\emptyset,\{\emptyset\}\}$.
    \item
      $\Ps(\{\emptyset,\{\emptyset\}\}) = \{\emptyset,\{\emptyset\},\{\{\emptyset\}\},\{\emptyset,\{\emptyset\}\}\}$.
    \item
      $\Ps(\{1,2\}) = \{\emptyset,\{1\},\{2\},\{1,2\}\}$.
    \end{itemize}
  \end{itemize}
  Нека имаме редица от множества $\{A_1,A_2,\dots,A_n\}$.
  Тогава имаме следните операции:
  \begin{itemize}
  \item
    {\bf Обединение на редица от множества}
    \[\bigcup^{n}_{i=1} A_i = \{x \mid \exists i (1\leq i\leq n\ \&\ x\in A_i)\}.\]
    % \[(\forall x)[x \in \bigcup^n_{i=1}A_i \iff (\exists i)[1 \leq i \leq n\ \wedge\ x \in A_i]].\]
  \item
    {\bf Сечение на редица от множества}
    \[\bigcap^{n}_{i=1} A_i = \{x \mid \forall i (1\leq i\leq n \rightarrow x\in A_i)\}.\]
    % \[(\forall x)[x \in \bigcap^n_{i=1}A_i \iff (\forall i)[1 \leq i \leq n\ \rightarrow\ x \in A_i]].\]
  \end{itemize}

% \begin{example}
%   Нека $A = \{x\in\Nat\mid x > 1\}$ и $B = \{x\in\Nat\mid x>3\}$. Тогава :
%   \begin{itemize}
%     \item
%       $A\cap B = \{x\in\Nat\mid x > 3\}$,
%     \item
%       $A\cup B = \{x\in\Nat\mid x > 1\}$,
%     \item
%       $A\setminus B = \{x\in\Nat\mid 1<x\leq 3\}$,
%     \item
%       $B\setminus A = \emptyset$,
%     % \item
%     %   $A\triangle B = \{x\in\Nat\mid 1<x\leq 3\}$
%     \end{itemize}
% \end{example}


\begin{problem}
  Проверете верни ли са свойствата:
  \begin{enumerate}[a)]
  \item
    $A\subseteq B \iff A\setminus B = \emptyset \iff A\cup B = B \iff A\cap B = A$;
  \item
    $A\setminus \emptyset = A$, $\emptyset\setminus A=\emptyset$, $A\setminus B = B\setminus A$.
  \item
    $A\cap (B\cup A) = A \cap B$;
  \item
    $A\cup(B\cap C) = (A\cup B)\cap(A\cup C)$ и $A \cap (B \cup C) = (A \cup B) \cap (A \cup C)$;
  % \item
  %   $C\subseteq A\ \&\ C\subseteq B \rightarrow C\subseteq A\cap B$;
  % \item
  %   $A\subseteq C\ \&\ B\subseteq C \rightarrow A\cup B\subseteq C$;
  \item
    $A\backslash B = A \iff A\cap B = \emptyset$;
  \item
    $A\backslash B = A\backslash (A\cap B)$ и $A\backslash B = A\backslash (A\cup B)$;
  \item
    $(A\cup B)\setminus C = (A\setminus C) \cup (B\setminus C)$;
  \item
    \marginpar{Не е вярно!}
    $A\setminus (B\setminus C) = (A\setminus B)\setminus C$;
  \item
    \marginpar{Закони на Де Морган}
    $C\setminus (A\cup B) = (C\backslash A)\cap(C\backslash B)$ и $C \backslash (A\cap B) = (C\backslash A)\cup(C\backslash B)$
  \item
    $C\backslash(\bigcup^{n}_{i=1} A_i) = \bigcap^{n}_{i=1} (C\backslash A_i)$ и $C \backslash(\bigcap^{n}_{i=1} A_i) = \bigcup^{n}_{i=1} (C\backslash A_i)$;
  \item
    $(A\backslash B)\backslash C = (A\backslash C)\backslash(B \backslash C)$ и $A\backslash (B\backslash C) = (A\backslash B) \cup (A\cap C)$;
  \item
    $A\subseteq B \Rightarrow \Ps(A) \subseteq \Ps(B)$;
  \item
    \marginpar{$X \subseteq A\cup B \stackrel{?}{\Rightarrow} X\subseteq A \vee X \subseteq B$}
    $\Ps(A\cap B) = \Ps(A) \cap \Ps(B)$ и $\Ps(A\cup B) = \Ps(A) \cup \Ps(B)$;
  \end{enumerate}
\end{problem}

За да дадем определение на понятието релация, трябва първо 
да въведем понятието декартово произведение на множества,
което пък от своя страна се основава на понятието наредена двойка.

\subsection*{Наредена двойка}
\index{наредена двойка}
За два елемента $a$ и $b$ въвеждаме опрецията {\bf наредена двойка} $\pair{a,b}$.
Наредената двойка $\pair{a,b}$ има следното характеристичното свойство:
\[a_1 = a_2\ \wedge\ b_1 = b_2\ \iff\ \pair{a_1,b_1} = \pair{a_2,b_2}.\]
Понятието наредена двойка може да се дефинира по много начини, стига да изпълнява харектеристичното свойство.
Ето примери как това може да стане:
\begin{enumerate}[1)]
\item
  \marginpar{Norbert Wiener (1914)}
  Първото теоретико-множествено определение на понятието наредена двойка е
  дадено от Норберт Винер:
  \[\pair{a,b} \df \{\{\{a\},\emptyset\},\{\{b\}\}\}.\]
\item
  \marginpar{Kazimierz Kuratowski (1921)}
  Определението на Куратовски се приема за ,,стандартно'' в наши дни:
  \[\pair{a,b} \df \{\{a\},\{a,b\}\}.\]
\end{enumerate}

\begin{problem}
  Докажете, че горните дефиниции наистина изпълняват харектеристичното свойство за наредени двойки.
\end{problem}

\begin{dfn}
  \marginpar{Пример за индуктивна (рекурсивна) дефиниция}
  Сега можем, за всяко естествено число $n \geq 1$,
  да въведем понятието наредена $n$-орка $\pair{a_1,\dots,a_n}$:
  \begin{align*}
    & \pair{a_1} \df a_1,\\
    & \pair{a_1,a_2,\dots,a_n} \df \pair{a_1,\pair{a_2,\dots,a_n}}
  \end{align*}
\end{dfn}

Оттук нататък ще считаме, че имаме операцията наредена $n$-орка, без да се интересуваме от нейната формална дефиниция.
 
\subsection*{Декартово произведение}
\marginpar{На англ. cartesian product}
\index{декартово произведение}
\marginpar{Считаме, че $(A\times B)\times C = A\times (B\times C) = A\times B \times C$}

За две множества $A$ и $B$, определяме тяхното декартово произведение като
\[A\times B = \{\pair{a,b}\mid a\in A\ \&\ b\in B\}.\]
За краен брой множества $A_1,A_2,\dots,A_n$, определяме
\[A_1\times A_2\times\cdots\times A_n = \{\pair{a_1,a_2,\dots,a_n}\mid a_1 \in A_1\ \&\ a_2\in A_2\ \&\ \dots\ \&\ a_n \in A_n\}.\]

\begin{problem}
  Проверете, че:
  \begin{enumerate}[a)]
  \item
    $A\times(B\cup C) = (A\times B) \cup (A\times C)$.
  \item
    $(A\cup B)\times C = (A\times C)\cup (B\times C)$.
  \item 
    $A\times(B\cap C) = (A\times B) \cap (A\times C)$.
  \item
    $(A \cap B)\times C = (A \times C)\cap(B\times C)$.
  \item 
    $A\times(B\setminus C) = (A\times B) \setminus (A\times C)$.
  \item
    $(A\setminus B)\times C = (A\times C)\setminus (B\times C)$.
  \end{enumerate}
\end{problem}

\subsection*{Основни видове бинарни релации}
% \marginpar{Бинарни релации}

Подмножествата $R$ от вида $R \subseteq A\times A\times\cdots\times A$ се наричат релации.
Релациите от вида $R\ \subseteq\ A\times A$ са важен клас, който ще срещаме често.
Да разгледаме няколко основни видове релации от този клас:
\begin{enumerate}[I)]
\item
  {\bf рефликсивна}, ако
  \[(\forall x\in A)[\pair{x,x}\in R].\]
  Например, релацията $\leq\ \subseteq\ \Nat\times\Nat$ е рефлексивна, защото
  \[(\forall x\in \Nat)[x \leq x].\]
% \item
%   {\bf антирефлексивна}, ако
%   \[(\forall x\in A)[\pair{x,x}\not\in R].\]
%   Например, релацията $<\ \subseteq\ \Nat\times\Nat$ е антирефлексивна, защото
%   \[(\forall x\in \Nat)[x \not< x].\]
\item
  {\bf транзитивна}, ако
  \[(\forall x,y,z\in A)[\pair{x,y}\in R\ \&\ \pair{y,z}\in R \rightarrow \pair{x,z}\in R].\]
  Например, релацията $\leq\ \subseteq\ \Nat\times\Nat$ е транзитивна, защото
  \[(\forall x,y,z\in A)[x \leq y\ \&\ y \leq z\ \rightarrow\ x\leq z].\]
\item
  {\bf симетрична}, ако
  \[(\forall x,y\in A)[\pair{x,y}\in R \rightarrow \pair{y,x}\in R].\]
  Например, релацията $=\ \subseteq\ \Nat\times\Nat$ е рефлексивна, защото
  \[(\forall x,y\in \Nat)[x = y\ \rightarrow\ y = x].\]
\item
  {\bf антисиметрична}, ако
  \[(\forall x,y\in A)[\pair{x,y}\in R\ \&\ \pair{y,x}\in R \rightarrow x = y].\]
  Например, релацията $\leq\ \subseteq\ \Nat\times\Nat$ е антисиметрична, защото
  \[(\forall x,y,z\in A)[x \leq y\ \&\ y \leq x\ \rightarrow\ x = y].\]
% \item
%   {\bf асиметрична}, ако
%   \[(\forall x,y)[\pair{x,y}\in R \rightarrow \pair{y,x}\not\in R].\]
%   Например, релацията $\leq\ \subseteq\ \Nat\times\Nat$ е асиметрична, защото
%   \[(\forall x,y\in \Nat)[x < y\ \rightarrow\ y \not< x].\]
\end{enumerate}

% \begin{remark}
%   Добре е да запомните как се наричат тези основни видове релации,
%   защото ще ги използваме често.
% \end{remark}

% \begin{example}
%   Да обобщим примерите от по-горе.
%   \begin{enumerate}[a)]
%   \item
%     Релацията $\leq\ \subseteq\ \Nat\times\Nat$ е рефлексивна, транзитивна и антисиметрична.
%   \item
%     Релацията $<\ \subseteq\ \Nat\times\Nat$ е антирефлексивна, транзитивна и асиметрична.
%   \item
%     Релацията $=\ \subseteq\ \Nat\times\Nat$ е рефлексивна, транзитивна и симетрична.
%   \end{enumerate}
% \end{example}

\begin{itemize}
\item
  Една бинарна релация $R$ над множеството $A$ се нарича {\bf релация на еквивалентност}, 
  ако $R$ е рефлексивна, транзитивна и симетрична.
\item 
  За всеки елемент $a \in A$, определяме неговия 
  {\bf клас на еквивалентност} относно релацията на еквивалентност $R$ по следния начин:
  \[[a]_R \df \{b\in A \mid \pair{a,b} \in R\}.\]
\end{itemize}

\begin{remark}
  Лесно се съобразява, че за всеки два елемента $a, b\in A$,
  \[\pair{a,b} \in R \iff [a]_R = [b]_R.\]
\end{remark}

\begin{example}
  За всяко естествено число $n\geq 2$, дефинираме релацията $R_n$ като
  \[\pair{x,y}\in R_n \iff x \equiv y\ (\bmod\ n).\]
  Ясно е, че $R_n$ са релации на еквивалентност.
\end{example}


\subsection*{Операции върху бинарни релации}

\begin{enumerate}[I)]
\item
  {\bf Композиция} на две релации $R \subseteq B\times C$ и $P \subseteq A\times B$ е релацията $R\circ P \subseteq A\times C$,
  определена като:
  \[R\circ P \df \{\pair{a,c} \in A\times C \mid (\exists b \in B)[\pair{a,b}\in P\ \&\ \pair{b,c} \in R]\}.\]
\item
  {\bf Обръщане} на релацията $R \subseteq A\times B$ е релацията $R^{-1}\subseteq B\times A$, 
  определена като:
  \[R^{-1} \df \{\pair{x,y} \in B\times A \mid \pair{y,x} \in R\}.\]
\item
  \marginpar{Очевидно е, че $P$ е рефлексивна релация, дори ако $R$ не е.}
  {\bf Рефлексивно затваряне} на релацията $R \subseteq A\times A$ е релацията
  \[P \df R \cup \{\pair{a,a}\mid a \in A\}.\]
\item
  {\bf Итерация} на релацията $R \subseteq A\times A$ дефинираме като за всяко естествено число $n$,
  дефинираме релацията $R^n$ по следния начин:
  \marginpar{Лесно се вижда, че  $R^1 = R$}
  \begin{align*}
    R^0 & \df \{\pair{a,a} \mid a \in A\}\\
    R^{n+1} & \df R^n \circ R.
  \end{align*}
\item
  \marginpar{\ding{45} Проверете, че $R^+$ е транзитивна релация!}
  {\bf Транзитивно затваряне} на $R \subseteq A\times A$ е релацията
  \[R^+ \df \bigcup_{n\geq 1} R^n.\]
\end{enumerate}

За дадена релация $R$, с $R^\star$ ще означаваме нейното рефлексивно и транзитивно затваряне.
От дефинициите е ясно, че \[R^\star = \bigcup_{n\geq 0} R^n.\]

\subsection*{Видове функции}

Функцията $f:A \to B$ е:
\begin{itemize}
\item
  \marginpar{(или $f$ е {\bf обратима})}
  {\bf инекция}\index{функция!инекция}, ако 
  \[(\forall a_1,a_2\in A)[a_1\neq a_2 \rightarrow f(a_1)\neq f(a_2)],\]
  или еквивалентно,
  \[(\forall a_1,a_2\in A)[f(a_1) = f(a_2) \rightarrow a_1 = a_2].\]
\item
  \marginpar{(или $f$ е {\bf върху} $B$)}
  {\bf сюрекция}\index{функция!сюрекция}, ако 
  \[(\forall b\in B)(\exists a\in A)[f(a) = b].\]
\item
  {\bf биекция}\index{функция!биекция}, ако е инекция и сюрекция.
\end{itemize}

\begin{problem}
  \marginpar{Канторово кодиране. Най-добре се вижда като се нарисува таблица}
  Докажете, че $f: \Nat \times \Nat\rightarrow \Nat$ е биекция, където
  \[f(x, y) = \frac{(x+y)(x+y+1)}{2} + x.\]
\end{problem}

\section{Азбуки, думи, езици}

\subsection*{Основни понятия}

\begin{itemize}
\item 
  \index{азбука}
  {\bf Азбука} ще наричаме всяко крайно множество, като обикновено ще я означаваме със $\Sigma$.
  \marginpar{Често ще използваме буквите $a$, $b$, $c$ за да означаваме букви.}
  Елементите на азбуката $\Sigma$ ще наричаме {\bf букви} или символи.
\item
  \index{дума}
  {\bf Дума} над азбуката $\Sigma$ е произволна крайна редица от елементи на $\Sigma$.
  Например, за $\Sigma = \{a,b\}$, $aababba$ е дума над $\Sigma$ с дължина $7$.
  С $\abs{\alpha}$ ще означаваме дължината на думата $\alpha$.
  \marginpar{Обикновено ще означаваме думите с $\alpha$, $\beta$, $\gamma$, $\omega$.}
\item
  Обърнете внимание, че имаме единствена дума с дължина $0$.
  Тази дума ще означаваме с $\varepsilon$ и ще я наричаме {\bf празната дума},
  т.е. $\abs{\varepsilon} = 0$.
\item
  С $a^n$ ще означаваме думата съставена от $n$ $a$-та.
  Формалната индуктивна дефиниция е следната:
  \begin{align*}
    a^0 & \df \varepsilon,\\
    a^{n+1} & \df a^na.
  \end{align*}
\item
  Множеството от всички думи над азбуката $\Sigma$ ще означаваме със $\Sigma^\star$.
  Например, за $\Sigma = \{a,b\}$,
  \[\Sigma^\star = \{\varepsilon,a,b,aa,ab,ba,bb,aaa,aab,\dots\}.\]
  Обърнете внимание, че $\emptyset^\star = \{\varepsilon\}$.
% \item
%   {\bf Лексикографска наредба}
\end{itemize}

\subsection*{Операции върху думи}

\begin{itemize}
\item 
  \index{конкатенация}
  Операцията {\bf конкатенация} взима две думи $\alpha$ и $\beta$ и образува 
  новата дума $\alpha\cdot\beta$ като слепва двете думи.
  Например $aba\cdot bb = ababb$.
  Обърнете внимание, че в общия 
  случай $\alpha\cdot\beta \neq \beta\cdot\alpha$. 
  \marginpar{Често ще пишем $\alpha\beta$ вместо $\alpha\cdot\beta$}
  Можем да дадем формална индуктивна дефиниция на операцията конкатенация по
  дължината на думата $\beta$.
  \begin{itemize}
  \item 
    Ако $\abs{\beta} = 0$, то $\beta = \varepsilon$.
    Тогава $\alpha\cdot \varepsilon \df \alpha$.
  \item
    Ако $\abs{\beta} = n+1$, то $\beta = \gamma b$, $\abs{\gamma} = n$.
    Тогава $\alpha\cdot\beta \df (\alpha\cdot\gamma)b$.
  \end{itemize}
\item
  Друга често срещана операция върху думи е {\bf обръщането} на дума.
  Дефинираме думата $\alpha^R$ като обръщането на $\alpha$ по следния начин.
  \begin{itemize}
  \item 
    Ако $\abs{\alpha} = 0$, то $\alpha = \varepsilon$ и $\alpha^R \df \varepsilon$.
  \item
    Ако $\abs{\alpha} = n+1$, то $\alpha = a\beta$, където $\abs{\beta} = n$.
    Тогава $\alpha^R \df (\beta^R)a$.
  \end{itemize}
  Например, $reverse^R = esrever$.
\item
  \index{дума!префикс}
  \index{дума!суфикс}
  Казваме, че думата $\alpha$ е {\bf префикс} на думата $\beta$,
  ако съществува дума $\gamma$, такава че $\beta = \alpha\cdot\gamma$.
  $\alpha$ е {\bf суфикс} на $\beta$, ако $\beta = \gamma\cdot\alpha$, за някоя дума $\gamma$.
\item
  \marginpar{Обърнете внимание, че $\emptyset\cdot A = A\cdot\emptyset = \emptyset$}
  \marginpar{Също така, $\{\varepsilon\}\cdot A = A\cdot\{\varepsilon\} = A$}
  Нека $A$ и $B$ са множества от думи.
  Дефинираме конкатенацията на $A$ и $B$ като
  \[A\cdot B \df \{\alpha\cdot\beta \mid \alpha\in A\ \&\ \beta \in B\}.\]
\item
  Сега за едно множество от думи $A$, дефинираме $A^n$ индуктивно:
  \begin{align*}
    A^0 & \df \{\varepsilon\},\\
    A^{n+1} & \df A^n \cdot A.
  \end{align*}
  Ако $A = \{ab, ba\}$, то
  $A^0 = \{\varepsilon\}$, $A^1 = A$, $A^2 = \{abab, abba, baba, baab\}$.
  Ако $A = \{a,b\}$, то $A^n = \{\alpha \in \{a,b\}^\star \mid \abs{\alpha} = n\}$.
\item
  За едно множеството от думи $A$, дефинираме:
  \marginpar{Операцията $\star$ е известна като звезда на Клини}
  \begin{align*}
    A^\star & \df \bigcup_{n\geq 0} A^n\\
    & = A^0 \cup A^1 \cup A^2 \cup A^3 \cup \dots\\
    A^+ & \df A\cdot A^\star.
  \end{align*}
\end{itemize}

\begin{problem}
  Проверете:
  \begin{enumerate}[a)]
  \item 
    операцията конкатенация е {\em асоциативна}, т.е. за всеки три думи $\alpha$, $\beta$, $\gamma$,
    \[(\alpha\cdot\beta)\cdot\gamma = \alpha\cdot(\beta\cdot\gamma);\]
  \item
    за множествата от думи $A$, $B$ и $C$,
    \[(A\cdot B)\cdot C = A\cdot (B\cdot C);\]
  \item
    $\{\varepsilon\}^\star = \varepsilon$;
  \item
    за произволно множество от думи $A$,
    $A^\star = A^\star\cdot A^\star$ и $(A^\star)^\star = A^\star$;
  \item
    за произволни букви $a$ и $b$,
    $\{a,b\}^\star = \{a\}^\star\cdot(\{b\}\cdot\{a\}^\star)^\star$.
  \end{enumerate}
\end{problem}

\begin{problem}
  Докажете, че за всеки две думи $\alpha$ и $\beta$ е изпълено:
  \begin{enumerate}[a)]
  \item 
    $(\alpha\cdot\beta)^R = \beta^R\cdot\alpha^R$;
  \item
    $\alpha$ е префикс на $\beta$ точно тогава, когато $\alpha^R$ е суфикс на $\beta^R$;
  \item
    $(\alpha^R)^R = \alpha$;
  \item
    $(\alpha^n)^R = (\alpha^R)^n$, за всяко $n \geq 0$.
  \end{enumerate}
\end{problem}

% \section*{Библиография}

% Повечето книги започват с уводна глава за множества, релации и езици.
% Например, \cite{rosen} и \cite{papadimitriou}.
% \begin{itemize}
% \item 
%   Глава 1 от \cite{rosen}.
% \item
%   Глава 1 от \cite{papadimitriou}.
% \item
%   На практика следваме Глава 2 от \cite{kozen} в описанието на думи и азбуки.
% \end{itemize}

%%% Local Variables: 
%%% mode: latex
%%% TeX-master: "EAI"
%%% End: 


\chapter{Регулярни езици и автомати}

\section{Автоматни езици}

% Един от източниците е втора и трета глава от книгата на Сипсер, \cite{sipser}.
% Друг основен източник е книгата на Пападимитриу и Люис, \cite{papadimitriou}.
%По Сипсер, стр. 35
\begin{dfn}
  Краен автомат е петорка $\A = \FA$, където
  \begin{enumerate}[1)]
  \item
    $Q$ е крайно множество от състояния;
  \item
    $\Sigma$ е азбука;
  \item
    % \marginpar{Тук нямаме $\varepsilon$-преходи}
    % \marginpar{(Sipser разглежда тотални $\delta$ функции)}
    $\delta:Q\times\Sigma\to Q$ е (частична) функция на преходите;
  \item
    $s\in Q$ е начално състояние;
  \item
    $F\subseteq Q$ е множеството от финални състояния, $F \neq \emptyset$.
  \end{enumerate}
\end{dfn}

\index{автомат!детерминиран}\index{автомат!тотален детерминиран}
Ако функцията на преходите $\delta$ е тотална функция, то казваме, 
че автоматът $\A$ е {\bf тотален}. Това означава, че за всяка двойка $(q,a) \in \Sigma\times Q$,
същесествува $q' \in Q$, за което $\delta(q,a) = q'$.

Нека имаме една дума $\alpha \in \Sigma^\star$, $\alpha = a_1a_2\cdots a_n$.
Казваме, че $\alpha$ се {\bf разпознава} от автомата $\A$, ако
съществува редица от състояния $q_0,q_1,q_2,\dots,q_n$, такива че:
\begin{itemize}
\item
  $q_0 = s$, началното състояние на автомата;
\item
  $\delta(q_i,a_{i+1}) = q_{i+1}$, за всяко $i = 0, \dots, n-1$;
\item
  $q_n \in F$.
\end{itemize}

% Алтернативен запис е следния:
% $(q,a\beta) \vdash (p,\beta)$, ако $\delta(q,a) = p$.
% $(q,\alpha\beta) \vdash^\star (p,\beta)$, ако $\delta^\star(q,\alpha) = p$.
% Тогава една дума $\alpha$ се разпознава от автомата, ако $(s,\alpha) \vdash^\star (p,\varepsilon)$ и $p \in F$.

Казваме, че $\A$ {\bf разпознава} езика $L$, ако $\A$ разпознава точно думите от $L$, т.е.
$L = \{\alpha \in \Sigma^\star \mid \A\mbox{ разпознава }\alpha\}$.
Обикновено означаваме езика, който се разпознава от даден автомат $\A$ с $\L(\A)$.
\index{език!автоматен}
В такъв случай ще казваме, че езикът $L$ е {\bf автоматен}.

При дадена (частична) функция на преходите $\delta$,
често е удобно да разглеждаме (частичната) функция $\delta^\star:Q\times\Sigma^\star \to Q$, кято е дефинирана по следния начин:
\marginpar{Това е пример за индуктивна (рекурсивна) дефиниция по дължината на думата $\alpha$}
\begin{itemize}
\item 
  $\delta^\star(q,\varepsilon) = q$, за всяко $q\in Q$;
\item
  $\delta^\star(q,\beta a) = \delta(\delta^\star(q,\beta), a)$, за всяко $q\in Q$, всяко $a\in\Sigma$ и $\beta\in\Sigma^\star$.
\end{itemize}
Тогава една дума $\alpha$ се {\em разпознава} от автомата $\A$ точно тогава, когато $\delta^\star(s,\alpha) \in F$.
Оттук следва, че
\[\L(\A) = \{\alpha\in\Sigma^\star \mid \delta^\star(s,\alpha) \in F\}.\]

\begin{prop}
  \label{pr:delta-star}
  $(\forall q\in Q)(\forall\alpha,\beta\in\Sigma^\star)[\delta^\star(q,\alpha\beta) = \delta^\star(\delta^\star(q,\alpha),\beta)]$.
  % В частност, 
  % $(\forall q\in Q)(\forall\alpha\in\Sigma^\star)(\forall b\in\Sigma)[\delta^\star(q,\alpha b) = \delta(\delta^\star(q,\alpha),b)]$.
\end{prop}
\begin{hint}
  \marginpar{\ding{45} Напишете доказателството!}
  Индукция по дължината на $\beta$.
\end{hint}

\index{моментно описание}
{\em Моментното описание} на изчисление с краен автомат представлява двойка от вида $(q,\alpha) \in Q\times\Sigma^\star$,
т.е. автоматът се намира в състояние $q$, а думата, която остава да се прочете е $\alpha$.
Удобно е да въведем бинарната релация $\vdash_\A$ над $Q\times\Sigma^\star$,
която ще ни казва как моментното описание на автомата $\A$ се променя след изпълнение на една стъпка:
\[(q,x\alpha) \vdash_\A (p,\alpha), \text{ ако } \delta(q,x) = p.\]
Рефлексивното и транзитивно затваряне на $\vdash_\A$ ще означаваме с $\vdash^\star_\A$.
Получаваме, че 
\[\L(\A) = \{\alpha\in\Sigma^\star \mid (s,\alpha) \vdash^\star_\A(p,\varepsilon)\ \&\ p \in F\}.\]

Нашата дефиниция на автомат позволява $\delta$ да бъде частична функция, т.е.
може да има $q\in Q$ и $a\in\Sigma$, за които $\delta(q,a)$ не е дефинирана.
Следващото твърдение ни казва, че ние съвсем спокойно можем да разглеждаме автомати
само с тотални функции на преходите  $\delta$.
\begin{prop}
  За всеки краен автомат $\A$, съществува {\em тотален} краен автомат $\A'$,
  за който $\L(\A) = \L(\A')$.
\end{prop}
\begin{proof}
  Нека $\A = \FA$.
  Дефинираме тоталния автомат 
  \[\A' = \pair{Q\cup\{q_e\}, \Sigma, \delta', s, F},\]
  като за всеки преход $(q,a)$, за който $\delta$ не е дефинирана, 
  дефинираме $\delta'$ да отива в новото състояние $q_e$.
  Ето и цялата дефиниция на новата функция на преходите $\delta'$:
  \begin{itemize}
  \item 
    \marginpar{$q_e$ - error състояние}
    $\delta'(q_e,a) = q_e$, за всяко $a\in\Sigma$;
  \item
    \marginpar{$\A'$ симулира $\A$}
    За всяко $q\in Q$, $a\in\Sigma$, ако $\delta(q,a) = p$, то
    $\delta'(q,a) = p$;
  \item
    За всяко $q\in Q$, $a\in\Sigma$, ако $\delta(q,a)$ не е дефинирано, то
    $\delta'(q,a) = q_e$.
  \end{itemize}
  \marginpar{\writedown Довършете доказателството!}
  Сега лесно може да се докаже, че $\L(\A) = \L(\A')$.
\end{proof}

\begin{prop}
  \label{pr:automata-union}
  Класът на автоматните езици е затворен относно операцията {\bf обединение}.
  Това означава, че ако $L_1$ и $L_2$ са два произволни автоматни езика над азбуката $\Sigma$, то $L_1\cup L_2$
  също е автоматен език.
\end{prop}
\begin{proof}
  \marginpar{Защо изискваме $\A_1$ и $\A_2$ да са тотални?}
  Нека $L_1 = \L(\A_1)$ и $L_2 = \L(\A_2)$,
  където $\A_1 = \FAn{1}$ и $\A_2 = \FAn{2}$ са {\bf тотални}.
  Определяме автомата $\A = \FA$, който разпознава $L_1\cup L_2$ по следния начин:
  \begin{itemize}
  \item
    $Q = Q_1\times Q_2$;
  \item
    \marginpar{Едновременно симулираме изчисление и по двата автомата}
    Определяме за всяко $\pair{r_1,r_2} \in Q$ и всяко $a \in \Sigma$,
    \[\delta(\pair{r_1,r_2},a) = \pair{\delta_1(r_1,a),\delta_2(r_2,a)};\]
  \item
    $s = \pair{s_1,s_2}$;
  \item
    $F = \{\pair{r_1,r_2}\mid r_1\in F_1\vee r_2 \in F_2\} = (F_1\times Q_2)\cup (Q_1\times F_2)$.
  \end{itemize}
  \marginpar{По-нататък ще дадем друга конструкция за обединение, която ще бъде по-ефективна.}
  \marginpar{\writedown Проверете, че $\L(\A) = \L(\A_1)\cup \L(\A_2)$}
\end{proof}

\begin{cor}
  Класът на автоматните езици е затворен относно операцията {\bf сечение}.
  Това означава, че ако $L_1$ и $L_2$ са два произволни автоматни езика над азбуката $\Sigma$, то $L_1\cap L_2$
  също е автоматен език.
\end{cor}
\begin{proof}
  \marginpar{\ding{45} Докажете, че така построения автомат $\A$ разпознава $L_1\cap L_2$!}
  Използвайте конструкцията на автомата $\A = \FA$ от \Prop{automata-union},
  с единствената разлика, че тук избираме финалните състояния да бъдат елементите на множеството
  \[F = \{\pair{q_1,q_2} \mid q_1 \in F_1\ \&\ q_2 \in F_2\} = F_1\times F_2.\]
\end{proof}

\begin{prop}
  Нека $L$ е автоматен език.
  Тогава $\Sigma^\star\setminus L$ също е автоматен език.
\end{prop}
\begin{proof}
  \marginpar{Защо искаме $\A$ да бъде тотален ?}
  Нека $L = L(\A)$, където $\A = \FA$ е {\bf тотален}.
  Да вземем автомата $\A' = \pair{Q,\Sigma,s,\delta,Q\setminus F}$,
  т.е. $\A'$ е същият като $\A$, с единствената разлика, че финалните състояния на $\A'$
  са тези състояния, които {\bf не} са финални в $\A$.
  \marginpar{\writedown Проверете, че $\Sigma^\star\setminus L = \L(\A')$}
\end{proof}

\begin{problem}
  За всеки от следните езици $L$, постройте автомат $\A$, който разпознава езика $L$.
  \begin{enumerate}[a)]
  \item 
    $L = \{a^nb\mid n \geq 0\}$;
  \item
    $L = \{\varepsilon, a,b\}$;
  \item
    $L = \emptyset$;
  \item
    $L = \{a,b\}^\star\setminus\{\varepsilon\}$;
  \item
    $L = \{a^nb^m\mid n,m \geq 0\}$;
  \item
    $L = \{a^nb^m\mid n,m \geq 1\}$;
  \item
    $L = \{a,b\}^\star \setminus \{a\}$;
  \item
    $L = \{w \in \{a,b\}^\star \mid \mbox{съдържа поне две }a\}$;
  \item
    $L = \{w \in \{a,b\}^\star \mid \mbox{съдържа поне две }a\mbox{ и поне едно }b\}$;
  \item
    $L = \{w \in \{a,b\}^\star \mid \mbox{на всяка нечетна позиция на }w\mbox{ е буквата }a\}$;
  \item
    $L = \{w \in \{a,b\}^\star \mid w\mbox{ съдържа четен брой }a\mbox{ и най-много едно }b\}$;
  \item
    $L = \{w \in \{a,b\}^\star\mid \abs{w} \leq 3\}$;
  \item
    $L = \{w \in \{a,b\}^\star \mid w \mbox{ не започва с }ab\}$;
  \item
    $L = \{w \in \{a,b\}^\star \mid w \mbox{ завършва с }ab\}$;
  \item
    $L = \{w \in \{a,b\}^\star \mid w \mbox{ съдържа }bab\}$;
  \item
    $L = \{w \in \{a,b\}^\star \mid w \mbox{ не съдържа }bab\}$;
  \item
    \marginpar{(решена е по-долу)}
    $L = \{w \in \{a,b\}^\star \mid w \mbox{ няма две последователни }a\}$;
  \item
    $L = \{w \in \{a,b\}^\star \mid w\mbox{ започва и завършва с буквата } a\}$;
  \item
    $L = \{w \in \{a,b\}^\star \mid w\mbox{ започва и завършва с една и съща буква}\}$;
  \item
    $L = \{w \in \{a,b\}^\star \mid \abs{w} \equiv 0\ (\bmod\ 2)\ \&\ w \mbox{ съдържа точно едно }a\}$;
  \item
    $L = \{w \in \{a,b\}^\star \mid \mbox{ всяко }a\mbox{ в }w\mbox{ се следва от поне едно }b\}$;
  \item
    $L = \{w \in \{a,b\}^\star \mid \abs{w} \equiv 0 \bmod 3\}$;
  \item
    \marginpar{$N_a(w)$ - броят на срещанията на буквата $a$ в думата $w$}
    $L = \{w \in \{a,b\}^\star \mid N_a(w) \equiv 1 \bmod 3\}$;
  \item
    $L = \{\omega \in \{a,b\}^\star \mid N_a(\omega) \equiv 0 \bmod 3\ \&\ N_b(\omega) \equiv 1 \bmod 2\}$;
  \item
    $L = \{\omega \in \{a,b\}^\star \mid N_a(\omega) \equiv 0 \bmod 2\ \vee\ \omega\mbox{ съдържа точно две }b\}$;
  \item
    $L = \{\omega \in \{a,b\}^\star \mid \omega \text{ съдържа равен брой срещания на }ab\text{ и на }ba\}$.
  \item
    $L = \{\omega_1 \sharp \omega_2 \sharp \omega_3 \mid \forall i \in [1,3](\omega_i \in \{a,b\}^\star\ \&\ |\omega_i| \geq i+1)\}$;
  \end{enumerate}
\end{problem}
  
\marginpar{\ding{45} За всички тези автомати, дефинирайте функциите на преходите им!}
  \begin{figure}[H]
    \begin{subfigure}[b]{0.5\textwidth}
      \begin{tikzpicture}[->,>=stealth,thick,node distance=45pt]
        \tikzstyle{every state}=[circle,minimum size=20pt,auto]
        
        \node[initial below, state]   (0) {$s$};
        \node[state]            (1) [right of=0]{$q_1$};
        \node[state]            (2) [right of=1]{$q_2$};
        \node[state,accepting]  (3) [right of=2]{$q_3$};
        
        \path 
        (0) edge [loop above]   node [above] {$a$}    (0)
        (0) edge [bend left=15] node [above] {$b$}    (1)
        (1) edge [loop above]   node [above] {$b$}    (1)
        (1) edge [bend left=15] node [above] {$a$}    (2)
        (2) edge [bend left=30] node [below] {$a$}    (0)
        (2) edge [bend left=15] node [above] {$b$}    (3)
        (3) edge [loop above]   node [above] {$a,b$}  (3);
      \end{tikzpicture}
      \caption{$\{\omega \in \{a,b\}^\star \mid \omega\mbox{ съдържа }bab\}$}
    \end{subfigure}
    \qquad
    \begin{subfigure}[b]{0.4\textwidth}
      \begin{tikzpicture}[->,>=stealth,thick,node distance=45pt]
        \tikzstyle{every state}=[circle,minimum size=20pt,auto]
        
        \node[initial below, state]   (0) {$s$};
        \node[state]            (1) [right of=0]{$q_1$};
        \node[state,accepting]  (2) [right of=1]{$q_2$};
        
        \path 
        (0) edge [loop above]   node [above] {$b$}    (0)
        (0) edge [bend left=15] node [above] {$a$}    (1)
        (1) edge [loop above]   node [above] {$b$}    (1)
        (1) edge [bend left=15] node [above] {$a$}    (2)
        (2) edge [loop above]   node [above] {$a,b$}  (2);
      \end{tikzpicture}
      \caption{$\{\omega \in \{a,b\}^\star \mid N_a(\omega) \geq 2\}$}
    \end{subfigure}
    \begin{subfigure}[b]{0.5\textwidth}
      \begin{tikzpicture}[->,>=stealth,thick,node distance=45pt]
        \tikzstyle{every state}=[circle,minimum size=20pt,auto]
        
        \node[initial, accepting, state] (0) {$s$};
        \node[state]                     (1) [right of=0]{$q_1$};
        
        \path 
        (0) edge [loop above]   node [above] {$b$}   (0)
        (0) edge [bend left=15] node [above] {$a$}   (1)
        (1) edge [bend left=15] node [below] {$b$}   (0);
      \end{tikzpicture}
      \caption{$\{\omega \in \{a,b\}^\star \mid $ всяко $a$ в $\omega$ се следва от поне едно $b\}$ }
      \end{subfigure}
      \qquad
      \qquad
      \begin{subfigure}[b]{0.5\textwidth}
      \begin{tikzpicture}[->,>=stealth,thick,node distance=45pt]
        \tikzstyle{every state}=[circle,minimum size=20pt,auto]
        
        \node[initial, state, accepting]   (0) {$s$};
        \node[state]                       (1) [right of=0]{$q_1$};
        \node[state]                       (2) [right of=1]{$q_2$};
        
        \path 
        (0) edge [loop above]   node   [above] {$b$}    (0)
        (0) edge [bend left=15] node   [above] {$a$}    (1)
        (1) edge [loop above]   node   [above] {$b$}    (1)
        (1) edge [bend left=15] node   [above] {$a$}    (2)
        (2) edge [loop above]   node   [above] {$b$}    (2)
        (2) edge [bend left=30] node   [below] {$a$}    (0);
      \end{tikzpicture}
      \caption{$\{\omega \in \{a,b\}^\star \mid N_a(\omega) \equiv 0\ (\bmod\ 3)\}$}
    \end{subfigure}
  \end{figure}    

В повечето от горните задачи е лесно да се съобрази, че построения автомат разпознава желания език.
При по-сложни задачи обаче, ще се наложи да дадем доказателство, като обикновено се прилага 
{\em метода на математическата индукция} върху дължината на думите.
Ще разгледаме няколко такива примера.

\begin{problem}
  Докажете, че езикът $L$ е автоматен, където
  \[L = \{\alpha \in \{a,b\}^\star\ \mid\ \alpha\mbox{ не съдържа две поредни срещания на }a\}.\]
\end{problem}
\begin{proof}
  Да разгледаме $\A = \FA$ с функция на преходите
  \begin{figure}[H]
    \begin{center}
      \begin{tikzpicture}[->,>=stealth,thick,node distance=45pt]
        \tikzstyle{every state}=[circle,minimum size=20pt,auto]
        
        \node[initial, accepting, state] (0) {$s$};
        \node[accepting, state]   (1) [right of=0]{$q_1$};
        \node[state]   (2) [right of=1]{$q_2$};
        
        \path 
        (0) edge [loop above]   node [above] {$b$}   (0)
        (0) edge [bend left=15] node [above] {$a$}   (1)
        (1) edge [bend left=15] node [below] {$b$}   (0)
        (1) edge [bend left=15] node [above] {$a$}   (2)
        (2) edge [loop above]   node [above] {$a,b$} (2);
      \end{tikzpicture}
    \end{center}
 \end{figure}

 Ще докажем, че $L = \L(\A)$.
 Първо ще се концентрираме върху доказателството на $\L(\A) \subseteq L$.
 \marginpar{Озн. $\abs{\alpha}$ - дължината на думата $\alpha$}
 Ще докажем с индукция по дължината на думата $\alpha$, че:
 \begin{enumerate}[(1)]
 \item 
   ако $\delta^\star(s,\alpha) = s$, то
   $\alpha$ не съдържа две поредни срещания на $a$
   и ако $\abs{\alpha} > 0$, то $\alpha$ завършва на $b$;
 \item
   ако $\delta^\star(s,\alpha) = q_1$, то
   $\alpha$ не съдържа две поредни срещания на $a$
   и завършва на $a$.
 \end{enumerate}

 За $\abs{\alpha} = 0$, то твърденията (1) и (2) са ясни (Защо?).
 Да приемем, че твърденията $(1)$ и $(2)$ са верни за произволни думи $\alpha$ с дължина $n$.
 Нека $\abs{\alpha} = n+1$, т.е. $\alpha = \beta x$, където $\abs{\beta} = n$ и $x \in \Sigma$.
 Ще докажем (1) и (2) за $\alpha$.
 \begin{itemize}[-]
 \item 
   % \marginpar{Тук използваме \Prop{delta-star}}
   Нека $\delta^\star(s,\beta x) = s = \delta(\delta^\star(s,\beta),x)$.
   Според дефиницията на функцията $\delta$, $x = b$ и $\delta^\star(s,\beta) \in \{s,q_1\}$.
   Тогава по {\bf И.П.} за (1) и (2), $\beta$ не съдържа две поредни срещания на $a$.
   Тогава е очевидно, че $\beta x$ също не съдържа две поредни срещания на $a$.
 \item
   Нека $\delta^\star(s,\beta x) = q_1 = \delta(\delta^\star(s,\beta),x)$.
   Според дефиницията на $\delta$, $x = a$ и $\delta^\star(s,\beta) = s$.
   Тогава по {\bf И.П.} за (1), $\beta$ не съдържа две поредни срещания на $a$
   и ако $\abs{\beta} > 0$, то завършва на $b$.
   Тогава е очевидно, че $\beta x$ също не съдържа две поредни срещания на $a$.
 \end{itemize}
 
 Така доказахме с индукция по дължината на думата, че за всяка дума $\alpha$
 са  изпълнени твърденията $(1)$ и $(2)$. По дефиниция, ако $\alpha \in \L(\A)$,
 то $\delta^\star(s,\alpha) \in \{s,q_1\}$ и от $(1)$ и $(2)$ следва, че и в двата случа
 $\alpha$ не съдържа две поредни срещания на буквата $a$, т.е. $\alpha \in L$.
 С други думи, доказахме, че 
 \[\L(\A) \subseteq L.\]

 Сега ще докажем другата посока, т.е. $L \subseteq \L(\A)$.
 Това означава да докажем, че
 \[(\forall \alpha \in \Sigma^\star)[\alpha \in L\ \Rightarrow\ \delta^\star(s,\alpha) \in F],\]
 \marginpar{Да напомним, че $p \Rightarrow q \equiv \neg q \Rightarrow \neg  p$}
 което е еквивалентно на
 \begin{equation}
   \label{eq:case2}
   (\forall \alpha \in \Sigma^\star)[\delta^\star(s,\alpha) \not\in F \ \Rightarrow\ \alpha\not\in L].
 \end{equation}
 Това е лесно да се съобрази.
 Щом $\delta^\star(s,\alpha) \not\in F$, то 
 $\delta^\star(s,\alpha) = q_2$ и думата $\alpha$ може да се представи по следния начин:
 \[\alpha = \beta a \gamma\ \&\ \delta^\star(s,\beta) = q_1.\]
 
 Използвайки свойство (2) от по-горе, понеже $\delta^\star(s,\beta) = q_1$, то
 $\beta$ не съдържа две поредни срещания на $a$, но завършва на $a$.
 Сега е очевидно, че $\beta a$ съдържа две поредни срещания на $a$ и 
 щом $\beta a$ е префикс на $\alpha$, то думата $\alpha \not\in L$.
 С това доказахме Свойство \ref{eq:case2}, а следователно и посоката $L\subseteq \L(\A)$.
\end{proof}

\begin{framed}
  За една дума $\alpha \in \{0,1\}^\star$, 
  нека с $\alpha_{(2)}$ да означим числото в десетична бройна система, което се представя в двоична бройна система като $\alpha$.
  Например, $1101_{(2)} = 1 \cdot 2^3+1\cdot 2^2+0\cdot 2^1+1\cdot 2^0 = 13$.
  Тогава имаме следните свойства:
  \begin{itemize}
  \item
    $\varepsilon_{(2)} = 0$,
  \item
    $(\alpha0)_{(2)} = 2\cdot(\alpha)_{(2)}$,
  \item
    $(\alpha1)_{(2)} = 2\cdot(\alpha)_{(2)} + 1$.
  \end{itemize}
\end{framed}
\marginpar{Да отбележим, че за всяко число $n$ има безкрайно много думи $\alpha$, за които $\alpha_{(2)} = n$. Например, $10_{(2)} = 010_{(2)} = 0010_{(2)} = \cdots$}

\begin{problem}
  Докажете, че $L = \{\omega \in \{0,1\}^\star \mid \omega_{(2)} \equiv 2\ (\bmod\ 3)\}$ е автоматен.
\end{problem}
\begin{proof}
  Нашият автомат ще има три състояния $\{q_0,q_1,q_2\}$, като началното състояние ще бъде $q_0$.
  Целта ни е да дефинираме така автомата, че да имаме следното свойство:
  \begin{equation}
    (\forall\alpha\in\Sigma^\star)(\forall i < 3)[\alpha_{(2)} \equiv i\ (\bmod\ 3)\ \Leftrightarrow\ \delta^\star(q_0,\alpha) = q_i],
  \end{equation}
  т.е. всяко състояние отговаря на определен остатък при деление на три.
  Понеже искаме нашия автомат да разпознава тези думи $\alpha$,
  за които $\alpha_{(2)} \equiv 2\mod 3$, финалното състояние ще бъде $q_2$.
  Дефинираме функцията $\delta$ следвайки следните свойства:
  \begin{itemize}
  \item
    \marginpar{$\delta(q_0,0) = q_0$}
    $\alpha_{(2)} \equiv 0 \bmod 3\ \Rightarrow\ (\alpha0)_{(2)} \equiv 0 \bmod 3$;
  \item 
    \marginpar{$\delta(q_0,1) = q_1$}
    $\alpha_{(2)} \equiv 0 \bmod 3\ \Rightarrow\ (\alpha1)_{(2)} \equiv 1 \bmod 3$;
  \item
    \marginpar{$\delta(q_1,0) = q_2$}
    $\alpha_{(2)} \equiv 1 \bmod 3\ \Rightarrow\ (\alpha0)_{(2)} \equiv 2 \bmod 3$;
  \item 
    \marginpar{$\delta(q_1,1) = q_0$}
    $\alpha_{(2)} \equiv 1 \bmod 3\ \Rightarrow\ (\alpha1)_{(2)} \equiv 0 \bmod 3$;
  \item
    \marginpar{$\delta(q_2,0) = q_1$}
    $\alpha_{(2)} \equiv 2 \bmod 3\ \Rightarrow\ (\alpha0)_{(2)} \equiv 1 \bmod 3$;
  \item 
    \marginpar{$\delta(q_2,1) = q_2$}
    $\alpha_{(2)} \equiv 2 \bmod 3\ \Rightarrow\ (\alpha1)_{(2)} \equiv 2 \bmod 3$.
  \end{itemize}
  Ето и картинка на автомата $\A$:
  \begin{figure}[H]
    % \begin{subfigure}[b]{0.3\textwidth}% [$L_1 = L(M_1)$]{
    \begin{center}
      \begin{tikzpicture}[->,>=stealth,thick,node distance=45pt]
        \tikzstyle{every state}=[circle,minimum size=15pt,auto]
        
        \node[initial,state]      (0) {$q_0$};
        \node[state]              (1) [right of=0]{$q_1$};
        \node[accepting, state]   (2) [right of=1]{$q_2$};
        
        \path 
        (0) edge  [loop above]    node [above]  {$0$} (0)
        (0) edge  [bend left=15]  node [above]  {$1$} (1)
        (2) edge  [bend left=15] node [below]  {$0$} (1)
        (1) edge  [bend left=15]  node [below]  {$1$} (0)
        (1) edge  [bend left=15] node [above]  {$0$} (2)
        (2) edge  [loop above]    node [above]  {$1$} (2);
      \end{tikzpicture}
      \end{center}
      \caption{$\L(\A) \stackrel{?}{=} \{\omega\in\{0,1\}^\star \mid \alpha_{(2)} \equiv 2\ (\bmod\ 3)\}$}
 %   \end{subfigure}
 \end{figure}
 \noindent Да разгледаме твърденията:
 \begin{enumerate}[(1)]
  \item 
    $\delta^\star(q_0,\alpha) = q_0\ \Rightarrow\ \alpha_{(2)} \equiv 0 \mod 3$;
  \item 
    $\delta^\star(q_0,\alpha) = q_1\ \Rightarrow\ \alpha_{(2)} \equiv 1 \mod 3$;
  \item 
    $\delta^\star(q_0,\alpha) = q_2\ \Rightarrow\ \alpha_{(2)} \equiv 2 \mod 3$.
  \end{enumerate}
  Ще докажем (1), (2) и (3) {\em едновременно} с индукция по дължината на думата $\alpha$.
  За $\abs{\alpha} = 0$, всички условия са изпълнени. (Защо?)
  Да приемем, че (1), (2) и (3) са изпълнени за думи с дължина $n$.
  Нека $\abs{\alpha} = n+1$, т.е. $\alpha = \beta x$, $\abs{\beta} = n$.
  За да приложим индукционното предположение, ще използваме следното свойство:
  \[\delta^\star(q_0,\beta x) = \delta(\delta^\star(q_0,\beta),x).\]
  % Ние знаем, че то е изпълнено от \Prop{delta-star}.
  
  Ще докажем подробно само (3) понеже другите твърдения се доказват по сходен начин.
  \marginpar{Обърнете внимание, че в доказателството на (3) използваме И.П. не само за (3), но и за (2)}
  Нека $\delta^\star(q_0,\beta x) = q_2$. 
  Имаме два случая:
  \begin{itemize}
  \item 
    $x = 0$. 
    Тогава, по дефиницията на $\delta$, 
    $\delta(q_1,0) = q_2$ и следователно, $\delta^\star(q_0,\beta) = q_1$.
    По {\bf И.П.} за (2) с $\beta$,
    \[\delta^\star(q_0,\beta) = q_1\ \Rightarrow\ \beta_{(2)} \equiv 1 \bmod 3\]
    Тогава, $(\beta0)_{(2)} \equiv 2 \mod 3$. Така доказахме, че
    \[\delta^\star(q_0,\beta 0) = q_2\ \Rightarrow\ (\beta 0)_{(2)} \equiv 2 \bmod 3.\]
  \item
    $x = 1$.
    Тогава, по дефиницията на $\delta$, $\delta(q_2,1) = q_2$ и следователно,
    $\delta^\star(q_0,\beta) = q_2$.
    По {\bf И.П.} за (3) с $\beta$,
    \[\delta^\star(q_0,\beta) = q_2\ \Rightarrow\ \beta_{(2)} \equiv 2 \mod 3.\]
    Тогава, $(\beta1)_{(2)} \equiv 2 \mod 3$. Така доказахме, че
    \[\delta^\star(q_0,\beta 1) = q_2\ \Rightarrow\ (\beta 1)_{(2)} \equiv 2 \mod 3.\]
  \end{itemize}
  
  За да докажем (1), нека $\delta^\star(q_0,\beta x) = q_0$. 
  \begin{itemize}
  \item 
    $x = 0$. Разсъжденията са аналогични, като използваме {\bf И.П.} за (1).
  \item
    $x = 1$. Разсъжденията са аналогични, като използваме {\bf И.П.} за (2).
  \end{itemize}
  
  По същия начин доказваме и (2). Нека $\delta^\star(q_0,\beta x) = q_1$. 
  \begin{itemize}
  \item 
    При $x = 0$, използваме {\bf И.П.} за (3).
  \item
    При $x = 1$, използваме {\bf И.П.} за (1).
  \end{itemize}

  От (1), (2) и (3) следва директно, че $\L(\A) \subseteq L$.
  
  За другата посока, нека $\alpha \in L$, т.е. $(\alpha)_{(2)} \equiv 2 \bmod 3$.
  Ако допуснем, че $\alpha \not\in \L(\A)$, то това означава, че $\delta^\star(q_0,\alpha) \in \{q_0,q_1\}$.
  Но в тези случаи получаваме от твърдения (1) и (2), че $(\alpha)_{(2)} \equiv 0 \bmod 3$ или $(\alpha)_{(2)} \equiv 1 \bmod 3$.
  Това е противоречие с избора на $\alpha \in L$. Следователно, ако $\alpha \in L$, то $\delta(q_0,\alpha) = q_2$.
  Така доказахме и посоката $L \subseteq \L(\A)$.
\end{proof}

\section{Регулярни езици}

\begin{dfn}
  \index{език!регулярен}
  \index{регулярен израз}
  \marginpar{Това е друг пример за индуктивна (рекурсивна) дефиниция.}
  Нека е дадена азбука $\Sigma$. Дефинираме множеството от
  {\em регулярни езици} над азбуката $\Sigma$ и едновременно с това 
  множеството от {\em регулярни изрази}, които разпознават тези езици.
  \begin{enumerate}[1)]
  \item
   за всеки символ $a \in \Sigma$, $\{a\}$ е регулярен език,
   който се разпознава от регулярния израз $a$;
  \item
    $\{\varepsilon\}$ е регулярен език,
    който се разпознава от регулярния израз $\varepsilon$;
  \item
    $\emptyset$ е регулярен език,
    който се разпознава от регулярния израз $\emptyset$;
  \item
    \index{обединение}
    $L_1\cup L_2$, където $L_1$ и $L_2$ са регулярни езици,
    който се разпознава от регулярния израз $(r_1 + r_2)$,
    където $r_1$ и $r_2$ са регулярните изрази за $L_1$ и $L_2$.
    Записваме, че $\L(r_1) \cup \L(r_2) = \L(r_1+r_2)$.
  \item
    \index{конкатенация}
    \marginpar{Тази операция се наричка конкатенация. Обикновено изпускаме знака $\cdot$}
    $L_1\cdot L_2 = \{uw\mid u \in L_1\ \&\ w \in L_2\}$, където $L_1$ и $L_2$ са регулярни езици,
    който се разпознава от регулярния израз $(r_1\cdot r_2)$,
    където $r_1$ и $r_2$ са регулярните изрази за $L_1$ и $L_2$.
    Записваме, че $\L(r_1)\cdot\L(r_2) = \L(r_1 \cdot r_2)$.
  \item
    \marginpar{Звезда на Клини}
    \index{звезда на Клини}
    $L^\star = \{w_1w_2\cdots w_n\mid n \in \Nat\ \&\ w_i \in L\mbox{ за всяко } i \leq n\}$,  където $L$ е регулярен език,
    който се разпознава от регулярния израз $(r^\star)$,
    където $r$ е регулярния израз за $L$.
    Записваме, че  $\L(r)^\star = \L(r^\star)$.
    Можем да запишем, че $L^\star = \bigcup_n L^n$, където
    $L^0 = \{\varepsilon\}$ и $L^{n+1} = L^n\cdot L$.
  \end{enumerate}
\end{dfn}


\begin{example}
  Нека да разгледаме няколко примера какво точно представлява прилагането
  на операцията звезда на Клини върху един език.
  \begin{itemize}
  \item 
    Нека $L = \{0,11\}$. Тогава:
    \begin{itemize}
    \item 
      $L^0 = \{\varepsilon\}$, $L^1 = L$,
    \item
      $L^2 = L^1\cdot L^1 = \{00,011,110,1111\}$,
    \item
      $L^3 = L^1\cdot L^2 = \{000,0011,0110,01111,1100,11011,11110,111111\}$.
    \end{itemize}
  \item
    Нека $L = \emptyset$.
    Тогава:
    \begin{itemize}
    \item 
      $L^0 = \{\varepsilon\}$,
    \item
      $L^1 = \emptyset$,
    \item
      $L^2 = L^1 \cdot L^1 = \emptyset$.
    \end{itemize}    
    Получаваме, че $L^\star = \{\varepsilon\}$, т.е. {\em краен} език
  \item
    Нека $L = \{0^i\mid i \in \Nat\} = \{\varepsilon, 0, 00, 000, \dots\}$.
    Тогава лесно може да се види, че $L = L^\star$.
  \end{itemize}
\end{example}

\begin{problem}
  За произволни регулярни изрази $r$ и $s$, 
  проверете:
  \begin{enumerate}[a)]
  \item 
    $r+s = s + r$;
  \item
    $(\varepsilon + r)^\star = r^\star$;
  \item
    $\emptyset^\star = \varepsilon$;
  \item
    $(r^\star s^\star) = (r+s)^\star$;
  \item
    $(r^\star)^\star = r^\star$;
  \item
    $(rs + r)^\star r = r(sr+r)^\star$;
  \item
    $s(rs+s)^\star r = rr^\star s(rr^\star s)^\star$;
  \item
    $(r+s)^\star = r^\star + s^\star$;
  \item
    $\emptyset^\star = \varepsilon^\star$;
  \end{enumerate}
\end{problem}

\begin{framed}
\begin{thm}[Клини]
  \index{Клини}
  Всеки автоматен език се описва с регулярен израз.
\end{thm}
\end{framed}
\begin{proof}
  \marginpar{стр. 79 от \cite{papadimitriou}, стр. 33 от \cite{hopcroft1}}
  Нека  $L = \L(\A)$, за някой краен детерминиран автомат $\A$.
  Да фиксираме едно изброяване на състоянията $Q = \{q_1,\dots,q_n\}$,
  като началното състояние е $q_1$.
  Ще означаваме с $L(i,j,k)$ множеството от тези думи, които
  могат да се разпознаят от автомата по път, който започва от $q_i$,
  завършва в $q_j$, и междинните състояния имат индекси $\leq k$.
  Например, за думата $\alpha = a_1a_2\cdots a_n$ имаме, че $\alpha \in L(i,j,k)$
  точно тогава, когато съществуват състояния $q_{l_1},\dots,q_{l_{n-1}}$, като $l_1,\dots,l_{n-1} \leq k$ и
  \[q_i\stackrel{a_1}{\rightarrow} q_{l_1} \stackrel{a_2}{\rightarrow} q_{l_2} \stackrel{a_3}{\rightarrow} \dots \stackrel{a_{n-1}}{\rightarrow} q_{l_{n-1}}\stackrel{a_n}{\rightarrow} q_j.\]
  Тогава за $n = \abs{Q}$, 
  \[L(i,j,n) = \{\alpha\in\Sigma^\star\mid \delta^\star(q_i,\alpha) = q_j\}.\]
  Така получаваме, че 
  \[\L(\A) = \bigcup\{L(1,j,n)\mid q_j \in F\} = \bigcup_{q_j\in F}L(1,j,n).\]
  Ще докажем с {\bf индукция по $k$}, че за всяко $i,j,k$, множествата от думи $L(i,j,k)$
  се описват с регулярен израз $r^k_{i,j}$
  \begin{enumerate}[a)]
  \item
    Нека $k = 0$. Ще докажем, че за всяко $i,j$, $L(i,j,0)$ се описва с регулярен израз.
    Имаме да разгледаме два случая.
    
    Ако $i = j$, то 
    \[L(i, j, 0) = \{\varepsilon\}\cup\{a\in\Sigma \mid \delta(q_i,a) = q_j\}.\]
    Ако $i \neq j$, то
    \[L(i, j, 0) = \{a\in\Sigma \mid \delta(q_i, a) = q_j\}.\]
  \item
    Да предположим, че $k > 0$ и за всяко $i$, $j$, можем да намерим регулярните изрази,
    съответстващи на $L(i,j,k-1)$. Тогава
    \[L(i,j,k) = L(i,j,k-1)\ \cup\ L(i,k,k-1)\cdot (L(k,k,k-1)^\star) \cdot L(k,j,k-1).\]
    Тогава по {\bf И.П.} следва, че $L(i,j,k)$ може да се опише с регулярен израз, който е
    \[r^{k-1}_{i,j} + r^{k-1}_{i,k}\cdot (r^{k-1}_{k,k})^\star\cdot r^{k-1}_{k,j}.\]
  \end{enumerate}
  Заключаваме, че за всяко $i,j,k$, $L(i,j,k)$ може да се опише с регулярен израз $r^{k}_{i,j}$.
  Тогава, ако $F = \{q_{i_1},\dots,q_{i_k}\}$, то $\L(\A)$ се описва с регулярния израз
  \[r^n_{1,i_1} + r^n_{1,i_2} + \dots + r^n_{1,i_k}.\]
\end{proof}

\begin{example}
  \label{fig:a1}
  Да разгледаме следния автомат:
  
  \begin{figure}[H]
    \begin{center}
      \begin{tikzpicture}[->,>=stealth,thick,node distance=45pt]
        \tikzstyle{every state}=[circle,minimum size=15pt,auto]
        
        \node[initial,state]      (1) {$q_1$};
        \node[accepting, state]   (2) [right of=1]{$q_2$};
        
        \path 
        (1) edge [loop above]  node [above] {$1$} (1)
        (1) edge  node [above] {$0$} (2)
        (2) edge [loop above] node [above] {$0,1$} (2);
      \end{tikzpicture}
      \end{center}
 \end{figure}

 За да намерим регулярния език за автомата от Пример \ref{fig:a1}, 
 трябва да намерим $r^2_{1,2}$, защото началното състояние е $q_1$, финалното е $q_2$ и 
 броят на състоянията в автомата е $2$.
 \begin{align*}
   r^0_{1,1} =\ & \varepsilon + 1,\\
   r^0_{1,2} =\ & 0,\\
   r^0_{2,1} =\ & \emptyset,\\
   r^0_{2,2} =\ & \varepsilon +  0 + 1,\\
    r^1_{1,2} =\ & r^0_{1,2} + r^0_{1,1}\cdot(r^0_{1,1})^\star \cdot r^0_{1,2} = 0 + (\varepsilon + 1)(\varepsilon + 1)^\star0 = 1^\star0,\\
    r^1_{2,2} =\ & r^0_{2,2} + r^0_{2,1} \cdot (r^0_{1,1})^\star\cdot r^0_{1,2} = \varepsilon + 0 + 1 + \emptyset(\varepsilon + 1)^\star0 = \varepsilon + 0 + 1\\
    r^2_{1,2} =\ & r^{1}_{1,2} + r^{1}_{1,2}(r^1_{2,2})^\star r^1_{2,2} \\
    =\ & 1^\star0 + 1^\star0 (\varepsilon + 0 + 1)^\star (\varepsilon + 0 + 1) = 1^\star 0 (0 + 1)^\star.
  \end{align*}
Ясно е, че $L_1$ се описва с регулярния израз $r^2_{1,2} = 1^\star 0 (0 + 1)^\star$.
\end{example}

Следващата ни цел е да видим, че имаме и обратната посока на горната лема.
Ще докажем, че всеки регулярен език е автоматен. За тази цел първо ще 
въведем едно обобщение на понятието краен детерминиран автомат.

\section{Недетерминирани крайни автомати}
\index{автомат!недетерминиран}
\begin{dfn}
  \marginpar{Въведени от Рабин и Скот \cite{rabin-scott}}
  \marginpar{За яснота, често ще означаваме с $\N$ недетерминирани автомати, а с $\A$ детерминирани автомати}
  Недетерминиран краен автомат представлява
  \[\N = \NFA,\]
  \begin{itemize}
  \item
    $Q$ е крайно множество от състояния;
  \item
    $\Sigma$ е крайна азбука;
  \item
    $\Delta: Q\times\Sigma \to \Ps(Q)$ е функцията на преходите.
    \marginpar{Да напомним, че $\Ps(Q) = \{R\mid R\subseteq Q\}$, $\abs{\Ps(Q)} = 2^{\abs{Q}}$}
    \marginpar{Сипсър \cite{sipser1} позволява $\varepsilon$-преходи}
    Обърнете внимание, че тя е тотална.
  \item
    $s \in Q$ е началното състояние;
  \item
    $F\subseteq Q$ е множеството от финални състояния.
  \end{itemize}
\end{dfn}

\begin{thm}
  За всеки НKА $\N$ съществува еквивалентен на него КДА $\D$, т.е. $\L(\N) = \L(\D)$.
\end{thm}
\begin{proof}
  Нека $\N = \NFA$. Ще построим КДА $\D = (Q',\Sigma,\delta,s',F')$.
  Конструкцията е следната:
  \marginpar{Да отбележим, че детерминираният автомат $\D$ има не повече от $2^{\abs{Q}}$ на брой състояния}
  \begin{itemize}
  \item
    $Q' = \Ps(Q)$;
  \item
    $\delta(R,a) = \{q\in Q\mid (\exists r\in R)[q\in\Delta(r,a)]\} = \bigcup_{r\in R}\Delta(r,a)$;
  \item
    $s' = \{s\}$;
  \item
    $F' = \{R \subseteq Q \mid R\cap F \neq \emptyset\}$.
  \end{itemize}
\end{proof}

% \begin{problem}
%   За дума $\alpha = a_1a_2\cdots a_n$, дефинираме $\alpha^R = a_na_{n-1}\cdots a_1$.
%   \marginpar{Индукция по $\abs{\beta}$.}
%   Докажете, че
%   \[(\forall \alpha,\beta\in\Sigma^\star)[(\alpha\beta)^R = \beta^R\alpha^R].\]
% \end{problem}

\begin{problem}
  За всеки КНА $\N$ съществува КНА $\N'$ с едно финално състояние, 
  за който $\L(\N) = \L(\N')$.
\end{problem}
\begin{hint}
  Вместо формална конструкция, да разгледаме един пример, който илюстрира идеята.
  \begin{figure}[H]
    \begin{subfigure}[b]{0.3\textwidth}
      \begin{tikzpicture}[framed,->,>=stealth,thick,node distance=45pt]
        \tikzstyle{every state}=[circle,minimum size=20pt,auto]
        \node[initial,state]      (1) {$s$};
        \node[state,accepting]     [above right of=1] (2) {$q_1$};
        \node[state,accepting]     [below right of=1] (3) {$q_2$};
        \path
        (1) edge [bend left=15] node  [above] {$a$} (2)
        (2) edge [bend left=15] node  [right] {$b$} (1)
        % (2) edge [loop above] node  [above] {$a$} (2)
        (2) edge [bend left=15] node  [right] {$a$} (3)
        (3) edge [bend left=15] node  [below] {$a$} (1)
        (3) edge [loop below] node  [right] {$b$} (3);
        % (1) edge [bend right=15] node [below] {$b$} (3);
      \end{tikzpicture}
      \caption{автомат $\N$}
    \end{subfigure}
    \qquad
    \qquad
    \begin{subfigure}[b]{0.4\textwidth}
      \begin{tikzpicture}[framed,->,>=stealth,thick,node distance=45pt]
        \tikzstyle{every state}=[circle,minimum size=20pt,auto]
        \node[initial,state]      (1) {$s$};
        \node[state]     [above right of=1] (2) {$q_1$};
        \node[state]     [below right of=1] (3) {$q_2$};
        \node[state,accepting]     [right=3cm of 1] (4) {$f$};
        \path
        (1) edge [bend left=15] node  [above] {$a$} (2)
        % (2) edge [loop above] node  [above] {$a$} (2)
        (2) edge [bend left=15] node  [right] {$b$} (1)
        (2) edge [bend left=15] node  [right] {$a$} (3)
        (3) edge [loop below] node  [right] {$b$} (3)
        (3) edge [bend left=15] node  [below] {$a$} (1)
        (1) edge [dashed,bend left=15] node  [above] {$a$} (4)
        (2) edge [dashed,bend left=15] node  [above] {$a$} (4)
        (3) edge [dashed,bend right=15] node  [below] {$b$} (4);
        % (1) edge [bend right=15] node [below] {$b$} (3);
      \end{tikzpicture}
    \caption{автомат $\N'$, $\L(\N') = \L(\N)$}
  \end{subfigure}
\end{figure}  
За произволен автомат $\N$, формулирайте точно конструкцията на $\N'$ с едно финално състояние и докажете, че наистина $\L(\N) = \L(\N')$.
Обърнете внимание, че примера показва, че е възможно $\N$ да е детерминиран автомат, но полученият $\N'$ да бъде недетерминиран.
\end{hint}

\begin{problem}
  \marginpar{Нека $\A$, $L = \L(\A)$, е само с едно финално състояние. }
  Докажете, че ако $L$ е автоматен език, то $L^R = \{\omega^R \mid \omega \in L\}$
  също е автоматен.
\end{problem}

\begin{lemma}
  Съществува КНА $\N = \NFA$, който разпознава езика $L(r)$, 
  където $r = \emptyset$, $r = \varepsilon$ или $r = a$, за $a\in \Sigma$.
\end{lemma}
\begin{proof}
  \begin{figure}[H]
    \begin{subfigure}[b]{0.2\textwidth}
      \label{subf:a1}
      \begin{tikzpicture}[->,>=stealth,thick,node distance=35pt]
        \tikzstyle{every state}=[circle,minimum size=15pt,auto]
        \node[initial,state]      (1) {$s$};
      \end{tikzpicture}
      \caption{$L(\emptyset)$}
    \end{subfigure}
    \qquad
    \begin{subfigure}[b]{0.2\textwidth}
      \begin{tikzpicture}[->,>=stealth,thick,node distance=35pt]
        \tikzstyle{every state}=[circle,minimum size=15pt,auto]
        \node[initial,state,accepting]      (1) {$s$};
      \end{tikzpicture}
      \caption{$L(\varepsilon)$}
    \end{subfigure}
    \qquad
    \begin{subfigure}[b]{0.3\textwidth}
      \begin{tikzpicture}[->,>=stealth,thick,node distance=35pt]
        \tikzstyle{every state}=[circle,minimum size=15pt,auto]
        \node[initial,state]      (1)              {$s$};
        \node[accepting,state]    (2) [right of=1] {$q$};
        \path 
        (1) edge  node [above] {$a$} (2);
      \end{tikzpicture}
      \caption{$L(a)$}
    \end{subfigure}
  \end{figure}
\end{proof}

\begin{lemma}
  Класът на автоматните езици е затворен относно операцията {\bf конкатенация}.
  Това означава, че ако $L_1$ и $L_2$ са два произволни автоматни езика, то $L_1\cdot L_2$
  също е автоматен език.
\end{lemma}
\begin{proof}
  Нека са дадени автоматите:
  \begin{itemize}
  \item
    $\N_1 = \NFAn{1}$, като $\L(\N_1) = L_1$;
  \item
    $\N_2 = \NFAn{2}$, като $\L(\N_2) = L_2$.
  \end{itemize}
  Ще дефинираме автомата $\N = \NFA$ като
  \[\L(\N) = L_1\cdot L_2 = \L(\N_1)\cdot\L(\N_2).\]
  \begin{itemize}
  \item
    $Q = Q_1 \cup Q_2$;
  \item
    $s = s_1$;
  \item
    $F = 
    \begin{cases}
      F_1 \cup F_2, & \text{ ако } s_2 \in F_2\\
      F_2,          & \text{ иначе}.
    \end{cases}$
  \item 
    $\Delta(q,a) = 
    \begin{cases}
      \Delta_1(q,a),                      & \text{ ако }q\in Q_1\setminus F_1\ \&\ a\in\Sigma\\
      \Delta_2(q,a),                      & \text{ ако }q\in Q_2\ \&\ a\in\Sigma\\
      \Delta_1(q,a) \cup \Delta_2(s_2,a), & \text{ ако }q \in F_1\ \&\ a\in\Sigma.
    \end{cases}$
  \end{itemize}
\end{proof}

\begin{figure}[H]
  \center
  \begin{subfigure}[b]{0.3\textwidth}
    \label{subf:a1}
    \begin{tikzpicture}[framed,->,>=stealth,thick,node distance=45pt]
      \tikzstyle{every state}=[circle,minimum size=15pt,auto]
      \node[initial,state,accepting]      (1) {$s_1$};
      \node[state]                        (2) [right of=1] {$q_1$};
      \node[state]                        (3) [above right of=2] {$q_2$};
      \node[state,accepting]              (4) [below right of=2] {$q_3$};
      \path
      (1) edge node [above] {$a$} (2)
      (2) edge node [above] {$a$} (3)
      (2) edge node [below] {$b$} (4)
      (3) edge [bend right=30] node [above] {$a$} (1)
      (4) edge [bend left=30] node [below] {$b$} (1);
    \end{tikzpicture}
    \caption{автомат $\N_1$}
  \end{subfigure}
  \qquad
  \qquad
  \qquad
  \begin{subfigure}[b]{0.3\textwidth}
    \begin{tikzpicture}[framed,->,>=stealth,thick,node distance=45pt]
      \tikzstyle{every state}=[circle,minimum size=15pt,auto]
      \node[initial,state]      (1) {$s_2$};
      \node[state]     [above right of=1] (2) {$q_4$};
      \node[state,accepting]     [below right of=1] (3) {$q_5$};
      \path
      (1) edge [bend left=15] node  [above] {$a$} (2)
      (2) edge [bend left=15] node  [right] {$a$} (3)
      (1) edge [bend right=15] node [below] {$b$} (3);
    \end{tikzpicture}
    \caption{автомат $\N_2$}
  \end{subfigure}
\end{figure}

\begin{example}
    За да построим автомат, който разпознава конкатенацията на $\L(\N_1)$ и $\L(\N_2)$,
    трябва да свържем финалните състояния на $\N_1$ с изходящите от $s_2$ състояния на $\N_2$.
    
    \begin{figure}[H]
      \center
      % \begin{subfigure}[b]{0.3\textwidth}
      \begin{tikzpicture}[framed,->,>=stealth,thick,node distance=2cm]
        \tikzstyle{every state}=[circle,minimum size=15pt,auto]
        \node[initial,state]                      (1) {$s_1$};
        \node[state] [right of=1]                 (2) {$q_1$};
        \node[state] [above right of=2]           (3) {$q_2$};
        \node[state] [below right of=2]           (4) {$q_3$};
        \node[state] [right=4cm of 1]             (5) {$s_2$};
        \node[state] [above right of=5]           (6) {$q_4$};
        \node[state,accepting] [below right of=5] (7) {$q_5$};
        \path
        (1) edge node [above]                         {$a$} (2)
        (2) edge node [above]                         {$a$} (3)
        (2) edge node [below]                         {$b$} (4)
        (3) edge [bend right=15] node [above]         {$a$} (1)
        (4) edge [bend left=15] node [below]          {$b$} (1)
        (5) edge [bend left=15] node [below]          {$a$} (6)
        (6) edge [bend left=15] node [right]          {$a$} (7)
        (5) edge [bend right=15] node [above]         {$b$} (7)
        (1) edge [dashed, bend left=45] node [above]  {$a$} (6)
        (1) edge [dashed, bend right=45] node [below] {$b$} (7)
        (4) edge [dashed, bend left=45] node [above]  {$a$} (6)
        (4) edge [dashed, bend left=10] node [above]  {$b$} (7);
      \end{tikzpicture}
      \caption{$\L(\N) = \L(\N_1)\cdot\L(\N_2)$}
  \end{figure}  
  Обърнете внимание, че $\N_1$ и $\N_2$ са детерминирани автомати, но $\N$ е недетерминиран.
  Също така, в този пример се оказва, че вече $s_2$ е недостижимо състояние, но в общия случай не можем да 
  го премахнем, защото може да има преходи влизащи в $s_2$.
\end{example}


\begin{lemma}
  Класът от автоматните езици е затворен относно операцията {\bf обединение}.
\end{lemma}
\begin{proof}
  Нека са дадени автоматите:
  \begin{itemize}
  \item 
    $\N_1 = \NFAn{1}$, като $L(\N_1) = L_1$;
  \item
    $\N_2=\NFAn{2}$, като $L(\N_2) = L_2$.
  \end{itemize}
  Ще дефинираме автомата $\N=\NFA$, така че
  \[L(\N) = L(\N_1) \cup L(\N_2).\]
  \begin{itemize}
  \item 
    $Q = Q_1 \cup Q_2 \cup \{s\}$;
  \item
    $F = 
    \begin{cases}
      F_1 \cup F_2 \cup \{s\}, & \text{ ако } s_1 \in F_1 \vee s_2 \in F_2\\
      F_1 \cup F_2,            & \text{ иначе } 
    \end{cases}$
  \item
    $
    \Delta(q,a) = 
    \begin{cases}
      \Delta_1(q,a),                       & \text{ ако } q\in Q_1\ \&\ a\in\Sigma\\
      \Delta_2(q,a),                       & \text{ ако } q\in Q_2\ \&\  a\in\Sigma\\
      \Delta_1(s_1,a) \cup \Delta_2(s_2,a), & \text{ ако } q = s\ \&\  a \in\Sigma.
    \end{cases}
    $
    % $\Delta(q,a) = \Delta_1(q,a)$, за всяко $q\in Q_1$, $a\in\Sigma$;
  % \item
  %   $\Delta(q,a) = \Delta_2(q,a)$, за всяко $q\in Q_2$, $a\in\Sigma$;
  % \item
  %   $\Delta(s,a) = \Delta_1(s_1,a) \cup \Delta_2(s_2,a)$, за всяко $a\in\Sigma$;
  \end{itemize}
\end{proof}
\begin{remark}
  В началното състояние на новопостроения автомат $\N$ не влизат ребра.
\end{remark}


\begin{example}
    За да построим автомат, който разпознава обединението на $\L(\N_1)$ и $\L(\N_2)$,
    трябва да свържем финалните състояния на $\N_1$ с изходящите от $s_2$ състояния на $\N_2$.
    
    \begin{figure}[H]
      \center
      % \begin{subfigure}[b]{0.3\textwidth}
      \begin{tikzpicture}[framed,->,>=stealth,thick,node distance=2cm]
        \tikzstyle{every state}=[circle,minimum size=20pt,auto]
        \node[initial,state,accepting]      (0) {$s$};
        \node[state,accepting]    [above right of=0]        (1) {$s_1$};
        \node[state]    [right of=1]        (2) {$q_1$};
        \node[state]                        (3) [above right of=2] {$q_2$};
        \node[state,accepting]                        (4) [below right of=2] {$q_3$};
        \node[state]    [below right=2cm of 0] (5) {$s_2$};
        \node[state]     [above right of=5] (6) {$q_4$};
        \node[state,accepting]     [below right of=5] (7) {$q_5$};
        \path
        (1) edge node [above]                  {$a$} (2)
        (2) edge node [above]                  {$a$} (3)
        (2) edge node [below]                  {$b$} (4)
        (3) edge [bend right=15] node [above]  {$a$} (1)
        (4) edge [bend left=15]  node [below]  {$b$} (1)
        (5) edge [bend left=15] node [below]   {$a$} (6)
        (6) edge [bend left=15] node  [right] {$a$} (7)
        (5) edge [bend right=15] node [above]  {$b$} (7)
        (0) edge [dashed, bend right=15] node [below]  {$a$} (2)
        (0) edge [dashed, bend right=15] node [below]  {$a$} (6)
        (0) edge [dashed, bend right=45] node [below]  {$b$} (7);
      \end{tikzpicture}
      \caption{$\L(\N) = \L(\N_1)\cup\L(\N_2)$}
  \end{figure}  
  Обърнете внимание, че $\N_1$ и $\N_2$ са детерминирани автомати, но $\N$ е недетерминиран.
  Освен това, новото състояние $s$ трябва да бъде маркирано като финално, защото $s_1$ е финално.
\end{example}

\begin{lemma}
  Класът от автоматните езици е затворен относно операцията {\bf звезда на Клини}.
\end{lemma}
\begin{proof}
  Нека е даден автомата $\N = \NFA$, за който е изпънено, че
  $L(\N) = L(r)$.
  Първата стъпка е да построим $\N_1 = \NFAn{1}$, такъв че 
  \[L(\N_1) = \bigcup_{n\geq 1} (L(\N))^n = \bigcup_{n\geq 1} (L(r))^n = L(r^+).\]
  \begin{itemize}
  \item
    $Q_1 = Q$;
  \item
    $s_1 = s$;
  \item
    $F_1 = F$;
  \item
    $
    \Delta_1(q,a) = 
    \begin{cases}
      \Delta(q,a), & \text{ ако } q\in Q\setminus F, a \in \Sigma\\
      \Delta(q,a) \cup \Delta(s,a), & \text{ ако } q\in F, a\in\Sigma.
    \end{cases}
    $
    % $\Delta_1(q,a) = \Delta(q,a)$, за всяко $q\in Q\setminus F$, $a\in\Sigma$;
  % \item
  %   $\Delta_1(q,a) = \Delta(q,a) \cup \Delta(s,a)$, за всяко $q\in F$, $a\in\Sigma$;
  \end{itemize}
  Накрая строим автомат $\N_2$, за който $L(\N_2) = \{\varepsilon\} \cup L(\N_1)$.
\end{proof}

\begin{figure}[H]
  %\begin{subfigure}[H]{0.3\textwidth}
  \label{subf:a1}
  \center  
  \begin{tikzpicture}[framed,->,>=stealth,thick,node distance=45pt]
    \tikzstyle{every state}=[circle,minimum size=15pt,auto]
    \node[initial,state]      (1) {$s_1$};
    \node[state]              (2) [right of=1] {$q_1$};
    \node[state,accepting]    (3) [right of=2] {$q_2$};
    \path
    (1) edge node [above] {$a$} (2)
    (2) edge node [above] {$b$} (3)
    (3) edge [bend left=45] node [below] {$b$} (1);
  \end{tikzpicture}
  \caption{автомат $\N_3$}
\end{figure}

\begin{example}
  Нека да приложим конструкцията за да намерим автомат разпознаващ $\L(\N_3)^\star$.

  \begin{figure}[H]
    \label{subf:a1}
    \center
    \begin{tikzpicture}[framed,->,>=stealth,thick,node distance=45pt]
      \tikzstyle{every state}=[circle,minimum size=20pt,auto]
      \node[initial,state,accepting]      (0) {$s$};
      % \node[state,accepting]      (4) [above right=0cm and 2cm of 0]{$s_2$};
      \node[state]      (1) [below right of=0] {$s_1$};
      \node[state]              (2) [right of=1] {$q_1$};
      \node[state,accepting]    (3) [right of=2] {$q_2$};
      \path
      (0) edge [dashed, bend left=15] node [above] {$a$} (2)
      (1) edge node [above] {$a$} (2)
      (2) edge node [below] {$b$} (3)
      (3) edge [bend left=45] node [below] {$b$} (1)
      (3) edge [dashed, bend right=45] node [above] {$a$} (2);        
    \end{tikzpicture}
    \caption{$\L(\N) = \L(\N_3)^\star = \L(\N_3)^+ \cup \{\varepsilon\}$}
  \end{figure}
    
  Лесно се вижда, че $\L(\N_1) = \{(abb)^nab\mid n\in\Nat\}$.
  Формално погледнато, след като построим автомат за езика $\L(\N_1)^+$, трябва да приложим
  конструкцията за обединение на автомата за езика $\L(\N_1)^+$ с автомата за езика $\{\varepsilon\}$.
  Защо трябва да добавим ново начално състояние $s$?
  Да допуснем, че вместо това сме направили $s_1$ финално.
  Тогава има опасност да разпознаем повече думи. Например, думата $abb$ би се разпознала от този автомат,
  но $abb \not\in\L(\N_1)^\star$.
  
\end{example}
% \begin{remark}
%   Запазваме свойството, че в началното състояние не влизат ребра.
% \end{remark}

\begin{problem}
  Да фиксираме една дума $\alpha$ над дадена азбука $\Sigma$.
  \marginpar{(текстовият файл $\beta \in \Sigma^\star$)}
  Опишете алгоритъм, който за вход произволен текстов файл $\beta$,
  отговаря дали думата $\alpha$ се среща в $\beta$.
  Каква е сложността на този алгоритъм относно дължините на $\alpha$ и $\beta$ ?
\end{problem}


\section{Езици, които не са регулярни}
\begin{lemma}[за покачването (регулярни езици)]
  \index{лема за покачването!регулярни езици}
  \label{lem:pumping-reg}
  \marginpar{На англ. се нарича \\ Pumping Lemma}
  \marginpar{Има подобна лема и за безконтекстни езици}
  \marginpar{Обърнете внимание, че $0 \in \Nat$ и $xy^0z =  xz$}
  Нека $L$ да бъде регулярен език.
  Съществува число $p\geq 1$, зависещо само от $L$, 
  за което за всяка дума $\alpha\in L, \abs{\alpha}\geq p$ може да 
  бъде записана във вида $\alpha = xyz$ и 
  \begin{enumerate}[1)]
  \item
    $|y|\geq 1$;
  \item
    $|xy|\leq p$;
  \item
    $(\forall i\in\Nat)[xy^iz \in L]$.
  \end{enumerate}
\end{lemma}
\begin{proof}
  \marginpar{стр. 88 от \cite{papadimitriou}, стр. 78 от \cite{sipser1}}
  Понеже $L$ е регулярен, той се разпознава от $\A = \FA$.
  Да положим $p = \abs{Q}$ и нека $\alpha = a_1a_2\cdots a_k$ е дума, за която $k \geq p$.
  Да разгледаме първите $p$ стъпки от изпълнението на $\alpha$ върху $\A$:
  \[q_0\stackrel{a_1}{\rightarrow} q_1 \stackrel{a_2}{\rightarrow} \dots \stackrel{a_p}{\rightarrow} q_p.\]
  Тъй като $\abs{Q} = p$, а по този път участват $p+1$ състояния $q_0,q_1,\dots,q_p$,
  то съществуват числа $i, j$, за които $0\leq i < j\leq p$ и $q_i = q_j$.
  Нека разделим думата $\alpha$ на три части по следния начин:
  \[x = a_1\cdots a_i,\quad y = a_{i+1}\cdots a_j,\quad z = a_{j+1}\cdots a_k.\]
  Ясно е, че $\abs{y} \geq 1$ и $\abs{xy} = j \leq p$.
  \marginpar{\ding{45} Докажете!}
  Освен това, лесно се съобразява, че за всяко $i \in\Nat$,
  $xy^iz \in L$. Да разгледаме само случая за $i = 0$.
  Думата $xy^0z = xz \in L$, защото имаме следното изчисление:
  \[q_0\stackrel{a_1}{\rightarrow} \cdots \stackrel{a_i}{\rightarrow} q_i\stackrel{a_{j+1}}{\rightarrow}q_{j+1}\cdots\stackrel{a_{p}}{\rightarrow}q_p\in F,\]
  защото $q_i = q_j$.
\end{proof}


% При фиксиран език $L$, условието на \Lem{pumping-reg} може формално да се запише така:
% {\scriptsize
% \[(\exists p \geq 1)(\forall \alpha \in L)[\abs{\alpha} \geq p \Rightarrow (\exists x,y,z\in\Sigma^\star)[\alpha = xyz\ \wedge\ \abs{y} \geq 1\ \wedge\ \abs{xy} \leq p\ \wedge\ (\forall i\in\Nat)[xy^iz \in L]]].\]}
% Отрицанието на горната формула може да се запише по следния начин:
% {\scriptsize  \[(\forall p \geq 1)(\exists \alpha \in L)[\abs{\alpha} \geq p\ \wedge (\forall x,y,z\in\Sigma^\star)[\alpha \neq xyz\ \vee\ \abs{y} \not\geq 1\ \vee\ \abs{xy} \not\leq p\ \vee\ (\exists i\in\Nat)[xy^iz \not\in L]]],\]}
% което е еквивалентно на:
% {\scriptsize
%   % \begin{equation}
%   %   \label{pump-neg}
%   \[(\forall p \geq 1)(\exists \alpha \in L)[\abs{\alpha} \geq p\ \wedge\ (\forall x,y,z\in\Sigma^\star)[(\alpha = xyz \wedge \abs{y} \geq 1\wedge \abs{xy} \leq p) \Rightarrow (\exists i\in\Nat)[xy^iz \not\in L]]].\]}
% % \end{equation}

% Това означава, че условието на \Lem{pumping-reg} може да се запише така:\\
% Ако условието \ref{pump-neg} е изпълнено, то $L$ не е регулярен.

% \begin{framed}
%   \Lem{pumping-reg} е полезна, когато искаме да докажем, че даден език $L$ {\bf не} е регулярен.
%   За да постигнем това, ние доказваме {\bf отрицанието} на условията от \Lem{pumping-reg} за $L$, т.е.
%   за всяка константа $p \geq 1$, намираме дума $\alpha \in L$, $\abs{\alpha}\geq p$, такава че за всяко разбиване на думата на три части, $\alpha = xyz$,
%   със свойствата $\abs{y} \geq 1$ и $\abs{xy} \leq p$, е изпълнено, че $(\exists i)[xy^iz \not\in L]$.
% \end{framed}

Практически е по-полезно да разглеждаме следната еквивалентна формулировка на лемата за покачването.
\marginpar{Контрапозиция на твърдението $p \to q$ е твърдението $\neg q \to \neg p$}
\begin{cor}[Контрапозиция на лемата за покачването]
  \label{cor:pumping-reg}
  \marginpar{Ясно е, че всеки краен език е регулярен. Нали?}
  Нека $L$ е произволен {\bf безкраен} език. Нека също така е изпълнено, че за всяко естествено число $p \geq 1$ можем да намерим дума $\alpha \in L$, $\abs{\alpha}\geq p$, такава че за всяко разбиване на думата на три части, $\alpha = xyz$,
  със свойствата $\abs{y} \geq 1$ и $\abs{xy} \leq p$, е изпълнено, че $(\exists i)[xy^iz \not\in L]$.
  Тогава $L$ {\bf не} е регулярен език.
\end{cor}
\begin{proof}
  \Lem{pumping-reg} гласи, че ако $L$ е регулярен език, то
  {\scriptsize
    \[(\exists p \geq 1)(\forall \alpha \in L)[\abs{\alpha} \geq p \Rightarrow (\exists x,y,z\in\Sigma^\star)[\alpha = xyz\ \wedge\ \abs{y} \geq 1\ \wedge\ \abs{xy} \leq p\ \wedge\ (\forall i\in\Nat)[xy^iz \in L]]].\]}
  Отрицанието на горното твърдение гласи, че ако 
  {\scriptsize  \[(\forall p \geq 1)(\exists \alpha \in L)[\abs{\alpha} \geq p\ \wedge (\forall x,y,z\in\Sigma^\star)[\alpha \neq xyz\ \vee\ \abs{y} \not\geq 1\ \vee\ \abs{xy} \not\leq p\ \vee\ (\exists i\in\Nat)[xy^iz \not\in L]]],\]}
  то $L$ {\bf не} е регулярен език.
  Горната формула е еквивалентна на:
  {\scriptsize
    % \begin{equation}
    %   \label{pump-neg}
    \[(\forall p \geq 1)(\exists \alpha \in L)[\abs{\alpha} \geq p\ \wedge\ (\forall x,y,z\in\Sigma^\star)[(\alpha = xyz \wedge \abs{y} \geq 1\wedge \abs{xy} \leq p) \Rightarrow (\exists i\in\Nat)[xy^iz \not\in L]]].\]}
\end{proof}


\begin{example}
  Езикът $L = \{a^nb^n \mid n\in \Nat\}$ {\bf не} е регулярен.
\end{example}
\begin{proof}
  \marginpar{Това е важен пример. По-късно ще видим, че този език е безконтекстен}
  % Да допуснем, че $L$ е регулярен.
  % Ще достигнем до противоречие като докажем отрицанието на условието на \Lem{pumping-reg},
  Ще докажем, че
  {\scriptsize
    \[(\forall p \geq 1)(\exists \alpha \in L)[\abs{\alpha} \geq p\ \wedge\ (\forall x,y,z\in\Sigma^\star)[(\alpha = xyz \wedge \abs{y} \geq 1\wedge \abs{xy} \leq p) \Rightarrow (\exists i\in\Nat)[xy^iz \not\in L]].\]}
  Доказателството следва стъпките:
  \begin{itemize}
  \item 
    Разглеждаме произволно число $p \geq 1$ (нямаме власт над избора на $p$).
  \item
    \marginpar{Няма общо правило, което да ни казва как избираме думата $\alpha$. Възможно е да пробаваме с няколко думи $\alpha$, докато намерим такава, която върши работа}
    Избираме дума $\alpha \in L$, за която $\abs{\alpha} \geq p$. Имаме свободата да изберем каквато дума $\alpha$
    си харесаме, стига тя да принадлежи на $L$ и да има дължина поне $p$.
    \marginpar{Обърнете внимание, че думата $\alpha$ зависи от константата $p$}
    Щом имаме тази свобода, нека да изберем думата $\alpha = a^pb^p \in L$.
    Очевидно е, че $\abs{\alpha} \geq p$.
  \item
    Разглеждаме произволно разбиване на $\alpha$ на три части, $\alpha = xyz$,
    за които изискваме свойствата $\abs{xy} \leq p$ и $\abs{y} \geq 1$ (не знаем нищо друго за $x$, $y$ и $z$ освен тези две свойства).
  \item
    Ще намерим $i\in\Nat$, за което $xy^iz \not\in L$.
    Понеже $\abs{xy} \leq p$, то $y = a^k$, за  $1\leq k \leq p$.
    Тогава, ако вземем $i = 0$, получаваме $xy^0z = a^{p-k}b^p$.
    Ясно е, че $xz \not\in L$, защото $p-k < p$.
  \end{itemize}  
  % Доказахме, че ако $L$ е регулярен език, то свойствата от \Lem{pumping-reg} не са изпълнени. Следователно, езикът $L$
  % не е регулярен.  
  Тогава от \Cor{pumping-reg} следва, че $L$ не е регулярен език.
\end{proof}

\begin{example}
  Езикът $L = \{a^mb^n \mid m,n\in \Nat\ \&\ m < n\}$ {\bf не} е регулярен.
\end{example}
\begin{proof}
  % Да допуснем, че $L$ е регулярен.
  % Следваме същата процедура както в предишния пример.
  Доказателството следва стъпките:
  \begin{itemize}
  \item 
    Разглеждаме произволно число $p \geq 1$.
  \item
    Избираме дума $\alpha \in L$, за която $\abs{\alpha} \geq p$. Имаме свободата да изберем каквато дума $\alpha$
    си харесаме, стига тя да принадлежи на $L$ и да има дължина поне $p$.
    Щом имаме тази свобода, нека да изберем думата $\alpha = a^{p}b^{p+1} \in L$. Очевидно е, че $\abs{\alpha} \geq p$.
  \item
    Разглеждаме произволно разбиване на $\alpha$ на три части, $\alpha = xyz$,
    за които изискваме свойствата $\abs{xy} \leq p$ и $\abs{y} \geq 1$ (не знаем нищо друго за $x$, $y$ и $z$ освен тези две свойства).
  \item
    Ще намерим $i\in\Nat$, за което $xy^iz \not\in L$.
    Понеже $\abs{xy} \leq p$, то $y = a^k$, за  $1\leq k \leq p$.
    Тогава ако вземем $i = 2$, получаваме 
    \[xy^2z = a^{p-k}a^{2k}b^{p+1} = a^{p+k}b^{p+1}.\]
    Ясно е, че $xy^2z \not\in L$, защото $p+k \geq p+1$.
  \end{itemize}
  Тогава от \Cor{pumping-reg} следва, че $L$ не е регулярен език.
\end{proof}

\begin{example}
  Езикът $L = \{a^n\ \mid\ n\mbox{ е просто число}\}$ не е регулярен.
\end{example}
\begin{proof}
  % Да допуснем, че $L$ е регулярен език. Ще достигнем до противоречие като докажем отрицанието на условието на \Lem{pumping-reg},
  % т.е. ще докажем, че
  % % {\scriptsize
  % \begin{align*}
  %   (\forall p \geq 1)(\exists \alpha \in L)[\abs{\alpha} \geq p\ \wedge\ (\forall x,y,z\in\Sigma^\star)[ & (\alpha = xyz \wedge \abs{y} \geq 1\wedge \abs{xy} \leq p) \Rightarrow\\
  %   & (\exists i\in\Nat)[xy^iz \not\in L]].
  % \end{align*}
  % }
  Доказателството следва стъпките:
  \begin{itemize}
  \item 
    Разглеждаме произволно число $p \geq 1$.
  \item
    Избираме дума $w \in L$, за която $\abs{w} \geq p$. Можем да изберем каквото $w$ 
    си харесаме, стига то да принадлежи на $L$ и да има дължина поне $p$.
    Нека да изберем думата $w \in L$, такава че $\abs{w} > p+1$.
    Знаем, че такава дума съществува, защото $L$ е безкраен език. По-долу ще видим защо този избор е важен за нашите разсъждения.
  \item
    Разглеждаме произволно разбиване на $w$ на три части, $w = xyz$,
    за които изискваме свойствата $\abs{xy} \leq p$ и $\abs{y} \geq 1$.
  \item
    Ще намерим $i$, за което $xy^iz \not\in L$,
    т.е. ще намерим $i$, за което 
    $\abs{xy^iz} = \abs{xz} + i\cdot\abs{y}$ е {\em съставно число}.
    Понеже $\abs{xy} \leq p$ и $\abs{xyz} > p+1$, то $\abs{z} > 1$.
    Да изберем $i = \abs{xz} > 1$. Тогава:
    \[\abs{xy^iz} = \abs{xz} + i.\abs{y} = \abs{xz} + \abs{xz}.\abs{y} = (1 + \abs{y})\abs{xz}\] е съставно число, следователно 
    $xy^iz \not\in L$.
  \end{itemize}
  Тогава от \Cor{pumping-reg} следва, че $L$ не е регулярен език.
\end{proof}

\begin{problem}
  Докажете, че езикът $L = \{a^{n^2}\ \mid\ n\in\Nat\}$ не е регулярен.  
\end{problem}
\begin{proof}
  % Да допуснем, че $L$ е регулярен. Отново ще докажем отрицанието на свойството за покачване от \Lem{pumping-reg}.
  Доказателството следва стъпките:
  \begin{itemize}
  \item 
    Разглеждаме произволно число $p \geq 1$.
  \item
    Избираме достатъчно дълга дума, която принадлежи на езика $L$.
    Например, нека $w = a^{p^2}$.
  \item
    Разглеждаме произволно разбиване на $w$ на три части, $w = xyz$, 
    като $\abs{xy} \leq p$ и $\abs{y} \geq 1$.
  \item
    Ще намерим $i$, за което $xy^iz \not\in L$.
    В нашия случай това означава, че $\abs{xz} + i\cdot\abs{y}$ не е точен квадрат.
    Тогава за $i = 2$,
    \[p^2 = \abs{xyz} < \abs{xy^2z} = \abs{xz} + 2\abs{y} \leq p^2 + 2p < (p+1)^2 .\]
    Получаваме, че $p^2 < \abs{xy^2z} < (p+1)^2$,
    откъдето следва, че $\abs{xy^2z}$ не е точен квадрат.
    Следователно, $xy^2z \not\in L$.
  \end{itemize}
  Тогава от \Cor{pumping-reg} следва, че $L$ не е регулярен език.  
\end{proof}

\begin{problem}
  Докажете, че езикът $L = \{a^{n!}\ \mid\ n\in\Nat\}$ не е регулярен.  
\end{problem}
\begin{proof}
  Доказателството следва стъпките:
  \begin{itemize}
  \item 
    Разглеждаме произволно число $p \geq 1$.
  \item
    Избираме достатъчно дълга дума, която принадлежи на езика $L$. Например, нека $\omega = a^{(p+2)!}$.
  \item
    Разглеждаме произволно разбиване на $\omega$ на три части, $\omega = xyz$, 
    като $\abs{xy} \leq p$ и $\abs{y} \geq 1$.
    Да обърнем внимание, че $1 \leq \abs{y} \leq p$
  \item
    Ще намерим $i$, за което $xy^iz \not\in L$.
    В нашия случай това означава, че $\abs{xz} + i\cdot\abs{y}$ не е от вида $n!$.
    Възможно ли е $xy^0z \in L$?
    Понеже $\abs{xyz} = (p+2)!$, това означава, че $\abs{xz} = k!$, за някое $k \leq p+1$.
    Тогава 
    \[\abs{y} = \abs{xyz} - \abs{xz} = (p+2)! - k! \geq (p+2)! - (p+1)! = (p+1).(p+1)! > p.\]
    Достигнахме до противоречие.
  \end{itemize}
  Тогава от \Cor{pumping-reg} следва, че $L$ не е регулярен език.  
\end{proof}

\subsection*{Следствия от лемата за покачването}

\begin{prop}
  \label{pr:automata-empty}
  Нека е даден автомата $\A = \FA$.
  Езикът $\L(\A)$ е {\bf непразен} точно тогава, когато съдържа дума $\alpha, \abs{\alpha} < \abs{Q}$.
\end{prop}
\begin{proof}
  Ще разгледаме двете посоки на твърдението.
  \begin{description}
  \item[$(\Rightarrow)$]
    Нека $L = \L(\A)$ е непразен език и нека $m = \min\{\abs{\alpha} \mid \alpha \in L\}$.
    Ще докажем, че $m < \abs{Q}$.    
    За целта, да допуснем, че $m \geq \abs{Q}$ и да изберем $\alpha \in L$, за която $\abs{\alpha} = m$.
    Според \Lem{pumping-reg}, съществува разбиване $xyz = \alpha$, 
    такова че $xz \in L$.
    При положение, че $\abs{y} \geq 1$, то $\abs{xz} < m$, което 
    е противоречие с минималността на $m$.
    Заключаваме, че нашето допускане е грешно. Тогава $m < \abs{Q}$, откъдето следва, че 
    съществува дума $\alpha \in L$ с $\abs{\alpha} < \abs{Q}$.
  \item[$(\Leftarrow)$]
    Тази посока е тривиална.
    Ако $L$ съдържа дума $\alpha$, за която $\abs{\alpha} < \abs{Q}$,
    то е очевидно, че $L$ е непразен език.
  \end{description}
\end{proof}

\begin{cor}
  Съществува алгоритъм, който проверява дали даден автомат разпознава непразен език.
\end{cor}

\begin{cor}
  Съществува алгоритъм, който определя дали два автомата $\A_1$ и $\A_2$ разпознават един и същ език.
\end{cor}
\begin{hint}
  Иползвайте  наблюдението, че
  \[\L(\A_1) = \L(\A_2) \iff (\L(\A_1)\setminus \L(\A_2)) \cup (\L(\A_2) \setminus \L(\A_1)) = \emptyset.\]
\end{hint}


\begin{prop}
  Регулярният език $L$, 
  разпознаван от КДА $\A$, е {\bf безкраен} точно тогава, когато съдържа дума $\alpha, \abs{Q} \leq \abs{\alpha} < 2\abs{Q}$.
\end{prop}
\begin{proof}
  Да разгледаме двете посоки на твърдението.
  \begin{description}
  \item[$(\Leftarrow)$]
    Нека $L$ е регулярен език, за който съществува дума $\alpha$, такава че $\abs{Q} \leq \abs{\alpha} < 2\abs{Q}$.
    Тогава от \Lem{pumping-reg} следва, че съществува разбиване $\alpha = xyz$ със свойството, че
    за всяко $i \in \Nat$, $xy^iz \in L$. Следователно, $L$ е безкраен, защото $\abs{y} \geq 1$.
  \item[$(\Rightarrow)$]
    Нека $L$ е безкраен език и % да приемем, че няма думи $\alpha$ със
    % свойството $\abs{Q} \leq \abs{\alpha} <  2\abs{Q}$.
    да вземем {\em най-късата} дума $\alpha \in L$, за която $\abs{\alpha} \geq 2\abs{Q}$.
    Понеже $L$ е безкраен, знаем, че такава дума съществува.
    Тогава отново по \Lem{pumping-reg}, имаме следното разбиване на $\alpha$:
    \[\alpha = xyz,\ \abs{xy} \leq \abs{Q},\ 1\leq \abs{y},\ xz \in L.\]
    Но понеже $\abs{xyz} \geq 2\abs{Q}$, а $1 \leq \abs{y} \leq \abs{Q}$, то $\abs{xyz} > \abs{xz} \geq \abs{Q}$ и понеже избрахме $\alpha = xyz$
    да бъде най-късата дума с дължина поне $2\abs{Q}$, заключаваме, че $\abs{Q} \leq \abs{xz} < 2\abs{Q}$ и $xz \in L$.
  \end{description}
\end{proof}

\begin{cor}
  Съществува алгоритъм, който проверява дали даден регулярен език е безкраен.
\end{cor}


\subsection*{Примери, за които лемата не е  приложима}

% \begin{problem}
%   \marginpar{Например, $c^+\{a^nb^n\mid n\in\Nat\}\cup (a\vert b)^\star$}
%   Да се даде пример за език $L$, който {\bf не} е регулярен, но удовлетворява
%   условието на \Lem{pumping-reg}.
% \end{problem}

\begin{example}
  Езикът $L = \{c^ka^nb^m\mid k,n,m \in \Nat\ \&\ k = 1\implies m = n\}$
  {\bf не} е регулярен, но условието за покачване от \Lem{pumping-reg} е изпълнено за него.
\end{example}
\begin{proof}
  Да допуснем, че $L$ е регулярен.
  Тогава ще следва, че 
  \[L_1 = L\cap ca^\star b^\star = \{ca^nb^n \mid n\in\Nat\}\]
  е регулярен,
  но с лемата за разрастването лесно се вижда, че $L_1$ не е.

  Сега да проверим, че условието за покачване от \Lem{pumping-reg} е изпълнено за $L$.
  Да изберем константа $p = 2$.
  Сега трябва да разгледаме всички думи $\alpha \in L$, $\abs{\alpha} \geq 2$
  и за всяка $\alpha$ да посочим разбиване $xyz = \alpha$, за което са изпълнени трите свойства от лемата.
  \marginpar{Условията за $x,y,z$ са:
    \begin{align*}
      & \abs{xy} \leq 2\\
      & \abs{y} \geq 1\\
      & (\forall i\in\Nat)(xy^iz \in L)
    \end{align*}}

  \begin{itemize}
  \item
    Ако $\alpha = a^n$ или $\alpha = b^n$, $n\geq 2$, то е  очевидно, че можем да
    намерим такова разбиване.
  \item
    $\alpha = a^nb^m$ и $n+m \geq 2$, $n \geq 1$.
    Избираме $x = \varepsilon$, $y = a$, $z = a^{n-1}b^m$.
  \item
    $\alpha = ca^nb^n$, $n\geq 1$.
    Избираме $x = \varepsilon$, $y = c$, $z = a^nb^n$.
  \item
    $\alpha = c^2a^nb^m$. 
    Избираме $x = \varepsilon$, $y = c^2$, $z = a^nb^m$.
  \item
    $\alpha = c^ka^nb^m$, $k \geq 3$.
    Избираме $x = \varepsilon$, $y = c$, $z = c^{k-1}a^nb^m$.
  \end{itemize}
\end{proof}

\section{Минимизация на КДА}

%\marginpar{\href{http://en.wikipedia.org/wiki/DFA_minimization}{Уикипедия}}

\begin{itemize}
\item
  \index{Майхил-Нероуд!релация}
  \marginpar{$\approx_L$ е известна като релация на Майхил-Нероуд}
  Нека $L \subseteq \Sigma^\star$ е език и нека $x,y \in \Sigma^\star$.
  Казваме, че $x$ и $y$ са {\bf еквивалентни относно} $L$, което записваме 
  като $x \approx_L y$, ако е изпълнено:
  \[(\forall z \in \Sigma^\star)[xz \in L \iff yz \in L].\]
  Това означава, че $x\approx_L y$, ако или и двете думи са в $L$ или и двете не са в $L$
  и освен това, като прибавим произволна дума на края на $x$ и $y$, новополучените
  думи са или и двете в $L$ или и двете не са в $L$.  
\item
  \marginpar{Трябва ли $\A$ да е тотален?}
  Нека $\A = \FA$ е КДА.
  Казваме, че две думи $\alpha,\beta \in \Sigma^\star$ са {\bf еквивалентни относно $\A$},
  което означаваме с $\alpha \sim_\A \beta$, ако 
  \[\delta^\star(s,\alpha) = \delta^\star(s,\beta).\]
\item
  Проверете, че $\approx_L$ и $\sim_\A$ са {\bf релации на еквивалентност}, т.е.
  те са рефлексивни, транзитивни и симетрични.
\item
  Класът на еквивалентност на думата $\alpha$ относно релацията $\approx_L$ означаваме като
  \[[\alpha]_L = \{\beta \in \Sigma^\star \mid \alpha \approx_L \beta\}.\]
  С $\abs{\approx_L}$ ще означаваме броя на класовете на еквивалентност на релацията $\approx_L$.
\item
  Класът на еквивалентност на думата $\alpha$ относно релацията $\sim_\A$ означаваме като
  \[[\alpha]_\A = \{\beta \in \Sigma^\star \mid \alpha \sim_\A \beta\}.\]
  \marginpar{$\abs{\sim_\A}$ също се нарича и {\bf индекс} на релацията $\sim_\A$}
  С $\abs{\sim_\A}$ ще означаваме броя на класовете на еквивалентност на релацията $\sim_\A$.
\item
  Казваме, че едно състояние $q$ е {\bf достижимо} в автомата $\A$,
  ако съществува дума $\alpha \in \Sigma^\star$, за която $\delta^\star_\A(s,\alpha) = q$.
  Премахването на недостижимите състояния от един автомат запазва разпознавания език.
\item
  Съобразете, че всяко състояние на $\A$, което е достижимо от началното състояние, определя клас на еквивалентност относно 
  релацията $\sim_\A$. Това означава, че ако за всяка дума означим  $q_\alpha = \delta^\star_\A(s,\alpha)$, то
  $\alpha \sim_\A \beta$ точно тогава, когато $q_\alpha = q_\beta$. Заключаваме, че броят на класовете на еквивалентност
  на $\sim_\A$ е равен на броя на достижимите от $s$ състояния.
\item
  Релациите $\approx_\L$ и $\sim_\A$ са {\em дясно-инвариантни}, т.е. за всеки две думи $\alpha$ и $\beta$
  е изпълнено:
  \marginpar{\ding{45} Проверете!}
  \begin{align*}
    \alpha \sim_\A \beta  &\implies (\forall \gamma\in\Sigma^\star)[\alpha\gamma \sim_\A \beta\gamma],\\
    \alpha \approx_\L \beta & \implies (\forall \gamma\in\Sigma^\star)[\alpha\gamma \approx_\L \beta\gamma].
  \end{align*}
\end{itemize}

\begin{thm}
  \label{th:rel-finer}
  За всеки КДА $\A = \FA$ е изпълнено:
  \[(\forall \alpha,\beta \in \Sigma^\star)[\alpha\sim_\A\beta \implies \alpha\approx_{\L(\A)}\beta].\]
  С други думи, 
  $[\alpha]_\A \subseteq [\alpha]_{\L(\A)}$, за всяка дума $\alpha \in \Sigma^\star$.
\end{thm}
\begin{proof}
%  \marginpar{стр. 95 от \cite{papadimitriou}}
  Да означим за всяка дума $\alpha$, $q_\alpha = \delta^\star_\A(s, \alpha)$.
  Лесно се съобразява, че за всеки две думи $\alpha$ и $\beta$ имаме 
  \begin{align*}
    \alpha \sim_\A \beta & \iff \delta^\star(s,\alpha) = \delta^\star(s,\beta) & (\text{по деф. на }\sim_\A)\\
    & \iff q_\alpha = q_\beta.
  \end{align*}
  Нека $\alpha \sim_\A \beta$. Ще проверим, че  $\alpha \approx_{\L(\A)} \beta$.
  За произволно $\gamma \in \Sigma^\star$ имаме:
  \begin{align*}
    \alpha\gamma \in \L(\A) & \iff \delta^\star(s,\alpha\gamma)\in F & (\text{по деф. на }\L(\A))\\
    & \iff \delta^\star(\delta^\star(s,\alpha),\gamma) \in F & (\text{по деф. на }\delta^\star)\\
    & \iff \delta^\star(q_\alpha, \gamma) \in F & (q_\alpha = \delta^\star(s,\alpha))\\
    & \iff \delta^\star(q_\beta, \gamma) \in F & (q_\alpha = q_\beta, \text{ защото }\alpha \sim_\A \beta)\\
    & \iff \delta^\star(\delta^\star(s,\beta),\gamma) \in F & (q_\beta = \delta^\star(s,\beta))\\
    & \iff \delta^\star(s,\beta\gamma) \in F & (\text{по деф. на }\delta^\star)\\
    & \iff \beta\gamma \in \L(\A) & (\text{по деф. на }\L(\A)).
  \end{align*}
  Заключаваме, че 
  \[(\forall \alpha,\beta \in \Sigma^\star)[\alpha\sim_\A\beta \implies \alpha\approx_{\L(\A)}\beta].\]
\end{proof}

\begin{cor}
  \label{cor:approx-less-sim}
  За всеки тотален КДА $\A$ е изпълнено, че
  \[\abs{\approx_{\L(\A)}} \leq \abs{\sim_\A}.\]
\end{cor}
\begin{proof}
  Нека $A = \{[\alpha]_{\L(\A)} \mid \alpha\in\Sigma^\star\}$ и $B = \{[\alpha]_\A \mid \alpha\in\Sigma^\star\}$.
  Да разгледаме изображението $f:B\to A$, определено като $f([\alpha]_\A) = [\alpha]_{\L(\A)}$.
  \begin{itemize}
  \item 
    Първо ще проверим, че $f$ е {\bf функция}, т.е. трябва да проверим, че 
    \[(\forall\alpha,\beta\in\Sigma^\star)[\alpha \sim_\A \beta \implies f([\alpha]_\A) = f([\beta]_\A)].\]
    
    Да допуснем, че съществуват думи $\alpha$ и $\beta$, такива че
    $[\alpha]_{\A} = [\beta]_{\A}$, но $f([\alpha]_{\A}) = [\alpha]_{\L(\A)} \neq [\beta]_{\L(\A)} = f([\beta]_{\A})$.
    Понеже $\sim_\A$ релация на еквивалентност, от $[\alpha]_{\L(\A)} \neq [\beta]_{\L(\A)}$
    следва, че $[\alpha]_{\L(\A)} \cap [\beta]_{\L(\A)} = \emptyset$.
    От \Th{rel-finer} следва веднага, че това е невъзможно, защото
    \[\emptyset \neq [\alpha]_\A = [\beta]_\A \subseteq [\alpha]_{\L(\A)} \cap [\beta]_{\L(\A)}.\]
  \item
    \marginpar{$(\forall a\in A)(\exists b\in B)(f(b) = a)$}
    Очевидно е, че $f$ е {\bf сюрекция}, защото на всеки клас $[\alpha]_{\L(\A)}$ съответства класа $[\alpha]_\A$.
  \item
    \marginpar{Защо?\\ \ding{45} Обяснете!}
    От това, че $f:B\to A$ е сюрективна функция следва, че $\abs{B} \leq \abs{A}$.
  \end{itemize}
\end{proof}

\begin{cor}
  \label{cor:upper-bound}
  Нека $L$ е произволен регулярен език $L$.  
  Всеки тотален КДА $\A$, който разпознава $L$ има свойството
  \[\abs{Q} \geq \abs{\approx_L}.\]
\end{cor}
\begin{proof}
  Да изберем $\A$, който разпознава $L$, бъде такъв, че да {\bf няма недостижими състояния}.
  Тъй като всяко достижимо състояние определя клас на еквивалентност относно $\sim_\A$,
  то получаваме, че $\abs{Q} = \abs{\sim_\A}$.
  Комбинирайки със \Cor{approx-less-sim},
  \[\abs{Q} = \abs{\sim_\A} \geq \abs{\approx_L}.\]
\end{proof}
Така получаваме {\em долна граница} за броя на състоянията в тотален автомат разпознаващ езика $L$.
Този брой е не по-малък от броя на класовете на еквивалентност на $\approx_L$.


\subsection*{Теорема за съществуване на МКДА}

\index{минимален автомат}
\begin{dfn}
  Нека $\A$ а тотален КДА, за който $L = \L(\A)$.
  Казваме, че $\A$ е {\bf минимален} за езика $L$, ако $\abs{Q_\A} = \abs{\approx_L}$.
\end{dfn}

% Да приемем, че сме фиксирали азбуката $\Sigma$.
\begin{thm}[Майхил-Нероуд]
  \label{th:myhill-nerode}
  \index{Майхил-Нероуд!теорема}
  \marginpar{Myhill-Nerode (1958)}
  \marginpar{Тази теорема не е доказана в \cite{sipser1}. Тук следваме \cite{papadimitriou}.}
  Нека $L\subseteq \Sigma^\star$ е регулярен език.
  Тогава съществува КДА $\A = \FA$, който разпознава $L$,
  с точно толкова състояния, колкото са класовете на еквивалентност на релацията $\approx_L$,
  т.е. $\abs{Q} = \abs{\approx_L}$.
\end{thm}
\begin{proof}
%  \marginpar{стр. 96 от \cite{papadimitriou}}
  Да фиксираме регулярния език $L$.
  Ще дефинираме тотален КДА $\A = \FA$, разпознаващ $L$, като:
  \begin{itemize}
  \item
    $Q = \{[\alpha]_L\mid \alpha\in \Sigma^\star\}$;
  \item
    $s = [\varepsilon]_L$;
  \item
    $F = \{[\alpha]_L\mid \alpha\in L\} = \{[\alpha]_L \mid [\alpha]_L \cap L \neq \emptyset\}$;
  \item
    Определяме изображението $\delta$ като 
    за всяка буква $x \in \Sigma$ и всяко състояние $[\alpha]_L\in Q$, 
    \[\delta([\alpha]_L,x) = [\alpha x]_L.\]
  \end{itemize}
  
  Първо, трябва да се уверим, че множеството от състояния $Q$ е крайно, т.е.
  релацията $\approx_\L$ има крайно много класове на еквивалентност.
  И така, тъй като $\L$ е регулярен език, то той се разпознава от някой тотален КДА $\A'$.
  От \Cor{upper-bound} имаме, че $\abs{Q^{\A'}} \geq \abs{\approx_L}$.
  Понеже $Q^{\A'}$ е крайно множество, то $\approx_L$ има крайно много класове и 
  следователно $Q$ също е крайно множество.

  Второ, трябва да се уверим, че изображението $\delta$ задава функция, т.е. 
  да проверим, че за всеки две думи $\alpha$, $\beta$ и всяка буква $x$,
  \[[\alpha]_L = [\beta]_L \implies \delta([\alpha]_L,x) = \delta([\beta]_L,x).\]
  Но това се вижда веднага, защото от определението на релацията $\approx_L$ следва, че
  ако $\alpha \approx_L \beta$, то за всяка буква $x$, $\alpha x \approx_L \beta x$,
  т.е. $[\alpha x]_L = [\beta x]_L$ и 
  \begin{align*}
    [\alpha]_L = [\beta]_L & \implies [\alpha x]_L = [\beta x]_L & (\text{свойство на }\approx_L)\\
    & \implies \delta([\alpha]_L,x) = [\alpha x]_L = [\beta x]_L = \delta([\beta]_L,x) & (\text{деф. на }\delta)
  \end{align*}
  
  Така вече сме показали, че $\A$ е коректно зададен тотален КДА.
  Остава да покажем, че $\A$ разпознава езика $L$, т.е. $\L(\A) = L$.
  За целта, първо ще докажем две помощни трърдения.

  \begin{prop}
    \label{pr:delta-myhill-nerode}
    За всеки две думи $\alpha$ и $\beta$,
    $\delta^\star([\alpha]_L,\beta) = [\alpha\beta]_L$.
  \end{prop}
  \begin{proof}
    Ще докажем това свойство с индукция по дължината на $\beta$.
    \begin{itemize}
    \item
      За $\beta = \varepsilon$ свойството следва директно от дефиницията на $\delta^\star$ като рефлексивно и транзитивно затваряне на $\delta$,
      защото $\delta^\star([\alpha]_L,\varepsilon) = [\alpha]_L$.
    \item
      Нека $\abs{\beta} = n+1$ и да приемем, че сме доказали твърдението за думи с дължина $n$.
      Тогава $\beta = \gamma a$, където $\abs{\gamma} = n$. Свойството следва от следните равенства:
      \begin{align*}
        \delta^\star([\alpha]_L, \gamma a) & = \delta(\delta^\star([\alpha]_L,\gamma),a) & (\text{ от деф. на }\delta^\star)\\
        & = \delta([\alpha\gamma]_L,a) & (\text{от {\bf И.П.} за }\gamma)\\
        & = [\alpha\gamma a]_L & (\text{от деф. на }\delta)\\
        & = [\alpha\beta]_L & (\beta = \gamma a).
      \end{align*}
    \end{itemize}
  \end{proof}
  
  \begin{prop}
    \label{pr:F-nerode-myhill}
    За всяка дума $\alpha$, $[\alpha]_L \cap L = \emptyset$ точно тогава, когато $\alpha \in L$.
  \end{prop}
  \begin{proof}
    За посоката $(\Rightarrow)$, нека $\beta \in [\alpha]_L \cap L$.
    Понеже $\beta \in [\alpha]_L$, имаме по дефиниция, че
    \[(\forall \gamma\in\Sigma^\star)[\alpha\gamma \in L \iff \beta\gamma\in L].\]
    В частност при $\gamma = \varepsilon$, получаваме $\alpha \in L \iff \beta \in L$.
    Тогава щом $\beta \in L$, то $\alpha \in L$.
    Посоката $(\Leftarrow)$ е очевидна, защото $\alpha \in [\alpha]_L$.
  \end{proof}
  \noindent 
  За да се убедим, че $L = \L(\A)$ е достатъчно да проследим еквивалентностите:
  \begin{align*}
    \alpha\in \L(\A) & \iff \delta^\star(s,\alpha) \in F & (\text{от деф. на }\L(\A))\\
    & \iff \delta^\star([\varepsilon]_L,\alpha) \in F & (\text{по деф. }s = [\varepsilon]_L)\\
    & \iff \delta^\star([\varepsilon]_L,\alpha) = [\alpha]_L \cap L \neq \emptyset & (\text{от деф. на }F)\\
    & \iff \delta^\star([\varepsilon]_L,\alpha) = [\alpha]_L\ \&\ \alpha \in L & (\text{от \Prop{F-nerode-myhill}})\\
    & \iff \alpha \in L & (\text{от \Prop{delta-myhill-nerode}}).
  \end{align*}
\end{proof}

\begin{dfn}
  \index{изоморфизъм}
  Нека $\A_1 = \FAn{1}$ и $\A_2 = \FAn{2}$.
  Казваме, че $\A_1$ и $\A_2$ са {\bf изоморфни}, което означаваме с $\A_1 \cong \A_2$, ако
  съществува биекция $f: Q_1\to Q_2$, за която:
  \begin{itemize}
  \item
    $f(s_1) = s_2$;
  \item
    $f[F_1] = \{f(q)\mid q\in F_1\} = F_2$;
  \item
    $(\forall a\in\Sigma)(\forall q\in Q_1)[f(\delta_1(q,a)) = \delta_2(f(q),a)]$.
  \end{itemize}
  Ще казваме, че $f$ задава изоморфизъм на $\A_1$ върху $\A_2$.
\end{dfn}

Това означава, че два автомата $\A_1$ и $\A_2$ са изоморфни, ако можем да получим $\A_2$
като преименуваме състоянията на $\A_1$.

\begin{cor}
  Нека е даден регулярният език $L$.
  Всички минимални автомати за $L$ са изоморфни на $\A_0$, автоматът построен в теоремата на Майхил-Нерод.
\end{cor}
\begin{proof}
  Нека $\A = \FA$ е произволен тотален автомат, за който $\L(\A) = L$ и $\abs{Q} = \abs{\approx_L}$.
  Съобразете, че $\A$ е {\em свързан}, т.е. всяко състояние на $\A$ е достижимо от началното.
  Искаме да докажем, че $\A \cong \A_0$.
  Понеже $\A$ е свързан, за всяко състояние $q$ можем да намерим дума $\omega_q$,
  за която $\delta^\star(s,\omega_q) = q$.
  Да дефинираме изображението $f:Q\to [\approx_L]$ като 
  \[f(q) = [\omega_q]_L.\]
  Ще докажем, че
  $f$ задава изоморфизъм на $\A$ върху $\A_0$. 
  \begin{itemize}
  \item
    Първо да съобразим, че ако $\delta^\star_\A(s,\alpha) = q$, то $[\omega_q]_L = [\alpha]_L$.
    Понеже $\delta^\star_\A(s,\alpha) = q = \delta^\star_\A(s,\omega_q)$, то $\omega_q \sim_\A \alpha$.
    От \Th{rel-finer} имаме, че
    \[\omega_q \sim_\A \alpha \implies \omega_q \approx_L \alpha.\]
    Това означава, $[\omega_q]_L = [\alpha]_L$ и следователно $f$ е определена коректно, т.е. $f$ е {\bf функция}.
  \item
    Ще проверим, че $f$ е {\bf инективна}, т.е.
    \[(\forall q_1,q_2 \in Q)[q_1\neq q_2 \implies f(q_1) \neq f(q_2)].\]
    Да допуснем, че има състояния $q_1 \neq q_2$, за които 
    \[f(q_1) = [\omega_{q_1}]_L = [\omega_{q_2}]_L = f(q_2).\]
    Тогава $\omega_{q_1} \not\sim_\A \omega_{q_2}$ и $\omega_{q_1} \approx_L \omega_{q_2}$.
    \marginpar{\writedown Обяснете!}
    Но тогава от \Cor{upper-bound} получаваме, че $\abs{\sim_\A} > \abs{\approx_L}$,
    което противоречи на минималността на $\A$.
  \item
    За да бъде $f$ {\bf сюрективна} трябва за всеки клас $[\beta]_L$ да съществува състояние $q$, за което $f(q) = [\beta]_L$.
    Понеже $\A$ е свързан, съществува състояние $q$, за което $\delta^\star_\A(s,\beta) = q$.
    Вече се убедихме, че в този случай $\beta \approx_L \omega_q$, защото $\beta \sim_\A \omega_q$.
    Тогава $f(q) = [\omega_q]_L = [\beta]_L$.
  \item
    За последно оставихме проверката, че $f$ наистина е {\bf изоморфизъм}:
    \begin{align*}
      f(\delta_\A(q,a)) & = f(\delta_\A(\delta^\star_\A(s,\omega_q),a)) & (\text{от избора на }\omega_q)\\
      & = f(\delta^\star_\A(s,\omega_qa)) & (\text{от деф. на }\delta^\star_\A)\\
      & = [\omega_qa]_L & (\text{от деф. на }f)\\
      & = \delta^\star_{\A_0}([\varepsilon]_L, \omega_qa) & (\text{от деф. на }\A_0)\\ 
      & = \delta_{\A_0}(\delta^\star_{\A_0}([\varepsilon]_L, \omega_q),a) & (\text{от деф. на }\delta^\star_{\A_0})\\
      & = \delta_{\A_0}([\omega_q]_L, a) & (\text{от \Prop{delta-myhill-nerode}})\\
      & = \delta_{\A_0}(f(q), a) & ( f(q) = [\omega_q]_L).
    \end{align*}
  \end{itemize}
\end{proof}

\subsection*{Проверка за регулярност на език}

\begin{framed}
  \begin{prop}
    Езикът $L$ е регулярен точно тогава, когато релацията $\approx_L$ има {\em крайно много} класове на еквивалентност.
  \end{prop}
\end{framed}
\begin{proof}
  Ако $L$ е регулярен, то той се разпознава от някой КДА $\A$, който има крайно много състояния 
  и следователно крайно много класове на еквивалентност относно $\sim_\A$.
  Релацията $\approx_L$ е по-груба от $\sim_\A$ и има по-малко класове на еквивалентност.
  Следователно, $\approx_L$ има крайно много класове на еквивалентност.
  
  За другата посока, ако $\approx_L$ има крайно много класове на еквивалентност, то можем да 
  построим КДА $\A$ както в доказателството на \Th{myhill-nerode}, който разпознава $L$.
\end{proof}

Това следствие ни дава още един начин за проверка дали даден език е регулярен.
За разлика от \Lem{pumping-reg}, сега имаме {\bf необходимо и достатъчно условие}.
При даден език $L$, ние разглеждаме неговата релация $\approx_L$.
Ако тя има крайно много класове, то езикът $L$ е регулярен.
В противен случай, езикът $L$ не е регулярен.

\begin{example}
  За езика $L = \{a^nb^n\mid n \in \Nat\}$ имаме, че $\abs{\approx_L} = \infty$,
  защото \[(\forall k,j\in\Nat)[k \neq j \implies [a^kb]_L \neq [a^jb]_L].\]
  Проверете, че $[a^kb]_L = \{a^kb,a^{k+1}b^{2},\dots,a^{k+l}b^{l+1},\dots\}$.
  Така получаваме, че релацията $\approx_L$ има безкрайно много класове на еквивалентност.
  Заключаваме, че този език {\bf не} е регулярен.
\end{example}

\begin{example}
  За езика $L = \{a^{n^2} \mid n \in \Nat\}$ имаме, че $\abs{\approx_L} = \infty$,
  защото \[(\forall m,n\in\Nat)[m \neq n \implies [a^{n^2}]_L \neq [a^{m^2}]_L].\]
  
  Без ограничение на общността, да разгледаме $n < m$ и думата $\gamma = a^{2n+1}$.
  Тогава $a^{n^2}\gamma = a^{(n+1)^2} \in L$, но 
  $m^2 < m^2 + 2n + 1 < (m+1)^2$ и следователно $a^{m^2}\gamma = a^{m^2+2n+1}\not\in L$.
\end{example}

\begin{example}
  За езика $L = \{a^{n!} \mid n \in \Nat\}$ имаме, че $\abs{\approx_L} = \infty$,
  защото \[(\forall m,n\in\Nat)[m \neq n \implies [a^{n!}]_L \neq [a^{m!}]_L].\]
  
  Без ограничение на общността, да разгледаме $n < m$ и думата $\gamma = a^{(n!)n}$.
  Тогава $a^{n!}\gamma = a^{(n+1)!} \in L$, но 
  $m! < m! + (n!)n < m! + (m!)m = (m+1)!$ и следователно $a^{m!}\gamma = a^{m!+(n!)n}\not\in L$.
\end{example}

\begin{problem}
  Да разгледаме редицата от числа $\{f_n\}$ зададена по следния начин:
  \begin{align*}
    & f_0 = f_1 = 1\\
    & f_{n+2} = f_{n} + f_{n+1}.
  \end{align*}
  Докажете, че $\abs{\approx_L} = \infty$, където
  \[L = \{a^{f_n} \mid n\in\Nat\}.\]
\end{problem}

\subsection*{Алгоритъм за намиране на минимален КДА.}
\begin{itemize}
\item
  Ако имаме даден регулярен език $L$, то ние знаем как да построим минимален автомат разпознаващ $L$.
\item
  Сега да фиксираме произволен КДА $\A = \FA$.
  Ще видим как можем да намерим минимален автомат $\A'$ разпознаващ $\L(\A)$.
  Това означава, че броят на състоянията на $\A'$ трябва да е равен на $\abs{\approx_{\L(\A)}}$.
\item
  Казваме, че две състояния $p,q$ на автомата  са {\bf еквивалентни}, означаваме $p\equiv_\A q$,
  ако \[p \equiv_\A q\ \dff\ (\forall \gamma\in \Sigma^\star)[\delta^\star(p,\gamma) \in F\ \iff\ \delta^\star(q,\gamma) \in F].\]
\item
  Релацията $\equiv_\A$ между състояния на автомата $\A$ е релация на еквивалентност. 
\item
  Нека $q_\alpha$ е състоянието, което съответства на думата $\alpha$ в $\A$, т.е.
  $\delta^\star_\A(s,\alpha) = q_\alpha$. 
  Тогава лесно се вижда, че:
  \marginpar{\ding{45} Проверете!}
  \[q_\alpha \equiv_\A q_\beta\ \iff\ \alpha\approx_{\L(\A)} \beta.\]
  Това означава, че ако в $\A$ {\bf няма недостижими състояния}, то 
  всяко състояние на $\A$ е от вида $q_\alpha$, за някое $\alpha$, и 
  \[\abs{\equiv_\A} = \abs{\approx_{\L(\A)}}.\]
\end{itemize}

% При даден език $L$ и {\bf тотален} КДА $\A = \FA$, който го разпознава, нашата цел е да построим нов КДА $\A_0$,
% който има толкова състояния колкото са класовете на еквивалентност на релацията $\approx_\L$.
% Това ще направим като ``слеем'' състоянията на $\A$, които са еквивалентни относно релацията $\equiv_\A$.
% Това означава, че всяко състояние на $\A_0$ ще отговаря на един клас на еквивалентност на релацията $\equiv_\A$.

Проблемът с намирането на класовете на еквивалентност на релацията $\equiv_\A$ е кванторът $\forall \gamma \in \Sigma^\star$
в нейната дефиницията. Алгоритъмът представлява намирането на релации $\equiv_n$, където
\[p\equiv_n q \dff (\forall\gamma\in\Sigma^{\leq n})[\delta^\star(p,\gamma) \in F\ \iff\ \delta^\star(q,\gamma) \in F],\]
където сме положили
\[\Sigma^{\leq n} = \{\alpha \in \Sigma^\star \mid \abs{\alpha} \leq n\}.\]
Релациите $\equiv_n$ представляват апроксимации на релацията $\equiv_\A$.
Обърнете внимание, че за всяко $n$, $\equiv_n$ е {\em по-груба} релация от $\equiv_{n+1}$, 
която на свой ред е по-груба от $\equiv_\A$.


Алгоритъмът строи $\equiv_n$ докато не срещнем $n$, за което $\equiv_n\ =\ \equiv_{n+1}$.
Тъй като броят на класовете на еквивалентност на $\equiv_\A$ е краен (не надминава $\abs{Q}$), то 
със сигурност ще намерим такова $n$, за което $\equiv_n\ =\ \equiv_{n+1}$.
Тогава заключаваме, че $\equiv_\A\ =\ \equiv_n$.

Понеже единствената дума с дължина $0$ e $\varepsilon$ и по определение $\delta^\star(p,\varepsilon) = p$, 
лесно се съобразява, че $\equiv_0$ има два класа на еквивалентност.
Единият е $F$, а другият е $Q\setminus F$.

\begin{prop}
  \label{pr:one-letter}
  За всеки две състояния $p,q \in Q$, и всяко $n$, $p \equiv_{n+1} q$ точно тогава, когато
  \begin{enumerate}[a)]
  \item
    $p \equiv_{n} q$ и
  \item
    $(\forall a \in \Sigma)[\delta(q,a) \equiv_{n} \delta(p,a)]$.
  \end{enumerate}
\end{prop}
\begin{proof}
  Да разгледаме следната последователност от еквивалентни преобразувания:
  % \marginpar{(стр. 99 от \cite{papadimitriou})}
  \begin{align*}
    p \equiv_{n+1} q \iff & (\forall \gamma\in\Sigma^{\leq n+1})[\delta^\star(p,\gamma)\in F \iff \delta^\star(q,\gamma) \in F] & (\text{от деф.})\\
    \iff & (\forall \gamma\in\Sigma^{\leq n})[\delta^\star(p,\gamma)\in F \iff \delta^\star(q,\gamma) \in F]\ \&\ \\
    & (\forall a\in\Sigma)(\forall \gamma\in\Sigma^{\leq n})[\delta^\star(p, a\gamma)\in F \iff \delta^\star(q, a\gamma) \in F]\\
    \iff & p \equiv_n q\ \& & (\text{от деф.})\\
    & (\forall \gamma\in\Sigma^{\leq n})[\delta^\star(\delta(p,a),\gamma)\in F \iff \delta^\star(\delta(q,a),\gamma) \in F] & (\text{\Prop{delta-star}})\\
    \iff & p \equiv_n q\ \&\ (\forall a\in\Sigma)[\delta(p,a) \equiv_n \delta(q,a)].
  \end{align*}
\end{proof}

\begin{cor}
  \marginpar{\ding{45} Докажете!}
  Ако $\equiv_n\ =\ \equiv_{n+1}$, то за всяко $k \geq n$, $\equiv_n\ =\ \equiv_k$.
\end{cor}
% \begin{proof}
%   Нека $\equiv_n\ =\ \equiv_k$, но $\equiv_n\ \neq\ \equiv_{k+1}$.
%   Това означава, че $p \equiv_n q$, но $p \not\equiv_{k+1} q$.
%   Понеже $p \equiv_k q$, от \Prop{one-letter} следва, че $\delta(p,a) \not\equiv_k \delta(q,a)$, за някое $a \in \Sigma$.
%   Но това означава, че $\delta(p,a) \not\equiv_n \delta(q,a)$, от където следва, че $p \not\equiv_{n+1} q$,
%   което е противоречие.
% \end{proof}

\begin{cor}
  \marginpar{\ding{45} Докажете!}
  Ако $\equiv_n\ =\ \equiv_{n+1}$, то $\equiv_n\ =\ \equiv_\A$.
\end{cor}

\begin{cor}
  \label{cor:delta-equiv}
  \marginpar{\ding{45} Докажете!}
  Ако $p \equiv_\A q$, то $\delta(p,a) \equiv_\A \delta(q,a)$.
\end{cor}

Нека е даден автомата $A = \FA$ и нека сме намерили $n$, 
за което $\equiv_n\ =\ \equiv_{n+1}$. Тогава $\equiv_\A\ =\ \equiv_n$.
Строим автомата $\A' = \pair{Q',\Sigma,s',\delta',F'}$ по следния начин:
\begin{itemize}
\item
  $Q' = \{[q]_{\equiv_\A} \mid q\in Q\}$;
\item
  $s' = [s]_{\equiv_\A}$;
\item
  $\delta'([q]_{\equiv_\A}, a) = [\delta(q,a)]_{\equiv_\A}$;
\item
  $F' = \{[q]_{\equiv_\A}\mid [q]_{\equiv_\A} \subseteq F\}$;
\end{itemize}

\marginpar{\ding{45} Обяснете защо $\delta'$ е функция!}
Лесно се вижда, че $\delta'$ е функция, т.е. горната дефиниция наистина задава автомат.
Също така, ако $\A$ е тотален автомат без недостижими състояния, то $\A'$ 
е тотален автомат без недостижими състояния.
Така от направените по-горе разсъждения следва, че $\abs{\equiv_\A} = \abs{\approx_{\L(\A)}}$.

% Нека първо да се убедим, че $\delta'$ е функция. Това означава да проверим, че ако
% за всеки две състояния $p$, $q$, и всяка буква $a$, 
% \[p \equiv_\A q \implies \delta(p,a) \equiv_\A \delta(q,a).\]
% Но това е точно \Cor{delta-equiv}.

\begin{prop}
  \label{pr:algorithm-myhill-nerode}
  \marginpar{\ding{45} Докажете с индукция по дължината на $\alpha$!}
  $(\forall q\in Q)(\forall \alpha\in\Sigma^\star)[\delta'^\star([q]_{\equiv_\A},\alpha) = [\delta^\star(q,\alpha)]_{\equiv_\A}]$.
\end{prop}
% \begin{hint}
%   Използвайте индукция по дължината на $\alpha$.
% \end{hint}

\begin{prop}
  \label{pr:finals-myhill-nerode}
  \marginpar{\ding{45} Обяснете!}
  Едно състояние $q \in F$ точно тогава, когато $[q]_{\equiv_\A} \subseteq F$.
\end{prop}
% \begin{proof}
%   За посоката $(\Leftarrow)$, нека $p \in [q]_\equiv \cap F \neq \emptyset$.
%   Понеже $p \equiv q$, то по дефиниция:
%   \[(\forall \gamma\in \Sigma^\star)[\delta^\star(p,\gamma) \in F\ \iff\ \delta^\star(q,\gamma) \in F].\]
%   В частност за $\gamma = \varepsilon$, $p \in F \iff q \in F$.
%   Следователно, $q \in F$.
%   Посоката $(\Rightarrow)$ е очевидна, защото $q \in [q]_\equiv$.  
% \end{proof}

\begin{prop}
  Автоматът $\A'$ е минимален автомат разпознаващ $\L(\A)$.
\end{prop}
\begin{proof}
  От направените по-горе разсъждения знаем, че $\abs{\equiv_\A} = \abs{\approx_{\L(\A)}}$.
  Затова е достатъчно е да проверим, че $\L(\A') = \L(\A)$.
  Достатъчно да се уверим, че следните еквивалентности са изпълнени:
  \begin{align*}
    \alpha \in \L(\A) & \iff \delta^\star(s,\alpha) = q\in F & (\text{от деф. на }\L(\A))\\
    & \iff [\delta^\star(s,\alpha)]_{\equiv_\A} = [q]_{\equiv_\A} \subseteq F & (\text{\Prop{finals-myhill-nerode}})\\
    & \iff \delta'^\star([s]_{\equiv_\A},\alpha) = [q]_{\equiv_\A} \subseteq F & (\text{\Prop{algorithm-myhill-nerode}})\\
    & \iff \delta'^\star([s]_{\equiv_\A},\alpha) = [q]_{\equiv_\A} \in F' & (\text{от деф. на }F')\\
    & \iff \alpha \in \L(\A') & (\text{от деф. на }\L(\A')).
  \end{align*}
\end{proof}

\begin{example}
  Да разгледаме следния краен детерминиран автомат $\A$.
  \begin{figure}[H]
    \begin{subfigure}[b]{.4\textwidth}
      \begin{tikzpicture}[->,>=stealth,thick,node distance=55pt]
        \tikzstyle{every state}=[circle,minimum size=20pt,auto]
        
        \node[initial above, state]   (0) {$0$};
        \node[state]            (1) [above right of=0]{$1$};
        \node[state]            (2) [below right of=0]{$2$};
        \node[state,accepting]  (3) [right of=1]{$3$};
        \node[state,accepting]  (4) [right of=2]{$4$};
        \node[state,accepting]  (5) [below right of=3]{$5$};
        
        
        \path 
        (0) edge  node [above] {$a$}   (1)
        (0) edge  node [below] {$b$}   (2)
        (1) edge node [above] {$a$}    (3)
        (1) edge [bend left=15] node [below] {$b$}    (4)
        (2) edge [bend left=15] node [left] {$b$}    (3)
        (2) edge node [below] {$a$}   (4)
        (4) edge  node [below] {$a,b$} (5)
        (3) edge  node [left] {$a,b$}  (5)
        (5) edge [loop above]   node [above] {$a,b$}  (5);
      \end{tikzpicture}
      \caption{Ще построим минимален автомат, разпознаващ $\L(\A)$}
    \end{subfigure}
    \qquad
    \qquad
    \begin{subfigure}[b]{0.5\textwidth}
      \begin{tikzpicture}[->,>=stealth,thick,node distance=45pt]
        \tikzstyle{every state}=[circle,minimum size=20pt,auto,scale=.9]
        
        \node[initial above, state]   (0) {$B_0$};
        \node[state]            (1) [right of=0]{$B_1$};
        \node[state,accepting]  (2) [right of=1]{$B_2$};
        
        \path 
        (0) edge [bend left=15] node [above] {$a,b$}   (1)
        (1) edge [bend left=15] node [above] {$a,b$}   (2)
        (2) edge [loop above] node [above] {$a,b$}   (2);
      \end{tikzpicture}
      \caption{Получаваме следния минимален автомат $\A_0$, $\L(\A_0) = \L(\A)$}
      \label{sub:min1}
    \end{subfigure}
  \end{figure}
  \marginpar{Съобразете, че $\L(\A) = \{\alpha \in \{a,b\}^\star \mid \abs{\alpha} \geq 2\}$.}

Ще приложим алгоритъма за минимизация за да получим минималния автомат за езика $L$.
За всяко $n = 0,1,2,\dots$, ще намерим класовете на еквивалентност на $\equiv_n$,
докато не намерим $n$, за което $\equiv_n\ =\ \equiv_{n+1}$.

\begin{itemize}
\item 
  Класовете на еквивалентност на $\equiv_0$ са два.
  Те са $A_0 = Q\setminus F = \{0,1,2\}$ и $A_1 = F = \{3,4,5\}$.
\item
  Сега да видим дали можем да разбием някои от класовете на еквивалентност на $\equiv_0$.
  
  \begin{tabular}{|c|c|c|c|c|c|c|}
    \hline
    $Q$ & $0$ & $1$ & $2$ & $3^\star$ & $4^\star$ & $5^\star$ \\
    \hline
    \hline
    $\equiv_0$ & $A_0$ & $A_0$ & $A_0$ & $A_1$ & $A_1$ & $A_1$\\
    \hline
    $a$ & $A_0$& $A_1$ & $A_1$ & $A_1$ & $A_1$ & $A_1$\\
    \hline
    $b$ & $A_0$& $A_1$ & $A_1$ & $A_1$ & $A_1$ & $A_1$\\
    \hline
  \end{tabular}

  Виждаме, че $0 \not\equiv_1 1$ и $1 \equiv_1 2$.
  Класовете на еквивалентност на $\equiv_1$ са 
  $B_0 = \{0\}$, $B_1 = \{1,2\}$, $B_2 = \{3,4,5\}$.
\item
  Сега да видим дали можем да разбием някои от класовете на еквивалентност на $\equiv_1$.
  
  \begin{tabular}{|c|c|c|c|c|c|c|}
    \hline
    $Q$ & $0$ & $1$ & $2$ & $3^\star$ & $4^\star$ & $5^\star$ \\
    \hline
    \hline
    $\equiv_1$ & $B_0$ & $B_1$ & $B_1$ & $B_2$ & $B_2$ & $B_2$\\
    \hline
    $a$ & $B_1$ & $B_2$ & $B_2$ & $B_2$ & $B_2$ & $B_2$\\
    \hline
    $b$ & $B_1$ & $B_2$ & $B_2$ & $B_2$ & $B_2$ & $B_2$\\
    \hline
  \end{tabular}

  Виждаме, че $\equiv_1\ =\ \equiv_2$.
  \marginpar{Получаваме, че $\equiv_\A\ =\ \equiv_1$}
  Следователно, минималният автомат има три състояния.
  Той е изобразен на Фигура \ref{sub:min1}.  
  Минималният автомат може да се представи и таблично:
  
  \begin{tabular}{|c|c|c|c|c|c|c|}
    % \hline
    % $Q$ & $0$ & $1$ & $2$ & $3^\star$ & $4^\star$ & $5^\star$ \\
    % \hline
    \hline
    $\delta$ & $B_0$ & $B_1$ & $B_2$ \\
    \hline
    $a$ & $B_1$ & $B_2$ & $B_2$ \\
    \hline
    $b$ & $B_1$ & $B_2$ & $B_2$ \\
    \hline
  \end{tabular}
\end{itemize}
\end{example}

\begin{example}
  Да разгледаме следния краен детерминиран автомат $\A$.
  \begin{figure}[H]
    % \begin{center}
    \begin{subfigure}[b]{0.4\textwidth}
      \begin{tikzpicture}[->,>=stealth,thick,node distance=55pt]
        \tikzstyle{every state}=[circle,minimum size=20pt,auto]
        
        \node[initial above, state]   (0) {$0$};
        \node[state,accepting]        (1) [above right of=0]{$1$};
        \node[state,accepting]        (2) [below right of=0]{$2$};
        \node[state]                  (3) [right of=1]{$3$};
        \node[state]                  (4) [right of=2]{$4$};
        \node[state,accepting]        (5) [below right of=3]{$5$};
        
        \path 
        (0) edge node [below] {$a$}   (1)
            edge node [below] {$b$}   (2)
        (1) edge node [above] {$a$}    (3)
            edge [bend left=15] node [below] {$b$}    (4)
        (2) edge [bend left=15] node [left] {$b$}    (3)
            edge node [below] {$a$}   (4)
        (4) edge node [below] {$a,b$} (5)
        (3) edge node [left] {$a,b$}  (5)
        (5) edge [loop above]   node [above] {$a,b$}  (5);
      \end{tikzpicture}
      \caption{Ще построим минимален автомат, разпознаващ $\L(\A)$}
    \end{subfigure}
    \qquad
    \qquad
    \begin{subfigure}[b]{0.4\textwidth}
      \begin{tikzpicture}[->,>=stealth,thick,node distance=45pt]
        \tikzstyle{every state}=[circle,minimum size=20pt,auto,scale=.9]
        
        \node[initial above, state]   (0) {$C_0$};
        \node[state,accepting]  (1) [right of=0]{$C_1$};
        \node[state]            (2) [right of=1]{$C_2$};
        \node[state,accepting]  (3) [right of=2]{$C_3$};
                
        \path 
        (0) edge [bend left=15] node [above] {$a,b$}   (1)
        (1) edge [bend left=15] node [above] {$a,b$}   (2)
        (2) edge [bend left=15] node [above] {$a,b$}   (3)
        (3) edge [loop above]   node [above] {$a,b$}   (3);
      \end{tikzpicture}
      \caption{Получаваме следния минимален автомат $\A_0$, $\L(\A_0) = \L(\A)$}
      \label{sub:min2}
    \end{subfigure}
  \end{figure}

  \marginpar{Съобразете, че $\L(\A) = \{a,b\} \cup \{\alpha \in \{a,b\}^\star \mid \abs{\alpha} \geq 3\}$.}
  
  Отново следваме същата процедура за минимизация.
  Ще намерим класовете на еквивалентност на $\equiv_n$,
  докато не намерим $n$, за което $\equiv_n\ =\ \equiv_{n+1}$.
  \begin{itemize}
  \item
    Класовете на екиваленост на $\equiv_0$ са 
    $A_0 = Q\setminus F = \{0,3,4\}$ и $A_1 = F = \{1,2,5\}$.
  \item
    Разбиваме класовете на еквивалентност на $\equiv_0$.
    
    \begin{tabular}{|c|c|c|c|c|c|c|}
      \hline
      $Q$ & 0 & $1^\star$ & $2^\star$ & 3 & 4 & $5^\star$ \\
      \hline
      \hline
      $\equiv_0$ & $A_0$ & $A_1$ & $A_1$ & $A_0$ & $A_0$ & $A_1$\\
      \hline
      $a$ & $A_1$& $A_0$ & $A_0$ & $A_1$ & $A_1$ & $A_1$\\
      \hline
      $b$ & $A_1$& $A_0$ & $A_0$ & $A_1$ & $A_1$ & $A_1$\\
      \hline
    \end{tabular}
    
    Виждаме, че $1 \not\equiv_1 5$ и $1 \equiv_0 5$.
    Следователно, $\equiv_0\ \neq\ \equiv_1$.
    Класовете на еквивалентност на $\equiv_1$ са 
    $B_0 = \{0,3,4\}$, $B_1 = \{1,2\}$, $B_2 = \{5\}$.
  \item
    Сега се опитваме да разбием класовете на еквивалентност на $\equiv_1$.

    \begin{tabular}{|c|c|c|c|c|c|c|}
      \hline
      $Q$ & 0 & $1^\star$ & $2^\star$ & 3 & 4 & $5^\star$ \\
      \hline
      \hline
      $\equiv_1$ & $B_0$ & $B_1$ & $B_1$ & $B_0$ & $B_0$ & $B_2$\\
      \hline
      $a$ & $B_1$ & $B_0$ & $B_0$ & $B_2$ & $B_2$ & $B_2$\\
      \hline
      $b$ & $B_1$ & $B_0$ & $B_0$ & $B_2$ & $B_2$ & $B_2$\\
      \hline
    \end{tabular}
    
    Имаме, че $0 \equiv_1 3$, но $0 \not\equiv_2 3$. Следователно $\equiv_1\ \neq\ \equiv_2$.
    Класовете на еквивалентност на $\equiv_2$ са 
    $C_0 = \{0\}$, $C_1 = \{1,2\}$, $C_2 = \{3,4\}$, $C_3 = \{5\}$.
  \item
    Отново опитваме да разбием класовете на $\equiv_2$.

      \begin{tabular}{|c|c|c|c|c|c|c|}
        \hline
        $Q$ & 0 & $1^\star$ & $2^\star$ & 3 & 4 & $5^\star$ \\
        \hline
        \hline
        $\equiv_2$ & $C_0$ & $C_1$ & $C_1$ & $C_2$ & $C_2$ & $C_3$\\
        \hline
        $a$ & $C_1$ & $C_2$ & $C_2$ & $C_3$ & $C_3$ & $C_3$\\
        \hline
        $b$ & $C_1$ & $C_2$ & $C_2$ & $C_3$ & $C_3$ & $C_3$\\
        \hline
      \end{tabular}
      
      Виждаме, че не можем да разбием $C_1$ или $C_2$.
      \marginpar{Получаваме, че $\equiv_\A\ =\ \equiv_2$}
      Следователно, $\equiv_2\ =\ \equiv_3$ и минималният автомат разпознаващ езика $L$
      има четири състояния. Вижте Фигура \ref{sub:min2} за преходите на минималния автомат.
      Минималният автомат може да се представи и таблично:

      \begin{tabular}{|c|c|c|c|c|}
        \hline
        $\delta$ & $C_0$ & $C_1$ & $C_2$ & $C_3$ \\
        \hline
        $a$ & $C_1$ & $C_2$ & $C_3$ & $C_3$ \\
        \hline
        $b$ & $C_1$ & $C_2$ & $C_3$ & $C_3$ \\
        \hline
      \end{tabular}
      
  \end{itemize}
\end{example}

% \section{Регулярни граматики}
% \index{граматика!регулярна}
% \section*{Библиография}

% Основни източници в тази глава са:
% \begin{itemize}
% \item 
%   глави 2 и 3 от \cite{hopcroft1}.
% \item
%   глави 2,3 и 4 от \cite{hopcroft2}.
% \item
%   Глава 1 от \cite{sipser1}.
% \item
%   глава 2 от \cite{papadimitriou}.
% \item
%   Първа част на \cite{kozen}. Въпросът за минимизация на автомат е разгледан подробно.
% \end{itemize}

%\bibentry{sipser1}

% \newpage
% \cite{min-hopcroft}
% \section{Въпроси}

% Вярно ли е, че:
% \begin{itemize}
%   % \item
% %   \marginpar{Не}
% %   езикът $\{a^nb^n\mid n \in \Nat \}$ е регулярен?
% % \item
% %   \marginpar{Не}
% %   езикът $\{a^nb^k\mid n > k\}$ е регулярен?
% % \item
% %   \marginpar{Не}
% %   езикът $\{a^{n^2}\mid n \in \Nat\}$ е регулярен?
% \item
%   \marginpar{Да}
%   за всеки два регулярни езика $R_1, R_2$, то $R_1 \setminus R_2$ е регулярен ?
% \item
%   \marginpar{Да}
%   за всеки краен език $F$ и всеки регулярен $R$, то $R\setminus F$ е регулярен ?
% \item
%   \marginpar{Да}
%   за всеки краен език $F$ и всеки рег. $R$, то $R\cup (\Sigma^\star \setminus F)$ е регулярен ?
% \item
%   \marginpar{Да}
%   съществува регулярен език $R$ и нерегулярен $K$, за които $R\cap K$ не е регулярен ?
% \item
%   \marginpar{Да}
%   съществува регулярен език $R$ и нерегулярен $K$, за които $R\setminus K$ не е регулярен ?
% \item
%   \marginpar{Не}
%   за всеки регулярен език $R$ и всеки $K \subseteq R$, то $R\setminus K$ е регулярен ?
% \item
%   \marginpar{Не}
%   Езикът $L = \{\omega \in \{a,b\}^\star \mid n_a(\omega) \text{ не дели }n_b(\omega)\}$ е регулярен?
% \item
%   \marginpar{Да}
%   съществува алгоритъм, който може да провери дали за даден регулярен израз $r$
%   е изпълнено, че $\abs{\L(r)} = 0$.
% \item
%   \marginpar{Да}
%   съществува алгоритъм, който може да провери дали за даден регулярен израз $r$
%   е изпълнено, че $\abs{\L(r)} < \infty$.
% \item
%   \marginpar{Да}
%   съществува алгоритъм, който може да провери дали за даден регулярен израз $r$
%   е изпълнено, че $\abs{\L(r)} = \infty$.
% \item
%   \marginpar{Да}
%   съществува алгоритъм, който може да провери дали за дадени регулярни изрази $r_1$ и $r_2$
%   е изпълнено, че $\L(r_1) = \L(r_2)$.
% \item
%   съществува алгоритъм, който може да провери дали за дадени регулярни изрази $r_1$ и $r_2$
%   е изпълнено, че $\L(r_1) \neq \L(r_2)$.
% \item
%   съществува алгоритъм, който може да провери дали за дадени регулярни изрази $r_1$ и $r_2$
%   е изпълнено, че $\L(r_1) \subseteq \L(r_2)$.
% \item
%   съществува алгоритъм, който може да провери дали за дадени регулярни изрази $r_1$ и $r_2$
%   е изпълнено, че $\L(r_1) \subsetneq \L(r_2)$.
% \item
%   съществува алгоритъм, който може да провери дали за дадени регулярни изрази $r_1$ и $r_2$
%   е изпълнено, че $\L(r_1) \cap \L(r_2) = \emptyset$.
% \item
%   съществува алгоритъм, който може да провери дали за дадени регулярни изрази $r_1$ и $r_2$
%   е изпълнено, че $\L(r_1) \cap \L(r_2) \neq \emptyset$.
% \item
%   съществува алгоритъм, който може да провери дали за дадени регулярни изрази $r_1$ и $r_2$
%   е изпълнено, че $\L(r_1) \cup \L(r_2) = \emptyset$.
% \item
%   съществува алгоритъм, който може да провери дали за дадени регулярни изрази $r_1$ и $r_2$
%   е изпълнено, че $\L(r_1) \cup \L(r_2) \neq \emptyset$.
% \item
%   съществува алгоритъм, който може да провери дали за дадени регулярни изрази $r_1$ и $r_2$
%   е изпълнено, че $\L(r_1) \setminus \L(r_2) = \emptyset$.
% \item
%   съществува алгоритъм, който може да провери дали за дадени регулярни изрази $r_1$ и $r_2$
%   е изпълнено, че $\L(r_1) \setminus \L(r_2) \neq \emptyset$.
% \end{itemize}

% \section{Домашна работа}

% \begin{itemize}
% \item
%   Вход - файл, в който е записан регулярен израз
% \item
%   Преобразуване на регулярния израз в обратен полски запис.
%   (\href{http://en.wikipedia.org/wiki/Shunting-yard_algorithm}{тук} 
%   добре е обяснено как става за произволни аритмечни изрази)
% \item
%   Строене на краен детерминиран автомат по регулярния израз.
% \item
%   Извеждане на автомата във формат за програмата \href{http://graphviz.org}{graphviz}.
%   (вижте \href{http://sundarpillay.blogspot.com/2012/02/graphviz-and-finite-automata-diagrams_05.html}{пример})
% \end{itemize}


%%% Local Variables: 
%%% mode: latex
%%% TeX-master: "EAI"
%%% End: 


\section{Допълнителни задачи}

\begin{problem}
  Докажете, че следните езици са регулярни:
  \marginpar{Озн. $N_a(\omega)$ - броят на срещанията на буквата $a$ в думата $\omega$}
  \begin{enumerate}[a)]
  \item
    $L = \{\alpha \in \{a,b\}^\star \mid \abs{N_a(\omega) - N_b(\omega)} \leq 2 \text{ за всяка представка $\omega$ на $\alpha$}\}$;
  \item
    $L = \{\alpha \in \{a,b\}^\star \mid \abs{N_a(\omega) - N_b(\omega)} > 2 \text{ за някоя представка $\omega$ на $\alpha$}\}$;
  \item
    $L = \{\alpha \in \{a,b\}^\star \mid \abs{N_a(\omega) - N_b(\omega)} > 2 \text{ за някоя наставка $\omega$ на $\alpha$}\}$.
  \end{enumerate}
\end{problem}


\begin{problem}
  Нека $\Sigma = \{a,b\}$.  Проверете дали $L$ е регулярен, където
  \begin{enumerate}[a)]
  \item
    $L = \{\alpha^R \mid \alpha \in L_0\}$, където $L_0$ е регулярен;
  \item
    \marginpar{$\alpha = a^pb^p$}
    $L = \{a^ib^i\ \mid\ i\in\Nat\}$;
  \item
    $L = \{a^ib^i\ \mid\ i,j\in\Nat\ \&\ i\neq j\}$;
  \item
    \marginpar{$\alpha = a^{p+1}b^p$.}
    $L = \{a^ib^j\ \mid\ i > j\}$;
  \item
    $L = \{a^nb^m \mid n\mbox{ дели }m\}$.
  \item
    $L = \{a^{2n}\ \mid\ n\geq 1\}$;
  \item
    $L = \{a^mb^na^{m+n}\ \mid\ m\geq 1\ \&\ n\geq 1\}$;
  \item
    $L = \{a^{n.m}\mid n,m\mbox{ са прости числа}\}$;
  \item
    % \marginpar{$N_x(\omega)$ - брой срещания на буквата $x$ в думата $\omega$}
    $L = \{\omega\in\{a,b\}^\star \mid N_a(\omega) = N_b(\omega)\}$;
  \item
    \marginpar{$\alpha = a^pba^pb$}
    $L = \{\omega\omega\mid \omega\in\{a,b\}^\star\}$;
  \item
    $L = \{\omega\omega^R\mid \omega\in\{a,b\}^\star\}$;
  \item
    $L = \{\alpha\beta\beta \in \{a,b\}^\star\mid \beta \neq \varepsilon\}$;
  \item
    $L = \{a^nb^nc^n\mid n\geq 0\}$;
  \item
    $L = \{\omega\omega\omega\mid \omega\in \Sigma^\star\}$;
  \item
    $L = \{a^{2^n}\mid n\geq 0\}$;
  \item
    $L = \{a^mb^n\mid n\neq m\}$;
  \item
    $L = \{a^{n!}b^{n!}\mid n\neq 1\}$;
  \item
    $L = \{a^{f_n} \mid f_0 = f_1 = 1\ \&\ f_{n+2} = f_{n+1} + f_{n}\}$;
  \item
    $L = \{\alpha \in \{a,b\}^\star \mid \abs{N_a(\alpha) - N_b(\alpha)} \leq 2\}$;
  \item
    $L = \{\alpha \in \{a,b\}^\star \mid \alpha = \alpha\beta\alpha\ \&\ \abs{\beta} \leq \abs{\alpha}\}$;
  \item
    $L = \{\alpha \in \{a,b\}^\star \mid \alpha = \beta\gamma\gamma^R\ \&\ \abs{\beta} \leq \abs{\gamma}\}$;
  \item
    $L = \{c^ka^nb^m \mid k,m,n > 0\ \&\ n \neq m\}$;
  \item
    $L = \{c^ka^nb^n \mid k > 0\ \&\ n \geq 0\}\cup\{a,b\}^\star$;
  \item
    $L = \{\omega \in \{a,b\}^\star \mid N_a(\omega)\text{ не дели }N_b(\omega)\}$;
  \item
    $L = \{\omega \in \{a,b\}^\star \mid N_a(\omega) < N_b(\omega)\}$;
  \item
    $L = \{\omega \in \{a,b\}^\star \mid N_a(\omega) = 2N_b(\omega)\}$;
  \item
    $L = \{\omega \in \{a,b\}^\star \mid \abs{N_a(\omega) - N_b(\omega)} \leq 3\}$.
  \end{enumerate}
\end{problem}

\begin{problem}
  Нека $L$ е регулярен език. Докажете, че 
  \[\text{Infix}(L) = \{\alpha \mid (\exists \beta,\gamma)[\beta\alpha\gamma \in L]\}\]
  също е регулярен език.
\end{problem}
\begin{hint}
  Най-лесно е да се построи автомат за $\text{Infix}(L)$ като се използва автомата за $L$.
\end{hint}

\begin{problem}
  Нека $\Sigma = \{a,b,c,d\}$.
  Да се докаже, че езика 
  \[L = \{a_1a_2\cdots a_{2n} \in \Sigma^\star \mid (\forall j \in [1,n])[a_{2j-1} = a_{2j}]\ \&\ d\text{ се среща $\leq 3$ пъти}\}\]
  е регулярен.
\end{problem}

\begin{problem}
  Нека $L_1$ и $L_2$ са регулярни езици. Докажете, че $L$ също е регулярен език, където
  \[L = \{\alpha \mid (\exists \beta,\gamma)[\beta\alpha\gamma \in L_1]\ \&\ (\alpha \in L_2 \vee \alpha^R \in L_2)\}.\]
\end{problem}

\begin{dfn}
  Да фиксираме две азбуки $\Sigma_1$ и $\Sigma_2$.
  Хомоморфизъм е изображение $h:\Sigma^\star_1 \to \Sigma^\star_2$ със свойството, че
  за всеки две думи $\alpha,\beta\in\Sigma^\star_1$,
  \[h(\alpha\beta) = h(\alpha)\cdot h(\beta).\]
\end{dfn}

Лесно се съобразява, че за всеки хомоморфизъм $h$, $h(\varepsilon) = \varepsilon$.

\begin{problem}
  Нека $L \subseteq \Sigma^\star_1$ е регулярен език и $h:\Sigma^\star_1\to\Sigma^\star_2$ е хомоморфизъм.
  Тогава
  $h(L) = \{h(\alpha) \in \Sigma^\star_2 \mid \alpha \in L\}$ е регулярен.
\end{problem}
\begin{hint}
  Индукция по построението на регулярни езици.
  % \begin{itemize}[-]
  % \item 
  %   За $L = \{a\}$, $h(L) = \{h(a)\}$.
  % \item
  %   $h(\emptyset) = \emptyset$.
  % \item
  %   Нека $L_1 = \L(r_1)$ и $L_2 = \L(r_2)$.
  %   Ще докажем, че $h(\L(r_1\cdot r_2))$ е регулярен.
  %   \begin{align*}
  %     h(\L(r_1\cdot r_2)) & = h(L_1\cdot L_2) & (\text{деф. на конкатенация})\\
  %     & = \{h(\gamma) \mid \gamma \in L_1 \cdot L_2\}\\
  %     & = \{h(\alpha\beta) \mid \alpha\in L_1\ \&\ \beta\in L_2\}\\
  %     & = \{h(\alpha)\cdot h(\beta) \mid \alpha \in L_1\ \&\ \beta \in L_2\} & (h\text{ е хомоморфизъм})\\
  %     & = \{\omega\gamma \mid \omega \in h(L_1)\ \&\ \gamma \in h(L_2)\}\\
  %     & = h(L_1)\cdot h(L_2).
  %   \end{align*}
  %   По И.П. имаме, че $h(L_1)$ и $h(L_2)$ са регулярни езици.
  %   Следователно, 
  %   \[h(\L(r_1\cdot r_2)) = h(L_1)\cdot h(L_2)\]
  %   е регулярен език.
  % \item
  %   От горното свойство имаме също, че за всяко $n$, $h(L^n) = h(L)^n$.
  % \item
  %   Освен това, 
  %   \begin{align*}
  %     h(\bigcup_n L_n) & = \{h(\alpha) \mid (\exists n)[\alpha \in L_n]\}\\
  %     & = \bigcup \{h(\alpha) \mid \alpha \in L_n\}\\
  %     & = \bigcup_n h(L_n).
  %   \end{align*}
  % \item
  %   Нека $L = \L(r^\star)$.
  %   Ще докажем, че $h(L^\star)$ е регулярен език.
  %   \begin{align*}
  %     h(L^\star) & = h(\bigcup_n L^n) & (\text{деф. на звезда на Клини})\\
  %     & = \bigcup_n h(L^n) & (\text{от горното свойство})\\
  %     & = \bigcup_n h(L)^n & (\text{от по-горното свойство})\\
  %     & = h(L)^\star & (\text{по деф.}).
  %   \end{align*}
  % \end{itemize}
\end{hint}

\begin{problem}
  Нека $L\subseteq \Sigma^\star_2$ е регулярен език и $h:\Sigma^\star_1\to\Sigma^\star_2$ е хомоморфизъм.
  Тогава езикът
  $h^{-1}(L) = \{\alpha \in \Sigma^\star_1 \mid h(\alpha) \in L\}$ е регулярен.  
\end{problem}
\begin{hint}
  Конструкция на автомат за $h^{-1}(L)$ при даден автомат за $L$.
  % Нека $\A$ е КДА разпознаващ езика $L$.
  % Ще построим $\A' = \pair{Q,\Sigma_1, \delta', s, F}$,
  % където дефинираме функцията на преходите $\delta'$ като $\delta'(q,a) = \delta^\star(q,h(a))$.
  % Понеже $h$ е хомоморфизъм, лесно се доказва с индукция по дължината на думата $\alpha \in \Sigma^\star_1$,
  % че $\delta'^\star(q,\alpha) = \delta^\star(q,h(\alpha))$.
  % Сега лесно се вижда, че $h^{-1}(\L(\A)) = \L(\A')$, защото:
  % \begin{align*}
  %   \alpha \in \L(\A') & \iff \delta'^\star(s,\alpha) \in F\\
  %   & \iff \delta^\star(s,h(\alpha)) \in F\\
  %   & \iff h(\alpha) \in \L(\A)\\
  %   & \iff \alpha \in h^{-1}(\L(\A)).
  % \end{align*}
\end{hint}

\begin{problem}
  \marginpar{\cite{papadimitriou} стр. 84}
  При дадени езици $L$, $L'$ над азбуката $\Sigma$, да разгледаме:
  \begin{enumerate}[a)]
  \item
    $\mbox{Pref}(L) = \{\alpha \in \Sigma^\star \mid (\exists \beta \in \Sigma^\star)[\alpha\beta \in L]\}$;
  \item
    $\mbox{Suf}(L) = \{\beta \in \Sigma^\star \mid (\exists \alpha \in \Sigma^\star)[\alpha\beta \in L]\}$;
  \item
    $\text{Infix}(L) = \{\alpha \mid (\exists \beta,\gamma)[\beta\alpha\gamma \in L]\}$;
  \item 
    $\frac{1}{2}(L) = \{\omega \in \Sigma^\star \mid (\exists \alpha \in \Sigma^\star)[\omega\alpha \in L\ \&\ \abs{\omega} = \abs{\alpha}]\}$;
  \item
  %   $\frac{1}{3}(L) = \{\omega \in \Sigma^\star \mid (\exists \alpha,\beta \in \Sigma^\star)[\omega\alpha\beta \in L\ \&\ \abs{\omega} = \abs{\alpha} = \abs{\beta}]\}$;
  % \item
    $L/L' = \{\alpha \in \Sigma^\star \mid (\exists \beta \in L')[\alpha\beta \in L]\}$;
  \item
    $\mbox{Max}(L) = \{\alpha \in \Sigma^\star \mid (\forall \beta\in\Sigma^\star)[\beta \neq \varepsilon\implies \alpha\beta \not\in L]\}$.
  \end{enumerate}
  За всички тези езици, докажете, че са регулярни при условие, че $L$ и $L'$ са регулярни.
  \marginpar{Тази конструкция няма да бъде ефективна}
  Освен това, докажете, че $L/L'$ е регулярен и при условието, че $L$ е регулярен, но $L'$ е произволен език.
\end{problem}
\begin{hint}
  \begin{enumerate}[a)]
  \item 
    Индукция по дефиницията на регулярен израз.
  \item[в)]
    Най-лесно е да се построи автомат за $\text{Infix}(L)$ като се използва автомата за $L$.
  \item[г)]
    Конструкция с автомат за $L$ и автомат за $L^R$.
  \end{enumerate}
\end{hint}

\begin{problem}
  За даден език $L$ над азбуката $\Sigma$, да разгледаме езиците:
  \begin{enumerate}[a)]
  \item
    $L' = \{\alpha \mid (\exists \beta\in\Sigma^\star)[\abs{\alpha} = 2\abs{\beta}\ \&\ \alpha\beta \in L]\}$;
  \item 
    $L'' = \{\alpha \mid (\exists \beta\in\Sigma^\star)[2\abs{\alpha} = \abs{\beta}\ \&\ \alpha\beta \in L]\}$;
  \item 
    $\frac{1}{3}(L) = \{\alpha \mid (\exists \beta,\gamma)[\abs{\alpha} = \abs{\beta} = \abs{\gamma}\ \&\ \alpha\beta\gamma \in L]\}$;
  \item
    $\frac{2}{3}(L) = \{\beta \mid (\exists \beta,\gamma)[\abs{\alpha} = \abs{\beta} = \abs{\gamma}\ \&\ \alpha\beta\gamma \in L]\}$;
  \item
    $\frac{3}{3}(L) = \{\gamma \mid (\exists \beta,\gamma)[\abs{\alpha} = \abs{\beta} = \abs{\gamma}\ \&\ \alpha\beta\gamma \in L]\}$;
  \item
    $\sqrt{L} = \{\alpha \mid (\exists \beta)[\abs{\beta} = \abs{\alpha}^2\ \&\ \alpha\beta \in L]\}$.
  \end{enumerate}
  Проверете, ако $L$ е регулярен, то кои от горните езици също са регулярни.
\end{problem}


\begin{problem}
  \marginpar{(\cite{sipser1}, стр. 90)}
  Да разгледаме езика
  \[L = \{\omega \in \{0,1\}^\star \mid \omega\text{ съдържа равен брой поднизове }01\text{ и }10\}.\]
  Например, $101 \in L$, защото съдържа по веднъж $10$ и $01$.
  $1010 \not\in  L$, защото съдържа два пъти $10$ и само веднъж $01$.
  Докажете, че $L$ е регулярен.
\end{problem}

% \begin{problem}
%   \marginpar{(\cite{kozen}, стр. 75)}
%   Да фиксираме азбука само с един символ $\Sigma = \{a\}$.
%   Множеството $U$ е {\em породен от аритметична прогресия}, ако съществуват числа $q \geq 0$ и $p > 0$,
%   такива че $(\forall n \geq q)[n \in U\ \iff\ n+p \in U]$.
%   Докажете, че $L \subseteq \{a\}^\star$ е регулярен точно тогава, когато множеството $\{m \mid a^m \in L\}$
%   е породено от аритметична прогресия.
% \end{problem}
% \begin{hint}
%   Разгледайте КДА за $L$.
% \end{hint}

\begin{problem}
  \marginpar{(\cite{kozen}, стр. 75; \cite{papadimitriou}, стр. 89)}
  Да фиксираме азбука само с един символ $\Sigma = \{a\}$.
  Да положим за всяко $p,q\in\Nat$, 
  \[\L(p,q) = \{a^k \mid (\exists n\in\Nat)[k = p+q\cdot n]\}.\]
  Ако за един език $L$ съществуват константи $p_1,\dots,p_k$ и $q_1,\dots,q_k$, такива че 
  \[L = \bigcup_{1\leq i \leq k} \L(p_i,q_i),\]
  то казваме, че $L$ е {\em породен от аритметични прогресии}.
  \begin{enumerate}[a)]
  \item 
    Докажете, че $L \subseteq \{a\}^\star$ е регулярен език точно тогава, когато $L$ е породен от аритметична прогресия.
  \item
    За {\em произволна} азбука $\Sigma$, докажете, че ако $L \subseteq \Sigma^\star$ е регулярен език,
    то езикът $\{a^{\abs{\omega}} \mid \omega \in L\}$  е породен от аритметични прогресии.
  \end{enumerate}
\end{problem}
\begin{hint}
  \begin{enumerate}[a)]
  \item 
    За едната посока, разгледайте КДА за $L$.
  \item
    За втората част, разгледайте $h:\Sigma\to\{a\}$ деф. като $(\forall b\in\Sigma)[h(b) = a]$.
    Докажете, че $h$ поражда хомоморфизъм между $\Sigma^\star$ и $\{a\}^\star$.
    Тогава $h(L) = \{a^{\abs{\omega}} \mid \omega \in L\}$, а
    ние знаем, че регулярните езици са затворени относно хомоморфни образи.  
  \end{enumerate}
\end{hint}

% \begin{hint}
%   \begin{itemize}
%   \item 
%     Докажете, че за всяко $p,q \in \Nat$, $L(p,q)$ е регулярен език.
%   \item
%     Докажете, че за крайно много $p_0,\dots,p_k$, $q_0,\dots,q_k$,
%     $\bigcup_{i \leq k}L(p_i,q_i)$ е регулярен език.
%   \item
%     С индукция по построението на регулярните езици, 
%     докажете, че ако $L$ е регулярен, то $L$ може да се представи
%     като крайно обединение на езици породени от аритметични прогресии.
%     Съществената част от доказателството се състои в следното:
%     \begin{itemize}
%     \item 
%       \marginpar{$L(p_1,q_1)\cdot L(p_2,q_2) = L(p_1+p_2,\mbox{НОД}(q_1,q_2))\setminus F$, където $F$ е крайно м-во, и ако $q_1 = q_2$, то $F = \emptyset$}
%       езикът $L(p_1,q_1) \cdot L(p_2,q_2)$ може да се представи като крайно обединение 
%       на езици породени от артиметични прогресии.
%     \item
%       езикът $L(p,q)^\star$ може да се представи като крайно обединение 
%       на езици породени от артиметични прогресии.
%     \end{itemize}
%   \end{itemize}
% \end{hint}



\begin{problem}
  Да разгледаме азбуката:
  \[\Sigma_3 = \left\{\begin{bmatrix} 0\\0\\0\end{bmatrix},\begin{bmatrix} 0\\0\\1\end{bmatrix},\begin{bmatrix} 0\\1\\0\end{bmatrix},\begin{bmatrix} 0\\1\\1\end{bmatrix},\dots,\begin{bmatrix} 1\\1\\1\end{bmatrix}\right\}.\]
  Докажете, че 
  $L = \left\{\begin{bmatrix} \alpha\\\beta\\\gamma\end{bmatrix} \in \Sigma^\star_3 \mid \alpha_{(2)}+\beta_{(2)} = \gamma_{(2)}\right\}$
  е автоматен език.
\end{problem}
% \begin{hint}
%   По-удобно е да построим автомат $\A$, $\L(\A) = L^R$.
%   Да започнем с състоянието $q_{\scriptscriptstyle{=}}$, за което искаме да имаме свойството, че за произволно състояние $q$,
%   \[\delta^\star(q, \tiny{ \begin{bmatrix} \alpha\\ \beta \\ \gamma\end{bmatrix} }) = q_{\scriptscriptstyle{=}}  \iff \alpha^R_{(2)} + \beta^R_{(2)} = \gamma^R_{(2)}.\]
%   Понеже за $\varepsilon + \varepsilon = \varepsilon$, състоянието $q_{\scriptscriptstyle{=}}$ ще бъде начално и финално за $\A$.
%   \begin{itemize}
%   \item 
%     Нека $\alpha_{(2)}+\beta_{(2)} = \gamma_{(2)}$. Тогава:
%     \begin{itemize}
%     \item 
%       $0\alpha + 0\beta = 0\gamma$;
%       \marginpar{$\delta(q_{\scriptscriptstyle{=}},\tiny{ \begin{bmatrix} 0\\ 0 \\ 0\end{bmatrix} }) = q_{\scriptscriptstyle{=}}$}
%     \item
%       $0\alpha + 1\beta = 1\gamma$;
%       \marginpar{$\delta(q_{\scriptscriptstyle{=}},\tiny{ \begin{bmatrix} 0\\ 0 \\ 0\end{bmatrix} }) = q_{\scriptscriptstyle{=}}$}
%     \item
%       $1\alpha + 0\beta = 1\gamma$;
%       \marginpar{$\delta(q_{\scriptscriptstyle{=}},\tiny{ \begin{bmatrix} 1\\ 0 \\ 1\end{bmatrix} }) = q_{\scriptscriptstyle{=}}$}
%     \item
%       $1\alpha + 1\beta = 10\gamma$. Този случай е по-специален и трябва да бъде разгледан отделно.
%       Трябва да отидем в състояние $q_1$, в което ще помним, че третия ред трябва да започва с $1$-ца.
%       \marginpar{$\delta(q_{\scriptscriptstyle{=}},\tiny{ \begin{bmatrix} 1\\ 1 \\ 0\end{bmatrix} }) = q_1$}
%     \item
%       За останалите $x \in \Sigma_3$, $\delta(q_{\scriptscriptstyle{=}},x) = q_{err}$,
%       където $q_{err}$ е състоянието, от което не можем да излезем.
%       % Остават $\delta(q_{\scriptscriptstyle{=}},\tiny{ \begin{bmatrix} 0\\ 1 \\ 0\end{bmatrix} }) = \delta(q_{\scriptscriptstyle{=}},\tiny{ \begin{bmatrix} 1\\ 0 \\ 0\end{bmatrix} }) = \delta(q_{\scriptscriptstyle{=}},\tiny{ \begin{bmatrix} 1\\ 1 \\ 1\end{bmatrix} }) = \delta(q_{\scriptscriptstyle{=}},\tiny{ \begin{bmatrix} 0\\ 0 \\ 1\end{bmatrix}}) = q_{err}$;
%     \end{itemize}
%   \item
%     Горните разглеждания ни подсказват, че ще ни трябва и състояние $q_1$, за което искаме да е изпълнено свойството,
%     че за произволно $q$,
%     \[\delta^\star(q, \tiny{ \begin{bmatrix} \alpha\\ \beta \\ \gamma\end{bmatrix} }) = q_{\scriptscriptstyle{1}}  \iff \alpha^R_{(2)} + \beta^R_{(2)} = 1\gamma^R_{(2)}.\]
%     Да разгледаме сега случая $\alpha + \beta = 1\gamma$. Тогава:
%     \begin{itemize}
%     \item 
%       \marginpar{$\delta(q_1,\tiny{ \begin{bmatrix} 0\\ 0 \\ 1\end{bmatrix} }) = q_{\scriptscriptstyle{=}}$}
%       Очевидно е, че $0\alpha + 0\beta = 1\gamma$;
%     \item
%       \marginpar{$\delta(q_1,\tiny{ \begin{bmatrix} 1\\ 1 \\ 1\end{bmatrix} }) = q_{1}$}
%       $1\alpha + 1\beta = 11\gamma$;
%     \item
%       \marginpar{$\delta(q_1,\tiny{ \begin{bmatrix} 1\\ 0 \\ 0\end{bmatrix} }) = q_{1}$}
%       $1\alpha + 0\beta = 10\gamma$;
%     \item
%       Аналогично, $0\alpha + 1\beta = 10\gamma$;
%     \item
%       За останалите $x \in \Sigma_3$, $\delta(q_{1},x) = q_{err}$.
%     \end{itemize}    
%     \marginpar{$\delta(q_1,\tiny{ \begin{bmatrix} 0\\ 1 \\ 0\end{bmatrix} }) = q_{1}$}
%   \end{itemize}
% \end{hint}

\begin{problem}
  Да разгледаме азбуката:
  \[\Sigma_2 = \left\{\begin{bmatrix} 0\\0\end{bmatrix},\begin{bmatrix} 0\\1\end{bmatrix},\begin{bmatrix} 1\\0\end{bmatrix},\begin{bmatrix} 1\\1\end{bmatrix}\right\}.\]
  % Една дума над азбуката $\Sigma_2$ ни дава два реда от $0$-ли и $1$-ци, които ще разглеждаме като числа в двоична бройна система.
  Да разгледаме езиците:
  \begin{enumerate}[a)]
  \item 
    $L_1 = \left\{\begin{bmatrix} \alpha\\ \beta\end{bmatrix} \in \Sigma^\star_2 \mid \alpha_{(2)} < \beta_{(2)}\right\}$;
  \item
    $L_2 = \left\{\begin{bmatrix} \alpha\\ \beta\end{bmatrix} \in \Sigma^\star_2 \mid 3\alpha_{(2)} = \beta_{(2)}\right\}$;
  \item
    $L_2 = \left\{\begin{bmatrix} \alpha\\ \beta\end{bmatrix} \in \Sigma^\star_2 \mid \alpha = \beta^R\right\}$;
  \end{enumerate}
  Докажете, че  $L_1$ и $L_2$ са автоматни, а $L_3$ не е автоматен.
\end{problem}
% \begin{hint}
%   Ще построим автомат $\A = \FA$ за езика $L^R_1$.
%   За улеснение, в рамките на тази задача ще пишем:
%   \begin{itemize}
%   \item 
%     $\alpha \equiv \beta$, ако $(\alpha^R)_{(2)} = (\beta^R)_{(2)}$,
%   \item
%     $\alpha < \beta$, ако $(\alpha^R)_{(2)} < (\beta^R)_{(2)}$,
%   \item
%     $\alpha > \beta$, ако $(\alpha^R)_{(2)} > (\beta^R)_{(2)}$.
%   \end{itemize}
%   Нека състоянията на автомата са $Q = \{q_{\scriptscriptstyle{=}},q_{\scriptscriptstyle{<}},q_{\scriptscriptstyle{>}}\}$.
%   Искаме да е изпълнено свойствата:
%   \begin{itemize}
%   \item 
%     % За всяко $q \in Q$,
%     $\delta^\star(q_{\scriptscriptstyle{=}}, \scriptsize{\begin{bmatrix} \alpha\\ \beta\end{bmatrix}}) = q_{\scriptscriptstyle{=}}$ точно тогава, когато $\alpha \equiv \beta$;
%   \item 
%     % За всяко $q \in Q$,
%     $\delta^\star(q_{\scriptscriptstyle{=}}, \scriptsize{\begin{bmatrix} \alpha\\ \beta\end{bmatrix}}) = q_{\scriptscriptstyle{<}}$ точно тогава, когато $\alpha < \beta$;
%   \item 
%     % За всяко $q \in Q$,
%     $\delta^\star(q_{\scriptscriptstyle{=}}, \scriptsize{\begin{bmatrix} \alpha\\ \beta\end{bmatrix}}) = q_{\scriptscriptstyle{>}}$ точно тогава, когато $\alpha > \beta$.
%   \end{itemize}
%   Множеството от финални състояния ще бъде $F = \{q_{\scriptscriptstyle{<}}\}$, а началното състояние $s = q_{\scriptscriptstyle{=}}$.
%   За да дефинираме функцията на преходите, трябва да разгледа няколко случая, в зависимост от това какво е отношението между $\alpha$ и $\beta$.
%   \begin{itemize}
%   \item
%     Нека $\alpha \equiv \beta$. Тогава:  
%     \begin{itemize}
%     \item 
%       \marginpar{$\delta(q_{\scriptscriptstyle{=}},\scriptsize{\begin{bmatrix} 0\\0\end{bmatrix}}) = \delta(q_{\scriptscriptstyle{=}},\scriptsize{\begin{bmatrix} 1\\1\end{bmatrix}}) = q_{\scriptscriptstyle{=}}$}
%       $\alpha 0 \equiv \beta 0$, $\alpha 1 \equiv \beta 1$;
%     \item
%       \marginpar{$\delta(q_{\scriptscriptstyle{=}},\scriptsize{\begin{bmatrix} 0\\1\end{bmatrix}}) = q_{\scriptscriptstyle{>}}$}
%       $\alpha 0 < \beta 1$;
%     \item
%       \marginpar{$\delta(q_{\scriptscriptstyle{=}},\scriptsize{\begin{bmatrix} 1\\0\end{bmatrix}}) = q_{\scriptscriptstyle{<}}$}
%       $\alpha 1 < \beta 0$;
%     \end{itemize}
%   \item 
%     Нека $\alpha < \beta$. Тогава:
%     \begin{itemize}
%     \item 
%       \marginpar{$\delta(q_{\scriptscriptstyle{<}},\scriptsize{\begin{bmatrix} 0\\0\end{bmatrix}}) = \delta(q_{\scriptscriptstyle{<}},\scriptsize{\begin{bmatrix} 1\\1\end{bmatrix}}) = \delta(q_{\scriptscriptstyle{<}},\scriptsize{\begin{bmatrix} 0\\1\end{bmatrix}}) = q_{\scriptscriptstyle{<}}$}
%       $\alpha 0 < \beta 0$, $\alpha 1 < \beta 1$, $\alpha 0 < \beta 1$;
%     \item
%       \marginpar{$\delta(q_{\scriptscriptstyle{<}},\scriptsize{\begin{bmatrix} 1\\0\end{bmatrix}}) = q_{\scriptscriptstyle{>}}$}
%       $\alpha 1 > \beta 0$;
%     \end{itemize}    
%   \item
%     Нека $\alpha > \beta$. Тогава:
%     \begin{itemize}
%     \item 
%       \marginpar{$\delta(q_{\scriptscriptstyle{>}},\scriptsize{\begin{bmatrix} 0\\0\end{bmatrix}}) = \delta(q_{\scriptscriptstyle{>}},\scriptsize{\begin{bmatrix} 1\\1\end{bmatrix}}) = \delta(q_{\scriptscriptstyle{>}},\scriptsize{\begin{bmatrix} 1\\0\end{bmatrix}}) = q_{\scriptscriptstyle{>}}$}
%       $\alpha 0 > \beta 0$, $\alpha 1 > \beta 1$, $\alpha 1 > \beta 0$;
%     \item
%       $\alpha 0 < \beta 1$;
%     \end{itemize}
%   \end{itemize}
%   Докажете, че за така дефинирания автомат $\A$, $\L(\A) = L^R_1$.
%   \marginpar{$\delta(q_{\scriptscriptstyle{>}},\scriptsize{\begin{bmatrix} 0\\1\end{bmatrix}}) = q_{\scriptscriptstyle{<}}$}
% \end{hint}

%%% Local Variables: 
%%% mode: latex
%%% TeX-master: "EAI"
%%% End: 


\chapter{Безконтекстни езици и стекови автомати}


\section{Безконтекстни граматики}
\index{граматика!безконтекстна}
% От Сипсер, същото е в слайдовете на Сашка
% Малко е тъпо, че в Пападимитриу дефиницията е различна. Там \Sigma \subseteq V

\begin{dfn}
  \marginpar{На англ. {\bf context-free grammar}}
  \marginpar{Други срещани наименования на български са {\bf контекстно-свободна}, {\bf контекстно-независима}}
%  \marginpar{Според йерархията на Чомски, това са граматики тип }
  Безконтекстна граматика e четворка от вида
  \[G = (V,\Sigma,R,S),\]
  където
  \begin{itemize}
  \item
    \marginpar{Променливите се наричат също нетерминали}
    $V$ е крайно множество от {\em променливи};
  \item
    \marginpar{Буквите се наричат също терминали.}
    $\Sigma$ е крайно множество от {\em букви}, $\Sigma \cap V = \emptyset$;
  \item
    $R \subseteq V\times (V\cup\Sigma)^\star$, крайно множество от {\em правила};
  \item
    $S \in V$ е началната променлива. 
  \end{itemize}
\end{dfn}

При дадена граматика $G$, за правилата на граматиката обикновено ще пишем $A \rightarrow \alpha$ вместо $(A,\alpha) \in R$.
Ще въведем и релация между думи $\alpha,\beta\in (V \cup \Sigma)^\star$, която ще казва, че думата $\beta$
се получаава от $\alpha$ като приложим правло от граматиката.
За две думи $u,v\in (V\cup\Sigma)^\star$ ще пишем $u \rightarrow_G v$, ако съществуват думи $x,y\in (\Sigma\cup V)^\star$, $A\in V$,
правило $A\rightarrow \alpha$ и $u = xAy$, $v = x\alpha y$.
С $\rightarrow^\star_G$ ще означаваме рефлексивното и транзитивно затваряне на релацията $\rightarrow_G$.
% \marginpar{Да се дефинира $\rightarrow^\star_G$}

Езикът породен от граматиката $G$ е множеството от думи
\[\L(G) = \{\alpha\in\Sigma^\star\mid S \rightarrow^\star_G \alpha\}.\]

\begin{dfn}
  \index{език!безконтекстен}
  Казваме, че езикът $L$ е {\bf безконтекстен}, ако съществува безконтекстна граматика $G$,
  за която $L = \L(G)$.
\end{dfn}

\begin{problem}
  Докажете, че езикът $L = \{a^mb^nc^k\mid m+n \geq k\}$ е безконтекстен.
\end{problem}
\begin{proof}
  Да разгледаме граматиката $G$ с правила
  \begin{align*}
    S& \rightarrow aSc\vert aS \vert B\\
    B& \rightarrow bBc\vert bB\vert\varepsilon.
  \end{align*}
  
  Лесно се вижда с индукция по $n$, че за всяко $n$ имаме свойствата:
  \marginpar{\ding{45} Докажете!}
  \begin{itemize}
  \item 
    $S \rightarrow^\star a^nSc^n$,
  \item
    $S \rightarrow^\star a^nS$,
  \item
    $B \rightarrow^\star a^nBc^n$,
  \item
    $B \rightarrow^\star b^nB$.
  \end{itemize}
  Комбинирайки горните свойства, можем да видим, че за всяко $n \geq k$,
  \begin{itemize}
  \item 
    $S \rightarrow^\star a^nSc^k$,
  \item
    $B \rightarrow^\star b^nBc^k$.
  \end{itemize}
  За да докажем, че $L \subseteq L(G)$, 
  да разгледаме една дума $\omega \in L$, т.е. $\omega = a^mb^nc^k$, където $m+n \geq k$.
  Имаме два случая:
  \begin{itemize}
  \item 
    $k \leq m$, т.е. $m = k+l$ и $m+n = k+l+n$.
    Тогава имаме изводите:
    \[S \rightarrow^\star a^kSc^k,\ S \rightarrow^\star a^lS,\ S \rightarrow B,\ B \rightarrow^\star b^nB,\ B \rightarrow \varepsilon.\]
    Обединявайки всичко това, получаваме:
    \[S \rightarrow^\star a^mb^nc^k.\]
  \item
    $k > m$, т.е. $k = m+l$, за някое $l > 0$, и $m+n = k+r = m+l+r$, за някое $r$.
    Тогава имаме изводите:
    \[S \rightarrow^\star a^mSc^m,\ S\rightarrow B,\ B\rightarrow^\star b^lBc^l,\ B\rightarrow b^rB,\ B\rightarrow\varepsilon,\]
    и отново получаваме $S \rightarrow^\star a^mb^nc^k$.
  \end{itemize}
  Така доказахме, че $\omega \in \L(G)$.
  
  Сега ще докажем, че $\L(G) \subseteq L$.
  С индукция по дължината на извода $l$,
  ще докажем, че ако $S \stackrel{l}{\rightarrow}\omega$, то $\omega \in M$, където
  \[M = \{a^nSc^k\mid n\geq k\}\cup\{a^nb^mBc^k\mid n+m\geq k\}\cup\{a^nb^mc^k\mid n+m\geq k\}.\]
  
  Ако $l = 0$, то е ясно, че $S \stackrel{0}{\rightarrow} S$ и $S \in M$.

  Нека $l > 0$ и $S \stackrel{l-1}{\rightarrow} \alpha \rightarrow \omega$.
  От {\bf И.П.} имаме, че $\alpha \in M$. Нека $\omega$ се получава от $\alpha$ с прилагане на правилото $C \rightarrow \gamma$.
  Разглеждаме всички варианти за думата $\alpha \in M$ и за правилото $C\rightarrow \gamma$ в граматиката $G$
  за да докажем, че  $\omega \in M$.
  Удобно е да представим всички случаи в таблица.
  \begin{center}
    \begin{tabular}{| c | c | c |}
      \hline
      $\alpha\in M$ & $C \rightarrow \gamma$ & $\omega \in M?$ \\ \hline
      $a^nSc^k$ & $S \rightarrow aSc$ & $a^{n+1}Sc^{k+1}$ \\ \hline
      $a^nSc^k$ & $S \rightarrow aS$ & $a^{n+1}Sc^{k}$ \\ \hline
      $a^nSc^k$ & $S \rightarrow B$ & $a^{n}Bc^{k}$ \\ \hline
      $a^nb^mBc^k$ & $B \rightarrow bBc$ & $a^nb^{m+1}Bc^{k+1}$\\ \hline
      $a^nb^mBc^k$ & $B \rightarrow bB$ & $a^nb^{m+1}Bc^{k}$\\ \hline
      $a^nb^mBc^k$ & $B \rightarrow \varepsilon$ & $a^nb^{m}c^{k}$\\ \hline
    \end{tabular}
  \end{center}
  Във всички случаи се установява, че $\omega \in M$.
  Сега, за всяка дума $\omega \in L(G)$ следва, че
  \[\omega \in \Sigma^\star \cap M = \{a^mb^nc^k\mid m+n \geq k\}.\]
\end{proof}


\begin{problem}
  \marginpar{
    $S \to aS \mid aSc \mid aB \mid bB$\\
    $B \to bB \mid bBc \mid \varepsilon$
}
  Докажете, че езикът $L = \{a^mb^nc^k\mid m+n \geq k + 1\}$ е безконтекстен.  
\end{problem}

\begin{problem}
  \label{pr:nanb}
  Докажете, че за произволна дума $\omega$ над азбуката $\{a,b\}$ са изпълнени свойствата:
  \begin{enumerate}[a)]
  \item 
    ако $n_a(\omega) = n_b(\omega) + 1$, то съществуват думи $\omega_1$, $\omega_2$, за които
    $\omega = \omega_1 a \omega_2$, $n_a(\omega_1) = n_b(\omega_1)$ и $n_a(\omega_2) = n_b(\omega_2)$.
  \item
    ако $n_b(\omega) = n_a(\omega) + 1$, то съществуват думи $\omega_1$, $\omega_2$, за които
    $\omega = \omega_1 b \omega_2$, $n_a(\omega_1) = n_b(\omega_1)$ и $n_a(\omega_2) = n_b(\omega_2)$.
  \end{enumerate}
\end{problem}
\begin{proof}
  Пълна индукция по дължината на думата $\omega$, за които $n_a(\omega) = n_b(\omega)+1$.
  \begin{itemize}
  \item 
    $\abs{\omega} = 1$. Тогава $\omega_1 = \omega_2 = \varepsilon$ и $\omega = a$.
  \item
    $\abs{\omega} = n+1$. Ще разгледаме два случая, в зависимост от първия символ на $\omega$.
    \begin{itemize}
    \item 
      Случаят $\omega = a\omega'$ е лесен. (Защо?)
    \item
      Интересният случай е $\omega = b\omega'$.    
      Тогава $\omega = b^{i+1}a\omega'$. Да разгледаме думата $\omega''$, която се получава от $\omega$
      като премахнем първото срещане на думата $ba$, т.е. 
      $\omega'' = b^i\omega'$ и $\abs{\omega''} = n-1$.
      Понеже от $\omega$ сме премахнали равен брой $a$-та и $b$-та, $n_a(\omega'') = n_b(\omega'')+1$.
      Според {\bf И.П.} за $\omega''$, можем да запишем думата като $\omega'' = \omega''_1a\omega''_2$
      и $n_a(\omega''_1) = n_b(\omega''_1)$, $n_a(\omega''_2) = n_b(\omega''_2)$.
      Понеже $b^i$ е префикс на $\omega''_1$, за да получим обратно $\omega$, трябва 
      да прибавим премахнатата част $ba$ веднага след $b^i$ в $\omega''_1$.
    \end{itemize}
  \end{itemize}
\end{proof}

\begin{problem}
  За произволна дума $\omega \in \{a,b\}^\star$, 
  докажете, че ако $n_a(\omega) > n_b(\omega)$, то съществуват думи $\omega_1$ и $\omega_2$,
  за които $\omega = \omega_1 a \omega_2$ и $n_a(\omega_1) \geq n_b(\omega_1)$, $n_a(\omega_2) \geq n_b(\omega_2)$.
\end{problem}

\begin{problem}
  Да се докаже, че езикът $L = \{\alpha \in \{a,b\}^\star\mid n_a(\alpha) = n_b(\alpha)\}$ 
  е безконтекстен.
\end{problem}
\begin{proof}
  \marginpar{  Алтернативна граматика за езика $L$ е
  \begin{align*}
    S& \rightarrow aB\vert bA\\
    A& \rightarrow a\vert aS\vert bAA\\
    B& \rightarrow b\vert bS\vert aBB
  \end{align*}}
  Една възможна граматика $G$ е следната: 
  \[S \rightarrow aSbS\vert bSaS \vert\varepsilon.\]
  Например, да разгледаме извода на думата $aabbba$ в тази граматика:
  \begin{align*}
    S & \to aSbS \to aaSbSbS \to aa\varepsilon bSbS \to aab\varepsilon bS \to aabbbSaS\\
    & \to aabbb\varepsilon a S \to aabbba.
  \end{align*}
  
  Като следствие от \Prob{nanb} може лесно да се изведе, че за думи $\omega$, за които $n_a(\omega) = n_b(\omega)$,
  е изпълнено следното:
  \begin{enumerate}[a)]
  \item 
    ако $\omega = a\omega'$, то
    $\omega = a\omega_1b\omega_2$ и $n_a(\omega_1) = n_b(\omega_1)$, $n_a(\omega_2) = n_b(\omega_2)$;
  \item
    ако $\omega = b\omega'$, то
    $\omega = b\omega_1a\omega_2$ и $n_a(\omega_1) = n_b(\omega_1)$, $n_a(\omega_2) = n_b(\omega_2)$.
  \end{enumerate}

  Сега първо ще проверим, че $L \subseteq L(G)$.
  За целта ще докажем с {\em пълна индукция} по дължината на думата $\omega$, че за всяка дума $\omega$ със свойството $n_a(\omega) = n_b(\omega)$ е изпълнено
  $S \rightarrow^\star \omega$.
  \begin{itemize}
  \item 
    Нека $\abs{\omega} = 0$. Тогава $S \rightarrow \varepsilon$.
  \item
    Нека $\abs{\omega} = k+1$. Имаме два случая.
    \begin{itemize}
    \item 
      $\omega = a\omega^\prime$, т.е. от свойство а), $\omega = a\omega_1b\omega_2$ и $n_a(\omega_1) = n_b(\omega_1)$, $n_a(\omega_2) = n_b(\omega_2)$.
      Тогава $\abs{\omega_1} \leq k$ и по И.П. $S \rightarrow^\star \omega_1$.
      Аналогично, $S \rightarrow^\star \omega_2$.
      Понеже имаме правило $S \rightarrow aSbS$, заключаваме че $S \rightarrow^\star a\omega_1b\omega_2$.
    \item
      $\omega = b\omega^\prime$, т.е. свойство б), $\omega = b\omega_1a\omega_2$ и $n_a(\omega_1) = n_b(\omega_1)$, $n_a(\omega_2) = n_b(\omega_2)$.
      Този случай се разглежда аналогично.
    \end{itemize}
  \end{itemize}
  
  Преминаваме към доказателството на другата посока, т.е. $L(G) \subseteq L$.
  Тук с индукция по дължината на извода $l$ ще докажем, че
  $S \stackrel{l}{\rightarrow} \omega$, то $\omega \in M$,
  където
  \[M = \{\omega \in \{a,b,S\}^\star \mid n_a(\omega) = n_b(\omega)\}.\]
  За $l = 0$  е ясно, че $S \stackrel{0}{\rightarrow^\star} S$.
  За $l = k+1$, то $S \stackrel{k}{\rightarrow^\star} \alpha \rightarrow \omega$.
  От {\bf И.П.} имаме, че $\alpha \in M$.
  Нека $\omega$ се получава от $\alpha$ с прилагане на правилото $C \rightarrow \gamma$.
  Разглеждаме всички варианти за думата $\alpha \in M$ и за правилото $C\rightarrow \gamma$ в граматиката $G$
  за да докажем, че  $\omega \in M$.
  Удобно е да представим всички случаи в таблица.
  \begin{center}
    \begin{tabular}{| c | c | c |}
      \hline
      $\alpha$ & $C \rightarrow \gamma$ & $\omega$ \\ \hline
      $\in M$ & $S \rightarrow aSbS$ & $\in M$ \\ \hline
      $\in M$ & $S \rightarrow bSaS$ & $\in M$ \\ \hline
      $\in M$ & $S \rightarrow \varepsilon$ & $\in M$ \\ \hline
    \end{tabular}
  \end{center}
  Във всички случаи лесно се установява, че $\omega \in M$.
  Така за всяка дума $\omega \in L(G)$ следва, че
  \[\omega \in \Sigma^\star \cap M = L.\]
\end{proof}

\begin{problem}
  Докажете, че следните езици са безконтекстни.
  \begin{enumerate}[a)]
  \item
    \marginpar{$S \rightarrow aSa\ \vert\ bSb\ \vert\ \varepsilon$}
    $L = \{ww^R \mid w \in \{a,b\}^\star\}$;
  \item
    \marginpar{$S \rightarrow aSa\ \vert\ bSb\ \vert\ a\vert\ b\ \vert\ \varepsilon$}
    $L = \{w \in \{a,b\}^\star \mid w = w^R\}$;
  \item
    $L = \{a^nb^{2m}c^{n} \mid m,n \in \Nat\}$;
  \item
    $L = \{a^nb^{m}c^{m}d^n \mid m,n \in \Nat\}$;
  \item
    $L = \{a^nb^{2k} \mid n,k \in \Nat\ \&\ n \neq k\}$;
  \item
    \marginpar{$S \rightarrow aSb | aS | a$}
    $L = \{a^nb^k \mid n > k\}$;
  \item
    $L = \{a^nb^k \mid n \geq 2k\}$;
  \item
    \marginpar{$S \rightarrow aSc | B,\ B \rightarrow bBc | \varepsilon$}
    $L = \{a^nb^mc^{n+m}\mid n,m \in \Nat\}$;
  \item
    \marginpar{$S \rightarrow aSc | aS | B$, $B\rightarrow bBc | bB | \varepsilon$}
    $L = \{a^nb^kc^m \mid n + k \geq m\}$;
  \item
    \marginpar{$S \rightarrow aSc | aS | aB | bB$,\\$B\rightarrow bBc | bB | \varepsilon$}
    $L = \{a^nb^kc^m \mid n + k \geq m+1\}$;
  \item
    $L = \{a^nb^kc^m \mid n + k \geq m+2\}$;
  \item
    \marginpar{$S \rightarrow aSc | aS | B | Bc$,\\$B\rightarrow bBc | bB | \varepsilon$}
    $L = \{a^nb^kc^m \mid n + k + 1 \geq m\}$;
  \item
    $L = \{a^nb^kc^m \mid n + k + 2 \geq m\}$;
  \item
    $L = \{a^nb^kc^m \mid n + k \leq m\}$;
  \item
    $L = \{a^nb^kc^m \mid n + k \leq m+1\}$;
  \item
    \marginpar{Обединение на три езика}
    $L = \{a^nb^mc^k \mid n, m, k \text{ не са страни на триъгълник}\}$.
  \item
    $L = \{a,b\}^\star \setminus \{a^{2n}b^n \mid n\in\Nat\}$;
  \item
    \marginpar{$S\to EaE$, $E \to aEbE | bEaE | \varepsilon$}
    $L = \{\alpha \in \{a,b\}^\star\mid n_a(\alpha) = n_b(\alpha) + 1\}$;
  \item
    \marginpar{$S\to E | SaS$, $E \to aEbE | bEaE | \varepsilon$}
    $L = \{\alpha \in \{a,b\}^\star\mid n_a(\alpha) \geq n_b(\alpha)\}$;
  \item
    $L = \{\alpha \in \{a,b\}^\star\mid n_a(\alpha) > n_b(\alpha)\}$;
  \item
    $L = \{\omega_1 a \omega_2 b \mid \omega_1,\omega_2 \in \{a,b\}^\star\ \&\ \abs{\omega_1} = \abs{\omega_2}\}$;
  \item
    $L = \{\alpha c \beta \mid \alpha,\beta \in \{a,b\}^\star\ \&\ \alpha^R\mbox{ е поддума на }\beta \}$.
  \end{enumerate}
\end{problem}

\begin{problem}
  Да разгледаме граматиката $G = \CFG$, където  $V = \{S,A,B\}$, $\Sigma = \{a,b\}$, а правилата $R$ са
  \[S \to AA | B, A \to B | bb, B \to aa | aB.\]
  Да се намери езика на тази граматика и да се докаже, че граматиката разпознава точно този език.
\end{problem}



\section{Езици, които не са безконтекстни}

\begin{lemma}[за покачването (безконтекстни езици)]
  \index{лема за покачването!безконтекстни езици}
  \label{lem:pumping-context} 
  \marginpar{(стр. 123 от \cite{sipser1}; стр. 125 от \cite{hopcroft1})}
  За всеки безконтекстен език $L$ съществува $p>0$, такова
  че ако $\alpha\in L, \abs{\alpha} \geq p$, то съществува разбиване на думата на пет части, $\alpha=xyuvw$,
  за което е изпълнено:
  \begin{enumerate}[1)]
  \item
    $\abs{yv}\geq 1$,
  \item
    $\abs{yuv}\leq p$, и
  \item
    $(\forall i\geq 0)[xy^iuv^iw\in L]$.
\end{enumerate}
\end{lemma}
\begin{proof}
  Нека $G$ е граматиката за езика $L$.
  \marginpar{За простота, можем да си мислим, че $G$ е в НФЧ. Тогава $b=2$.}
  Нека \[b = \max\{\abs{\beta} \mid A\rightarrow_G \beta\}.\]
  Можем да приемем, че $b \geq 2$.
  \marginpar{Възлите във вътрешността на дървото са променливи, а листата са букви или $\varepsilon$}
  Това означава, че във всяко дърво на извод, всеки възел има
  не повече от $b$ наследника.
  Нека $p = b^{\abs{V}}+1$. Ще покажем, че $p$ е константа на покачването за граматиката $G$.
  Това означава, че всяка дума с дължина поне $p$ в езика $L$ има дърво на извод с височина
  поне $\abs{V} + 1$.
  
  Нека $\abs{\alpha} \geq p$ и $T$ е дърво на извода за думата $\alpha$.
  Понеже думата $\alpha$ може да има много дървета на извод, нека $T$ да бъде {\bf с минимален брой възли}.
  От направените по-горе разсъждения е ясно, че височината на $T$ е поне $\abs{V} + 1$,
  Следователно, по най-дългия път $\pi$ в $T$ имаме поне $\abs{V}+2$ възела, от които
  поне $\abs{V}+1$ са променливи, защото само листата могат да не са променливи.
  Да разгледаме последните $\abs{V}+1$ променливи по пътя $\pi$.
  От принципа на Дирихле следва, че измежду тези $\abs{V}+1$ променливи има поне една повтаряща се.
  Нека $R$ да бъде една такава променлива.
  Последните две повтаряния на $R$ разделят думата $\alpha$ на пет части.
  Нека $\alpha = xyuvw$.
  \begin{enumerate}[1)]
  \item
    $\abs{yv}\geq 1$,
    защото ако допуснем, че $\abs{yv} = 0$,
    то ще достигнем до противоречие с минималността на $T$.
  \item
    $\abs{yuv} \leq p$, защото сме избрали най-долното $R$.
  \item
    $xy^iuv^iw \in L$, защото можем да заменим поддървото 
    с корен последното $R$ за поддървото с корен предпоследното $R$.
    В случая $i = 0$, правим обратното.
  \end{enumerate}
\end{proof}

\begin{cor}[Контрапозиция на лемата за покачването]
  \label{cor:pumping-context-free}
  \marginpar{Ако $L$ е краен език, то е ясно, че $L$ е безконтекстен.}
  Нека $L$ е произволен {\bf безкраен} език. Нека също така е изпълнено, че за всяко естествено число $p \geq 1$ можем да намерим дума $\alpha \in L$, $\abs{\alpha}\geq p$, такава че за всяко разбиване на думата на пет части, $\alpha = xyuvw$,
  със свойствата $\abs{yv} \geq 1$ и $\abs{yuv} \leq p$, е изпълнено, че $(\exists i)[xy^iuv^iw \not\in L]$.
  \marginpar{\writedown Докажете! Аналогично е на \Cor{pumping-reg}}
  Тогава $L$ {\bf не} е безконтекстен език.
\end{cor}

\begin{cor}
  \marginpar{\writedown Докажете!}
  Нека $G$ е безконтекстна граматика и $p$ е константата на покачването за $G$, $L = \L(G)$.
  Тогава $\abs{L} = \infty$ точно тогава, когато съществува $\alpha \in L$, за която $p \leq \abs{\alpha} < 2p$.
\end{cor}
% \begin{proof}
%   Ако съществува дума $\alpha \in L$, за която $\abs{\alpha} \geq p$, то от \Lem{pumping-context} следва,
%   че $\abs{L} = \infty$, защото $\alpha = xyuvw$ и $xy^iuv^iw \in L$, за всяко $i\in\Nat$.

%   За другата посока, нека сега $\abs{L} = \infty$.
%   Да изберем най-късата дума $\alpha \in L$, за която $\abs{\alpha} \geq p$.
%   Ще докажем, че $p \leq \abs{\alpha} < 2p$. За целта да допуснем, че $\abs{\alpha} \geq 2p$.
%   Тогава от \Lem{pumping-context} следва, че $\alpha = xyuvw$, $\abs{yv} \geq 1$, $\abs{yuv} \leq p$, $xy^0uv^0w = xuw \in L$.
%   Ако $\abs{xuw} < p$, то $\abs{yv} > p$, защото $\abs{yv} + \abs{xuw} = \abs{\alpha} \geq 2p$, и следователно $\abs{yuv} > p$, което е противоречие.
%   Следва, че $\abs{\alpha} > \abs{xuw} \geq p$.
%   Получихме, че думата $xuw\in L$ и $\abs{xuw} \geq p$. Това е противоречие с минималността на $\alpha$.
% \end{proof}

% \begin{framed}
%   \Lem{pumping-context} е полезна, когато искаме да докажем, че даден език $L$ {\bf не} е безконтекстен.
%   За целта, доказваме отрицанието на свойствата от \Lem{pumping-context} за $L$, т.е.
%   за всяка константа $p$, ние намираме дума $\alpha \in L$, $\abs{\alpha}\geq p$, такава че за всяко разбиване на думата на пет части, $\alpha = xyuvw$,
%   със свойствата $\abs{yv} \geq 1$ и $\abs{yuv} \leq p$, е изпълнено, че $(\exists i)[xy^iuv^iw \not\in L]$.
% \end{framed}


\begin{example}
  \label{example:anbncn}
  Езикът $L = \{a^nb^nc^n\ \mid\ n\in\Nat\}$ не е безконтекстен.
\end{example}
\begin{proof}
  За да докажем, че $L$ не е безконтекстен, прилагаме \Cor{pumping-context-free} по следния начин:
  \begin{itemize}
  \item 
    Разглеждаме произволна константа $p \geq 1$.
  \item
    Избираме дума $\alpha \in L$, $\abs{\alpha} \geq p$.
    В случая, нека $\alpha = a^pb^pc^p$.
  \item
    Разглеждаме произволно разбиване $xyuvw = \alpha$, за което $\abs{xyv} \leq p$ и $1 \leq \abs{yv}$.
  \item
    Трябва да изберем $i$, за което $xy^iuv^iw \not\in L$.
    Знаем, че поне едно от $y$ и $v$ не е празната дума.
    Имаме няколко случая за $y$ и $v$.
    \begin{itemize}
    \item
      $y$ и $v$ са думи съставени от една буква.
      В този случай получаваме, че $xy^2uv^2w$ има различен брой букви $a$, $b$ и $c$.
    \item
      $y$ или $v$ е съставена от две букви.
      Тогава е възможно да се окаже, че $xy^2uv^2w$ да има равен брой $a$, $b$ и $c$,
      но тогава редът на буквите е нарушен.
    \item
      понеже $\abs{yuv} \leq p$, то не е възможно в $y$ или $v$ да се срещат и трите букви.
    \end{itemize}  
    Оказа се, че във всички възможни случаи за $y$ и $v$, 
    $xy^2uv^2w \not\in L$.
  \end{itemize}
  Така от \Cor{pumping-context-free} следва, че езикът $L$ не е безконтекстен.
\end{proof}

% \begin{problem}
%   Да се даде пример за език $L$, който {\bf не} е безконтекстен, но удовлетворява
%   лемата за разрастването.
% \end{problem}

\begin{example}
  Приложете лемата за покачването за да докажете, че
  езикът $L$ не е безконтекстен, където:
  \begin{enumerate}[a)]
  \item
    $L = \{a^ib^jc^k\ \mid\ 0 \leq i \leq j \leq k\}$;
  \item
    $L = \{\beta\beta\mid \beta\in \{a,b\}^\star\}$;
  \item
    $L = \{a^{n^2}\mid n\in\Nat\}$.
  \end{enumerate}
\end{example}
\begin{proof}
  \begin{enumerate}[a)]
  \item
    Да фиксираме думата $\alpha = a^pb^pc^p$ и да разгледаме
    едно произволно нейно разбиване, $\alpha = xyuvw$, за което
    $\abs{yuv} \leq p$ и $1 \leq \abs{yv}$.
    Знаем, че поне една от $y$ и $v$ не е празната дума.
    \begin{itemize}
    \item
      $y$ и $v$ са съставени от една буква.
      Имаме три случая.
      \begin{enumerate}[i)]
      \item
        $a$ не се среща в $y$ и $v$.
        Тогава $xy^0vu^0w$ съдържа повече $a$ от $b$ или $c$.
      \item
        $b$ не се среща в $y$ и $v$.
        Ако $a$ се среща в $y$ или $v$, тогава $xy^2uv^2w$ съдържа повече $a$ от $b$
        Ако $c$ се среща в $y$ или $v$, тогава $xy^0uv^0w$ съдържа по-малко $c$ от $b$.
      \item
        $c$ не се среща в $y$ и $v$.
        Тогава $xy^2uv^2w$ съдържа повече $a$ или $b$ от $c$.
      \end{enumerate}      
    \item
      $y$ или $v$ е съставена от две букви.
      Тук разглеждаме $xy^2uv^2w$ и съобразяваме, че редът на буквите е нарушен.
    \end{itemize}
  \item
    \marginpar{Защо $\alpha = a^pba^pb$ не е добър кандидат?}
    Разгледайте $\alpha = a^pb^pa^pb^p$, т.е. $\beta = a^pb^p$ и $\alpha = \beta\beta$.
    Нека $xyuvw = \alpha$ е произволно разбиване на $\alpha$, за което е изпълнено, че
    $\abs{yuv} \leq p$ и $1\leq \abs{yv}$.
    \begin{itemize}
    \item
      Ако $yuv$ е в първата част на думата, то 
      $xy^0uv^0w = a^ib^ja^pb^p \not\in L$.
      Аналогично ако $yuv$ е във втората част на думата.
    \item
      Ако $yuv$ е в двете части на думата, то 
      $xy^0uv^0w = a^pb^ia^jb^p \not\in L$.
    \end{itemize}    
  \item
    Решава се аналогично както за регулярни езици.
  \end{enumerate}
\end{proof}


\begin{thm}
  Безконтекстните езици {\bf не} са затворени относно сечение и допълнение.
\end{thm}
\begin{proof}
  Да разгледаме езика
  \[L_0 = \{a^nb^nc^n\mid n\in\Nat\},\] за който вече знаем от Пример \ref{example:anbncn}, че не е безконтекстен.
  Да вземем също така и безконтекстните езици 
  \marginpar{\writedown Защо са безконтекстни?}
  \[L_1 = \{a^nb^nc^m\mid n,m\in\Nat\},\ L_2 = \{a^mb^nc^n\mid n,m\in\Nat\},\]
  \begin{itemize}
  \item 
    Понеже $L_0 = L_1\cap L_2$, то заключаваме, че безконтекстните езици не са затворени 
    относно операцията сечение.
  \item
    \marginpar{Озн. $\ov{L} = \Sigma^\star \setminus L$}
    Да допуснем, че безконтекстните езици са затворени относно операцията допълнение.
    Тогава  $\ov{L}_1$ и $\ov{L}_2$ са безконтекстни.
    Знаем, че безконтекстните езици са затворени относно обединение. 
    Следователно, езикът $L_3 = \ov{L}_1 \cup \ov{L}_2$ също е безконтекстен.
    Ние допуснахме, че безконтекстните са затворени относно допълнение, следователно $\ov{L}_3$
    също е безконтекстен.
    Но тогава получаваме, че езикът
    \[L_0 = L_1 \cap L_2 = \ov{\ov{L}_1 \cup \ov{L}_2} = \ov{L}_3\]
    е безконтекстен, което е противоречие.
  \end{itemize}
\end{proof}


\section{Алгоритми}

\subsection{Опростяване на безконтекстни граматики}

\subsubsection*{Премахване на безполезните променливи}

Нека е дадена безконтекстната граматика $G = \CFG$.
\marginpar{\cite{hopcroft1} стр. 88}
Една променлива $A$ се нарича {\bf полезна}, ако съществува извод от следния вид:
\[S \to^\star \alpha A \beta \to^\star \gamma,\]
където $\gamma \in \Sigma^\star$, а $\alpha,\beta \in (V \cup \Sigma)^\star$.
Това означава, че една променлива е полезна, ако участва в извода на някоя дума в езика на граматиката.
Една променлива се нарича {\bf безполезна}, ако не е полезна.
Целта ни е да получим еквивалентна граматика $G'$ без безполезни променливи.
Ще решим задачата като разгледаме две леми.

\begin{lemma}
  \label{lem:useless1}
  Нека е дадена безконтекстната граматика $G = \CFG$ и $\L(G) \neq \emptyset$.
  Съществува алгоритъм, който намира граматика $G' = \pair{V',\Sigma,S,R'}$, за която 
  $\L(G) = \L(G')$, и за всяка променлива $A' \in V'$, съществува дума $\alpha \in \Sigma^\star$,
  за която $A' \to^\star \alpha$.
\end{lemma}
\begin{proof}
  Да разгледаме следната проста итеративна процедура.
  \begin{algorithm}[H]
    \caption{Намираме $V' = \{A \in V\mid (\exists \alpha \in \Sigma^\star)[A \to^\star \alpha]\}$}
    \label{alg:useless}
    \begin{algorithmic}[1]
      \State $V' := \emptyset$
      \State $V'' := \{A \in V \mid (\exists \alpha \in \Sigma^\star)[A \to \alpha]\}$
      \While{$V' \neq V''$}
      \State $V' := V''$
      \State $V'' := V' \cup \{A \in V \mid (\exists \alpha \in (\Sigma \cup V')^\star)[A \to \alpha]\}$
      \EndWhile
      \State \Return $V'$
    \end{algorithmic}
  \end{algorithm}
  Трябва да докажем, че във $V'$ са точно полезните променливи за $G$.
  Очевидно е, че ако $A \in V'$, то $A$ е полезна променлива.
  \marginpar{\writedown Докажете!}
  За другата посока, с индукция по дължината на извода се доказва, че ако $A \to^\star_G \omega$,
  то $A \in V'$.
  
  Правилата на $G'$ са всички правила на $G$, в които участват променливи от $V'$ и букви от $\Sigma$.
\end{proof}

\begin{lemma}
  \label{lem:useless2}
  Съществува алгоритъм, който по дадена безконтекстна граматика $G = \CFG$, намира $G' = \pair{V',\Sigma',S,R'}$, $\L(G') = \L(G)$,
  със свойството, че за всяко $x \in V' \cup \Sigma'$ съществуват $\alpha, \beta \in (V'\cup\Sigma')^\star$,
  за които $S \to^\star \alpha x \beta$,
  т.е. всяка променлива или буква в $G'$ е достижима от началната променлива $S$.
\end{lemma}
\begin{proof}
  Намираме $V'$ и $\Sigma'$ итеративно, като в началото $V' = \{S\}$, $\Sigma' = \emptyset$.
  Ако $A \in V'$ и имаме правила $A \to \alpha_0 | \alpha_1 | \dots | \alpha_n$ в $G$,
  то за всяко $i = 0,\dots,n$ добавяме всички променливи на $\alpha_i$ към $V'$ и всички нетерминали на $\alpha_i$ към $\Sigma'$.
\end{proof}

\begin{thm}
  Всеки непразен безконтекстен език $L$ се поражда от безконтекстна граматика $G$
  без безполезни правила.
\end{thm}
\begin{proof}
  \marginpar{Важна ли е последователността на прилагане?}
  Нека е дадена безконтекстна граматика $G$ пораждаща $L$.
  Прилагаме върху $G$ първо процедурата от \Lem{useless1} и след това върху резултата прилагаме процедурата от \Lem{useless2}.
\end{proof}

\subsubsection*{Премахване на $\varepsilon$-правила}
\index{$\varepsilon$-правила}
За да премахнем правилата от вида $A \to \varepsilon$, следваме процедурата:
\marginpar{Броят на правилата може да се увеличи експоненциално, защото в най-лошия случай извеждаме всички подмножества на дадено множество от променливи}
\begin{enumerate}[1)]
\item 
  Намираме множеството $E = \{A \in V \mid A \to^\star \varepsilon\}$ по следния начин.
  Първо, $E := \{A \in V \mid A \to \varepsilon\}$.
  След това, за всяко правило от вида $B \to X_1\cdots X_k$, 
  ако всяко $X_i \in E$, то добавяме $B$ към $E$.
\item
  Строим множеството от правила $R'$, в което няма правила $\varepsilon$-правила по следния начин.
  За всяко правило $A \to X_1\cdots X_k$ в $R$,
  добавяме към $R'$ всички правила от вида $A \to \alpha_1\cdots\alpha_k$, където:
  \begin{itemize}[-]
  \item 
    ако $X_i \not\in E$, то $\alpha_i = X_i$;
  \item
    ако $X_i \in E$, то $\alpha_i = X_i$ или $\alpha_i = \varepsilon$;
  \item
    не всички $\alpha_i$-та са $\varepsilon$.
  \end{itemize}
\end{enumerate}

\begin{example}
  Нека е дадена граматиката $G$ с правила
  \[S\rightarrow D,D\rightarrow AD|b,A\rightarrow AB|BC|a, B\rightarrow AA|EC,C\rightarrow \varepsilon|CA|a, E\rightarrow \varepsilon|aEb.\]
  Тогава $E = \{X \in V \mid X \rightarrow^\star_G \varepsilon\} = \{A,B,C,E\}$.
  Това означава, че $\varepsilon \not\in \L(G)$.
  Граматиката $G'$ без $\varepsilon$-правила, за която $\L(G') = \L(G)$ има следните правила
  $S \to D, D\to AD|D|b, A \to A|B|C|AB|BC|a,B\to A|E|C|AA|EC, C \to C|A|CA|a, E \to aEb|ab$.
\end{example}

\subsubsection*{Премахване на преименуващи правила}
\index{преименуващи правила}
Преименуващите правила са от вида $A \to B$.
Нека е дадена граматика $G = \CFG$, в която има преименуващи правила.
Ще построим еквивалентна граматика $G'$ без преименуващи правила.
В началото нека в $R'$ да добавим всички правила от $R$, които не са преименуващи.
След това, за всякa променлива $A$, за която $A \to^\star_G B$,
ако $B \to \alpha$ е правило в $R$, което не е преименуващо,
то добавяме към $R'$ правилото $A \to \alpha$.

\begin{example}
  Нека е дадена граматиката $G$ с правила  
  \[A\rightarrow B|S,B\rightarrow C|BC,C\rightarrow AB|a|b,S\rightarrow B|CC|b.\]
  Първо добавяме към $R'$ правилата $B \to BC, C \to AB|a|b, S \to CC|b$.
  \begin{itemize}
  \item 
    Лесно се съобразява, че $A \to^\star_G B,S,C$.
    Добавяме правилата $A \to BC|AB|a|b|CC$.
  \item
    Имаме $B \to^\star_G C$.
    Добавяме правилата $B \to AB|a|b$.
  \item
    Имаме $S \to^\star_G B,C$.
    Добавяме правилата $S \to BC|AB|a|b$.
  \end{itemize}
  Накрая получаваме, че граматиката $G'$ има правила
  $A \to BC|AB|a|b|CC, B \to AB|a|b|BC, C \to AB|a|b, S \to BC|AB|CC|a|b$.
\end{example}

\subsection{Нормална Форма на Чомски}

\begin{dfn}
%[стр. 99 от \cite{sipser}]
\index{Нормална форма на Чомски}
Една безконтекстна граматика е в {\em нормална форма на Чомски}, ако
всяко правило е от вида
\[A \rightarrow BC\mbox{ и }A \rightarrow a,\]
като $B, C$ {\em не могат} да бъдат променливата за начало $S$.
Освен това, позволяваме правилото $S\to\varepsilon$.
\footnote{На стр. 151 в \cite{papadimitriou} дефиницията е малко по-различна.
Там дефинират $G$ да бъде в нормална форма на Чомски ако $R \subseteq V\times(V\cup\Sigma)^2$.
В този случай губим езиците $\{\varepsilon\}$ и $\{a\}$, за $a\in\Sigma$.}
\end{dfn}

\begin{thm}
  Всеки безконтекстен език $L$ е генериран от контекстно-свободна
  граматика в нормална форма на Чомски.
\end{thm}
\begin{proof}
%  \marginpar{Броят на правилата може да се увеличи експоненциално.}
  Нека имаме контекстно-свободна граматика $G$, за която $L = L(G)$.
  Ще построим контекстно-свободна граматика $G^\prime$ в нормална форма на Чомски, $L = L(G^\prime)$.
  % [стр. 99 от \cite{sipser}]
  Следваме следната процедура:
  \begin{itemize}
  \item
    Добавяме нов начален символ $S_0$ и правило $S_0 \to S$.
  \item
    \marginpar{Време $O(n)$}
    Съкращаваме дължината на правилата.
    Заменяме правилата от вида $A\to u_1\dots u_n$, $n\geq 3$, $u_i \in V\cup\Sigma$, с
    правилата \[A\to u_1A_1,\ A_1\to u_2A_2,\ \dots,\ A_{n-2} \to u_{n-1}u_n.\]
    където $A_i$ са нови променливи.
  \item
    \marginpar{Време $O(n^2)$}
    За всяка променлива $A \neq S_0$ премахваме правилата от вида $A\to\varepsilon$.
    Това правим по следния начин.
    
    Ако имаме правило от вида $R \to Au$ или $R\to u A$, $u \in V \cup \Sigma$,
    то добавяме правилото $R\to u$.
    %Правим това за всяко срещане на променливата $A$ в дясната страна на правило.
    Например, 
    \begin{itemize}
    \item 
      ако имаме правило $R\to aA$, то добавяме правилото $R \to a$;
    \item
      ако имаме правило $R\to AA$, то добавяме правилото $R \to A$.
    \end{itemize}
    Ако имаме правило от вида $R\to A$, то добавяме правилото $R\to\varepsilon$
    само ако променливата $R$ още не е преминала през процедурата за премахване на $\varepsilon$.
  \item
    \marginpar{Време $O(n^2)$}
    \marginpar{Памет $O(n^2)$}
    Премахваме преименуващите правила, т.е. правила от вида $A\to B$.
    Заменяме всяко правило от вида $B \to \beta$ с $A\to \beta$,
    освен ако $A \to \beta$ е вече премахнато преименуващо правило.
  \item
    \marginpar{Време $O(n)$}
    За правила от вида $A\to u_1 u_2$, където $u_1, u_2 \in V \cup \Sigma$, 
    заменяме всяка буква $u_i$ с новата променлива $U_i$
    и добавяме правилото $U_i\to u_i$.
    Например, правилото $A \to aB$ се заменя с правилото $A \to XB$ и добавяме правилото $X \to a$,
    където $X$ е нова променлива.
  \end{itemize}
\end{proof}

\begin{thm}
  При дадена безконтекстна граматика $G$ с дължина $n$, можем да намерим еквивалентна
  на нея граматика $G'$ в нормална форма на Чомски за време $O(n^2)$,
  като получената граматика е с дължина $O(n^2)$.
\end{thm}


% \begin{problem}
%   Нека е дадена граматиката  $G = \pair{\{S,A,B,C,D,E\}, \{a,b\},S, R}$.
%   \begin{enumerate}[a)]
%   \item
%     Намерете множеството $\{X \in V \mid X \rightarrow^\star_G \varepsilon\}$.
%   \item
%     Вярно ли е, че $\varepsilon \in L(G)$?
%   \item
%     Постройте граматика $G_1$ без $\varepsilon$-правила, за която $L(G_1)=L(G)\setminus\{\varepsilon\}$.
%   \end{enumerate}
%   Множеството от правила $R$ на граматиката $G$ е зададено като:
%   \begin{enumerate}[a)]
%   \item
%     $R = \{S\rightarrow D,D\rightarrow AD|b,A\rightarrow ACB|BC|a, B\rightarrow ABCA|CEC,C\rightarrow \varepsilon|CA|a, E\rightarrow \varepsilon|aEb\}$;
%   \item
%     $R = \{S \rightarrow aD, D\rightarrow \varepsilon|ABBA|ADD,A\rightarrow DEB|a,B\rightarrow DDD|DC|b,C\rightarrow CCE|a, E\rightarrow \varepsilon|bEa\}$;
%   \item
%     $R = \{ S\rightarrow D,D\rightarrow AD|b,A\rightarrow AB|BC|a, B\rightarrow AB|CC, C\rightarrow \varepsilon|CA|a, E\rightarrow a|EB\}$;
%   \item
%     $R = \{ S \rightarrow AD|a, D\rightarrow \varepsilon|BB|AD,A\rightarrow DB|a,B\rightarrow DD|DC|b,C\rightarrow CE|a, E\rightarrow AB|b|EA\}$;
%   \item
%     $R =\{S\rightarrow AS|SB|SS,B\rightarrow CA|b, C\rightarrow AA|a|BA,A\rightarrow \varepsilon|BS\}$;
%   % \item
%   %   $R = \{S\rightarrow AB|AC,B\rightarrow \varepsilon |BC|b,A\rightarrow BB|CC|a,C\rightarrow CS|a\}$;
%   % \item
%   %   $R = \{S\rightarrow AS|SB|SS,B\rightarrow AC|b, C\rightarrow A|a|AB,A\rightarrow \varepsilon|BS\}$;
%   \item
%     $R = \{S\rightarrow BA|CA,B\rightarrow \varepsilon |BC|b,A\rightarrow BB|CC|a, C\rightarrow CS|a\}$;
%   \item
%     $R = \{S\rightarrow AS|b,A\rightarrow AC|BC|a, B\rightarrow BC|CC,C\rightarrow \varepsilon|CA|a\}$;
%   \item
%     $R = \{S\rightarrow \varepsilon|BA|AS,A\rightarrow SB|a,B\rightarrow SS|SC|b,
%     C\rightarrow CC|a\}$; 
%   \end{enumerate}
% \end{problem}

\begin{problem}
  Нека е дадена граматиката  $G = \pair{\{S,A,B,C\}, \{a,b\}, S, R}$.
  Използвайте обща конструкция, за да премахнете ,,дългите'' правила 
  (т.е. правила с дължина поне 2, които не са в н.ф. на Чомски) от $ G$ като при това получите 
  безконтестна граматика $G_1$ с език $L(G)=L(G_1)$, където:
  \begin{enumerate}[a)]
  \item
    $R = \{S \rightarrow \varepsilon|ab|aAba, A\rightarrow aBCb, B\rightarrow bbb, C\rightarrow aC\vert aCaC\}\rangle$;
  \item
    $R = \{S \rightarrow \varepsilon|ab|baAb, A\rightarrow BaBb,B\rightarrow b,C\rightarrow AbA\vert aCCa\}$;
  \item
    $R = \{A\rightarrow BSB|a,B\rightarrow ba|BC,C\rightarrow BaSA|a|b,S\rightarrow CC|b\}$;
  \item
    $R = \{A\rightarrow BAS,B\rightarrow CB,C\rightarrow ab|ABbS,S\rightarrow CC|b\}$;
  \end{enumerate}
\end{problem}


% \begin{problem}
%   Намерете безконтекстна граматика в нормална форма на Чомски за езиците от задача 6.
% \end{problem}


\subsection{Проблемът за принадлежност}

\begin{thm}
  Съществува {\em полиномиален} алгоритъм , който проверява дали дадена дума принадлежни на граматиката $G$.
  \marginpar{За дума $\alpha$, алгоритъмът работи за време $O(\abs{\alpha}^3)$}
\end{thm}
% \begin{proof}[стр. 154 от \cite{papadimitriou}]
Можем да приемем, че $G = \CFG$ е граматика в нормална форма на Чомски.
Нека $\alpha = a_1a_2\dots a_n$ е дума, за която искаме да проверим дали $\alpha \in L(G)$.
\marginpar{Това е алгоритъм на Cocke, Younger и Kasami (CYK), който е пример за динамично програмиране (стр. 195 от \cite{kozen})}
\begin{algorithm}[H]
  \caption{Проверка за $\alpha \in L(G)$}
  \label{alg:belongs-to-grammar}
  \begin{algorithmic}[1]
    \State $n := \abs{\alpha}$ \Comment{Вход дума $\alpha = a_1\cdots a_n$}
    \ForAll{$i\in [1,n]$}
    \State $V[i,i] = \{A \in V \mid A\rightarrow a_i\}$
    \EndFor
    \ForAll{$i,j \in [1,n]\ \&\ i \neq j$}
    \State $V[i,j] = \emptyset$
    \EndFor      
    \ForAll{$s \in [1, n)$} \Comment{Дължина на интервала}
    \ForAll{$i \in [1, n-s]$}\Comment{Начало на интервала}
    \ForAll{$k \in [i, i + s)$}\Comment{Разделяне на интервала}
    \If{$\exists A\to BC \in R\ \&\ B \in V[i,k]\ \&\ C\in V[k+1,i+s]$}
    \State $V[i,i+s] := V[i,i+s] \cup \{A\}$
    \EndIf
    \EndFor
    \EndFor
    \EndFor
    \If{$S \in V[1,n]$}
    \State \Return \texttt{True}\Comment{Има извод на думата от $S$}
    \Else
    \State \Return \texttt{False}
    \EndIf
  \end{algorithmic}
\end{algorithm}

\begin{lemma}
  За дадена граматика в нормална форма на Чомски и дума $\alpha$, 
  за всяко $0 \leq s < \abs{\alpha}$, след $s$-тата итерация на алгоритъма (редове 6 - 10), за всяка позиция $i = 1,\dots,n-s$,
  \[V[i,i+s] = \{A \in V \mid A \rightarrow^\star_G a_i\dots a_{i+s}\}.\]
\end{lemma}
\begin{proof}
  Пълна индукция по $s$.
  За $s = 0$  е ясно. (Защо?)

  Нека твърдението е вярно за $s < n$. Ще докажем твърдението за $s+1$, т.е. за всяко $i = 1,\dots,n-s-1$,
  \[V[i,i+s+1] = \{A \in V \mid A \rightarrow^\star_G a_i\dots a_{i+s+1}\}.\]
  % Да разгледаме $A \in V[i,i+s+1]$.
  За едната посока, да разгледаме първoто правило в извода $A \to^\star_G a_i\cdots a_{i+s+1}$.
  Понеже граматиката $G$ е в НФЧ, то е от вида $A \to BC$ и тогава съществува $t\in[0,s]$, за което 
  $B \to^\star a_i\cdots a_{i+t}$ и $C \to^\star a_{i+t+1}\cdots a_{i+s+1}$.
  От {\bf И.П.} получаваме, че $B \in V[i,i+t]$ и $C \in V[i+t+1,i+s+1]$.
  Тогава от ред 10 на алгоритъма е ясно, че $A \in V[i,i+s+1]$.
  
  За другата посока, нека $A \in V[i,i+s+1]$.
  Единствената стъпка на алгоритъма, при която може да сме добавили $A$ към множеството $V[i,i+s+1]$ е ред 10.
  Тогава имаме, че съществува $k\in [i,i+s]$, за което $B \in V[i,k]$, $C \in V[k+1,i+s+1]$, и $A\to BC$ е правило в граматиката $G$.
  От {\bf И.П.} имаме, че $B \to^\star_G a_i\cdots a_k$ и $C \to^\star_G a_{k+1}\cdots a_{i+s+1}$.
  Заключаваме веднага, че $A \to^\star_G a_i\cdots a_{i+s+1}$.
\end{proof}

\begin{example}
  Нека е дадена граматиката $G$ с правила 
  $S\rightarrow a|AB|AC, C\rightarrow SB|AS,A\rightarrow a, B\rightarrow b$.
  Ще приложим $CYK$ алгоритъма за да проверим дали думата $aaabb \in \L(G)$.
  \begin{itemize}
  \item 
    $V[1,1] = V[2,2] = V[3,3] = \{S,A\}$.
    $V[4,4] = V[5,5] = \{B\}$.
  \item
    $V[1,2] = V[2,3] = \{C\}$.
    $V[3,4] = \{S,C\}$.
    $V[4,5] = \emptyset$.
  \item
    $V[1,3] = \{S\} \cup \emptyset$.
    $V[2,4] = \{S,C\} \cup \emptyset$.
    $V[3,5] = \emptyset \cup \{C\}$.
  \item
    $V[1,4] = \{S,C\} \cup \emptyset \cup \emptyset = \{S,C\}$.
    $V[2,5] = \{S\} \cup \emptyset \cup \{C\} = \{S,C\}$
  \item
    $V[1,5] = \{S,C\} \cup \emptyset \cup \emptyset \cup \{C\}= \{S,C\}$.
  \end{itemize}
  Понеже $S \in V[1,5]$, то $aaabb \in \L(G)$.
\end{example}

\begin{thm}
  \marginpar{\cite{hopcroft1}, стр. 137}
  Съществуват алгоритми, които определят по дадена безконтекстна граматика $G$ дали:
  \begin{enumerate}[a)]
  \item 
    $\abs{\L(G)} = 0$;
  \item
    $\abs{\L(G)} < \infty$;
  \item
    $\abs{\L(G)} = \infty$.
  \end{enumerate}
\end{thm}
\begin{proof}
  Нека е дадена една безконтекстна граматика $G$.
  \begin{description}
  \item[($\L(G) = \emptyset?$)]
    Прилагаме алгоритъма за премахване на безполезните променливи.
    Ако открием, че $S$ е безполезна променлива, то $\L(G) = \emptyset$.
  \item[($\abs{\L(G)} < \infty?$ или $\abs{\L(G)} = \infty?$)]
    Нека да разгледаме граматиката $G'$ в НФЧ {\em без безполезни} променливи, за която $\L(G) = \L(G')$.
    От граматиката $G' = \pair{V',\Sigma,S,R'}$ строим граф с възли променливите от $V'$ като
    за $A,B \in V'$ имаме ребро $A \to B$ точно тогава, когато съществува $C \in V'$,
    за което $A \to BC$ или $A \to CB$ е правило в $R'$.
    
    Ако в получения граф имаме цикъл, то $\abs{\L(G')} = \infty$.
    В противен случай, $\abs{\L(G')} < \infty$.
  \end{description}
\end{proof}

\section{Недетерминирани стекови автомати}

\index{автомат!недетерминиран стеков}
%Sipser p.102
\begin{dfn}
  \marginpar{На англ. {\bf Push-down automaton}}% (стр. 157 от \cite{kozen})}
  Недетерминиран стеков автомат е 7-орка от вида
  \[P = \PDA,\] където:
  \begin{itemize}
  \item
    $Q$ е крайно множество от състояния;
  \item  
    $\Sigma$ е крайна входна азбука;
  \item
    $\Gamma$ е крайна стекова азбука;
  \item
    $\# \in \Gamma$ е символ за дъно на стека;
  \item
    $s\in Q$ е начално състояние;
  \item
    \marginpar{Озн. $\Ps_{fin}(A)$ - крайните подмножества на $A$}
    $\Delta:Q\times(\Sigma \cup \{\varepsilon\})\times\Gamma\rightarrow \Ps_{fin}(Q\times\Gamma^\star)$ 
    е функция на преходите;    
  \item
    $F\subseteq Q$ е множество от заключителни състояния.
  \end{itemize}
\end{dfn}

\marginpar{Instanteneous description}
{\em Моментно описание} (или конфигурация) на изчислението със стеков автомат представлява тройка от вида $(q,\alpha,\gamma) \in Q\times\Sigma^\star\times\Gamma^\star$,
т.е. автоматът се намира в състояние $q$, думата, която остава да се прочете е $\alpha$,
а съдържанието на стека е думата $\gamma$.
Удобно е да въведем бинарната релация $\vdash_P$ над $Q\times\Sigma^\star\times\Gamma^\star$,
която ще ни казва как моментното описание на автомата $P$ се променя след изпълнение на една стъпка:
\[(q,x\alpha,Y\gamma) \vdash_P (p,\alpha,\beta\gamma), \text{ ако } \Delta(q,x,Y) \ni (p,\beta),\]
\[(q,\alpha,Y\gamma) \vdash_P (p,\alpha,\beta\gamma), \text{ ако } \Delta(q,\varepsilon,Y) \ni (p,\beta).\]
Рефлексивното и транзитивно затваряне на $\vdash_P$ ще означаваме с $\vdash^\star_P$.
Сега вече можем да дадем дефиниция на език, разпознаван от стеков автомат $P$.
\begin{itemize}
\item
  $\L_F(P)$ е езика, който се разпознава от $P$ {\bfс финално състояние},
  \[\L_F(P) = \{\omega \mid (q_0,\omega,\#) \vdash^\star_P (q,\varepsilon,\alpha)\ \&\ q \in F\}.\]    
\item
  $\L_S(P)$ е езика, който се разпознава от $P$  {\bf с празен стек},
  \[\L_S(P) = \{w\mid (q_0,w,\#) \vdash^\star_P (q,\varepsilon,\varepsilon)\}.\]    
\end{itemize}

\begin{example}
  \label{ex:anbn}
  За езика $L = \{a^nb^n\mid n\in\Nat\}$ съществува стеков автомат $P$, такъв че
  $L = \L_S(P)$.
  Да разгледаме $P = \PDA$, където
  \begin{itemize}
  \item
    $Q = \{q\}$;
  \item
    $\Sigma = \{a,b\}$;
  \item
    $\Gamma = \{\#,A\}$, където символът $\#$ служи за дъно на стека, а броят на $A$-тата в стека ще показват колко букви $a$ сме прочели от думата;
  \item
    $F = \emptyset$, защото разпознаваме с празен стек, а не с финално състояние;
  \item 
    $\Delta(q,a,\#) = \{(q, A\#)\}$;
  \item 
    $\Delta(q,\varepsilon,\#) = \{(q,\varepsilon)\}$;
  \item 
    $\Delta(q,b,A) = \{(q,\varepsilon)\}$.
  \end{itemize}
  Вместо доказтелство, да видим как думата $a^2b^2$ се разпознава от автомата с празен стек:
  \marginpar{\writedown Докажете, че $L = \L_S(P)$!}
  \begin{align*}
    (q,a^2b^2,\#) & \vdash_P (q,ab^2,A\#) \\
    & \vdash_P (q,b^2, AA\#)\\
    & \vdash_P (q,b,A\#)\\
    & \vdash_P (q,\varepsilon,\#)\\
    & \vdash_P (q,\varepsilon,\varepsilon).
  \end{align*}
\end{example}

\begin{example}
  За езика $L = \{\omega\omega^R \mid \omega \in \{a,b\}^\star\}$ съществува стеков автомат $P$, такъв че
  $L = \L_S(P)$.
  Нека $P = \PDA$, където:
  \begin{itemize}
  \item 
    $\Delta(q, a, \#) = \{(q, A\#)\}$;
  \item 
    $\Delta(q, b, \#) = \{(q, B\#)\}$;
  \item
    $\Delta(q, a, A) = \{(q, AA), (p, \varepsilon)\}$;
  \item
    $\Delta(q, a, B) = \{(q, AB)\}$;
  \item
    $\Delta(q, b, B) = \{(q, BB), (p, \varepsilon)\}$;
  \item
    $\Delta(q, b, A) = \{(q, BA)\}$;
  \item
    $\Delta(p, a, A) = \{(p,\varepsilon)\}$;
  \item
    $\Delta(p, b, B) = \{(p,\varepsilon)\}$;
  \item
    $\Delta(q, \varepsilon, \#) = \{(q,\varepsilon)\}$;
  \item
    $\Delta(p, \varepsilon, \#) = \{(p,\varepsilon)\}$;
  \end{itemize}
  Основното наблюдение, което трябва да направим за да разберем конструкцията на автомата е, че
  всяка дума от вида $\omega\omega^R$ може да се запише като $\omega_1aa\omega^R_1$ или $\omega_1bb\omega^R_1$.
  Да видим защо $P$ разпознава думата $abaaba$ с празен стек.
  Започваме по следния начин:
  \begin{align*}
    (q,abaaba,\#) & \vdash_P (q,baaba,A\#)\\
    & \vdash_P (q, aaba, BA\#) \\
    & \vdash_P (q, aba, ABA\#).
  \end{align*}
  Сега можем да направим два избора как да продължим. Състоянието $p$ служи за маркер, което ни казва, че вече сме започнали 
  да четем $\omega^R$. Поради тази причина, продължаваме така:
  \begin{align*}
    (q, aba, ABA\#) & \vdash_P (p, ba, BA\#)\\
    & \vdash_P (p, a, A\#)\\
    & \vdash_P (p, \varepsilon, \#) \\
    & \vdash_P (p,\varepsilon,\varepsilon).
  \end{align*}
  Да проиграем още един пример. Да видим защо думата $aba$ не се извежда от автомата.
  \begin{align*}
    (q,aba,\#) & \vdash_P (q, ba,A\#)\\
    & \vdash_P (q, a, BA\#)\\
    & \vdash_P (q, \varepsilon, ABA\#).
  \end{align*}
  \marginpar{\writedown Докажете, че $\L_S(P) = L$ !}
  От последното моментно описание на автомата нямаме нито един преход, следователно
  думата $aba$ не се разпознава от $P$ с празен стек.
\end{example}


\begin{thm}
  \marginpar{(\cite{hopcroft1}, стр. 114) }
  Нека $L$ е произволен език над азбука $\Sigma$.
  \begin{enumerate}[1)]
  \item 
    Ако съществува НСА $P$, за който $L = \L_F(P)$, то съществува НСА $P^\prime$, за който $L = \L_S(P^\prime)$.
  \item
    Ако съществува НСА $P$, за който $L = \L_S(P)$, то съществува НСА $P^\prime$, за който $L = \L_F(P^\prime)$.
  \end{enumerate}
  С други думи, езиците разпознавани от НСА с празен стек са точно езиците разпознавани от НСА с финално състояние.
\end{thm}
\begin{proof}
  \begin{enumerate}[1)]
  \item 
    Нека $L = \L_F(P)$, където $P = \PDA$.
    Ще построим $P^\prime$, така че да симулира $P$ и като отидем във финално състояние ще изпразним стека.
    Нека
    \[P^\prime = \langle{Q\cup\{q_e,s^\prime\},\Sigma,\Gamma \cup \{\$\},\$,s^\prime,\Delta^\prime,\emptyset}\rangle,\]
    където $\$ \not\in \Gamma$.
    Важно е $P^\prime$ да има собствен нов символ за дъно на стека. В противен случай е възможно за някоя дума $\alpha \not\in \L_F(P)$
    стековият автомат $P'$ да си изчисти стека и така да разпознаем повече думи.
    \begin{itemize}
    \item 
      \marginpar{- започваме симулацията}
      $\Delta'(s^\prime,\varepsilon,\$) = \{(s,\#\$)\}$;
    \item
      \marginpar{- симулираме $P$}
      $\Delta'(q,a,X)$ включва множеството $\Delta(q,a,X)$, за всяко $q\in Q$, $a\in\Sigma_\varepsilon$, $X\in\Gamma$;
    \item
      \marginpar{- ако сме във финално, започваме да чистим стека}
      $\Delta'(q,\varepsilon,X)$ съдържа също и елемента $(q_e,\varepsilon)$, за всяко $q\in F$, $X \in \Gamma \cup \{\$\}$;
    \item
      \marginpar{- изчистваме стека}
      $\Delta'(q_e,\varepsilon,X) = \{(q_e,\varepsilon)\}$, за всяко $X \in \Gamma \cup \{\$\}$;
    \item
      $\Delta'$ няма други правила.
    \end{itemize}
  \item
    Сега имаме $L = \L_S(P)$, където $P = \langle{Q,\Sigma,\Gamma,\#,s,\Delta,\emptyset}\rangle$. 
    Да положим
    \[P^\prime = \langle{Q\cup\{s^\prime,q_f\}, \Sigma, \Gamma \cup \{\$\}, \Delta^\prime, \$, \{q_f\}}\rangle.\]
    $P^\prime$ ще симулира $P$ като ще внимаваме кога $P$ изчиства символа $\#$. Тогава ще искаме да отидем във финалното състояние $q_f$.
    \begin{itemize}
    \item 
      \marginpar{- започваме симулацията}
      $\Delta'(s',\varepsilon,\$) = \{(s, \#\$)\}$;
    \item
      \marginpar{- симулираме $P$}
      $\Delta'(q,a,X) = \Delta(q,a,X)$, за всяко $q \in Q$, $a \in \Sigma_\varepsilon$, $X \in \Gamma$;
    \item
      \marginpar{- щом сме стигнали до $\$$, значи $P$ е изчистил стека си}
      $\Delta'(q,\varepsilon,\$) = \{(q_f,\varepsilon)\}$.
    \end{itemize}
  \end{enumerate}
\end{proof}

\begin{problem}
  Като използвате стековия автомат от Пример \ref{ex:anbn}, дефинирайте автомат $P'$, за който
  $\L_F(P') = \{a^nb^n \mid n\in\Nat\}$.
\end{problem}

\begin{framed}
\begin{thm}
  \label{th:push-down-context-free}
  Класът на езиците, които се разпознават от краен стеков автомат съвпада с
  класа на безконтекстните езици.
\end{thm}
\end{framed}
\begin{proof}
  \marginpar{(\cite{hopcroft1}, стр. 117)}
  Ще разгледаме двете посоки на твърдението поотделно.
  \begin{enumerate}[1)]
  \item 
    Нека е дадена безконтекстна граматика $G = \CFG$.
    Нашата цел е да построим стеков автомат $P$, така че $\L_S(P) = \L(G)$.
    Нека  \[P = \langle{\{q\},\Sigma,\Sigma\cup V,S,q,\Delta,\emptyset}\rangle,\]
    където функцията на преходите е:
    \begin{align*}
      & \Delta(q,\varepsilon,A) = \{(q,\alpha)\mid A\to\alpha\mbox{ е правило в граматиката }G\}\\
      & \Delta(q,a,a) = \{(q,\varepsilon)\}
    \end{align*}
  \item
    Нека имаме $P = \langle{Q, \Sigma, \Gamma, \Delta, s, \#, \emptyset}\rangle$.
    Ще дефинираме безконтекстна граматика $G$, за която $\L_S(P) = \L(G)$.
    Променливите на граматика са 
    \[V = \{[q,A,p] \mid q,p \in Q, A \in \Gamma\}.\]
    Правилата на $G$ са следните:
    \begin{itemize}
    \item
      $S \to [s,\#,q]$, за всяко $q \in Q$;
    \item
      $[q,A,q_{m+1}] \to a[q_1,B_1,q_2][q_2,B_2,q_3]\dots [q_m,B_m,q_{m+1}]$,
      където 
      \[(q_1,B_1\dots B_m) \in \Delta(q, a, A)\]
      и произволни $q,q_1,\dots,q_{m+1} \in Q$,
      $a \in \Sigma \cup \{\varepsilon\}$.

      Да обърнем внимание, че е възможно $m = 0$.
      Това означава, че $(q_1,\varepsilon) \in \Delta(q, a, A)$ и тогава имаме правилото $[q,A,q_{1}] \to a$, където $a \in \Sigma \cup \{\varepsilon\}$.
    \end{itemize}
    Трябва да докажем, че:
    \[[q,A,p] \rightarrow^\star_G \alpha\ \Leftrightarrow\ (q,\alpha,A) \vdash^\star_P (p,\varepsilon,\varepsilon).\]
    \begin{description}
    \item[$(\Rightarrow)$]
      С пълна индукция по $i$, ще докажем, че 
      \[(q,\alpha,A) \vdash^i_P (p,\varepsilon,\varepsilon)\ \implies\ [q,A,p] \Rightarrow^\star_G \alpha.\]
      Ако $i = 1$, то е лесно, защото $\alpha \in \Sigma \cup\{\varepsilon\}$ и $m = 0$.

      Ако $i > 1$, нека $\alpha = a\beta$. Тогава:
      \marginpar{Възможно е $a = \varepsilon$}
      \[(q,a\beta,A) \vdash_P (q_1,\beta,B_1\dots B_n) \vdash^{i-1}_P (p, \varepsilon, \varepsilon).\]
      Да разбием думата $\beta$ на $n$ части, $\beta = \beta_1\cdots \beta_n$, със свойството, че след като прочетем $\beta_i$ 
      сме премахнали променливата $B_i$ от върха на стека. Това означава, че :
      \begin{align*}
        & (q_j, \beta_j, B_j) \vdash^{l_j}_P (q_{j+1},\varepsilon,\varepsilon), \text{ за }j = 1,\dots,n-1,\\
        & (q_n, \beta_n, B_n) \vdash^{l_n}_P (p,\varepsilon,\varepsilon),
      \end{align*}
      където $l_1+l_2+\cdots+l_n = i-1$.
      Сега по {\bf И.П.} получаваме:
      \begin{align*}
        & (q_j, \beta_j, B_j) \vdash^{l_j}_P (q_{j+1},\varepsilon,\varepsilon) \implies [q_j,B_j, q_{j+1}] \rightarrow^\star_G \beta_j, \text{ за }за j = 1,\dots,n-1,\\
        & (q_n, \beta_n, B_n) \vdash^{l_n}_P (p,\varepsilon,\varepsilon) \implies [q_n,B_n, p] \rightarrow^\star_G \beta_n.
      \end{align*}
      Обединявайки тези изводи с правилото
      \[[q,A,p] \rightarrow_G a[q_1,B_1,q_2]\dots[q_n,B_n,p],\]
      получаваме извода
      \[[q,A,p] \rightarrow^\star_G a\beta.\]
    \item[$(\Leftarrow)$]
      Отново с пълна индукция по $i$ ще докажем, че
      \[[q,A,p] \rightarrow^i_G \alpha \implies (q,\alpha,A) \vdash^\star_P (p,\varepsilon,\varepsilon).\]
      Ако $i = 1$, то имаме $[q,A,p] \rightarrow \alpha$, където $\alpha = a$ или $\alpha = \varepsilon$.
      Ако $i > 1$, то имаме, че $\alpha = a\beta$ и за някое $n$, 
      \[[q,A,p] \rightarrow_G a[q_1,B_1,q_2][q_2,B_2,q_3]\dots[q_n,B_n,p] \rightarrow^{i-1}_G \beta.\]
      Отново нека $\beta = \beta_1\dots \beta_n$, където 
      \begin{align*}
        & [q_j,B_j,q_{j+1}] \rightarrow^{i_j}_G \beta_j, \text{ за } j = 1,\dots,n-1,\\
        & [q_{n},B_n,p ] \rightarrow^{i_n}_G \beta_n,
      \end{align*}
      където $i_1 + i_2 + \cdots + i_n = i-1$.
      От {\bf И.П.} получаваме, че 
      \begin{align*}
        & [q_j,B_j,q_{j+1}] \rightarrow^{i_j}_G \beta_j \implies (q_j,\beta_j,B_j) \vdash^\star_P (q_{j+1},\varepsilon,\varepsilon),\ j = 1,\dots,n-1\\
        & [q_n,B_n,p] \rightarrow^{i_n}_G \beta_n \implies (q_n,\beta_n,B_n) \vdash^\star_P (p,\varepsilon,\varepsilon),
      \end{align*}
      Обединявайки всичко, което знаем, получаваме:
      \begin{align*}
        (q, a\beta, A) & \vdash_P (q_1, \beta_1\cdots\beta_n, B_1\cdots B_n)\\
        & \vdash^\star_P (q_2, \beta_{2}\cdots\beta_n, B_2\cdots B_n)\\
        & \dots\\
        & \vdash^\star_P (q_n, \beta_n, B_n)\\
        & \vdash^\star_P (p, \varepsilon, \varepsilon)
      \end{align*}
    \end{description}
  \end{enumerate}
\end{proof}

% \begin{problem}
%   Нека е дадена граматиката $G = \pair{\{S,A,B\},\{a,b\},S,R\}}$.
%   Постройте стеков автомат $P = \PDA$, такъв че $\L_S(P) = \L(G)$, където правилата $R$ на граматиката $G$ са зададени като:
%   \begin{enumerate}[a)]
%     % За едно тези двете да се даде пример как става 
%   \item
%     $R = \{S\rightarrow ASB\vert \varepsilon, A\rightarrow aAa\vert a, B\rightarrow bBb\vert b\}$;
%   \item
%     $R = \{S\rightarrow ASB\vert \varepsilon, A\rightarrow aA\vert a, B\rightarrow Bb\vert b\}$;
%   \item
%     $R =\{S\rightarrow SA|\varepsilon,A\rightarrow BSa|B, B\rightarrow b|BS|ab\}$;
%   \item
%     $R = \{S\rightarrow AS|\varepsilon,A\rightarrow SaBB|A, B\rightarrow b|BBbS|AA\}$;
%   \end{enumerate}
% \end{problem}

\begin{example}
  Нека е дадена граматиката $G$ с правила 
  $S\rightarrow ASB\vert \varepsilon, A\rightarrow aAa\vert a, B\rightarrow bBb\vert b$.
  Ще построим стеков автомат $P = \PDA$, такъв че $\L_S(P) = \L(G)$.
  \begin{itemize}
  \item
    $\Sigma = \{a,b\}$;
  \item 
    $\Gamma = \{A,S,B,a,b\}$;
  \item
    $\# = S$;
  \item
    $Q = \{q\}$;
  \item
    $F = \emptyset$;
  \item
    Дефинираме релацията на преходите, следвайки конструкцията от \Th{push-down-context-free}:
    \begin{itemize}
    \item 
      $\Delta(q,\varepsilon, S) = \{\pair{q,ASB}, \pair{q,\varepsilon}\}$;
    \item
      $\Delta(q, \varepsilon, A) = \{\pair{q, aAa}, \pair{q, a}\}$;
    \item
      $\Delta(q, \varepsilon, B) = \{\pair{q, bBb}, \pair{q, b}\}$;
    \item
      $\Delta(q, a, a) = \{\pair{q,\varepsilon}\}$;
    \item
      $\Delta(q, b, b) = \{\pair{q,\varepsilon}\}$.
    \end{itemize}
  \end{itemize}
\end{example}


\begin{thm}
  \marginpar{(стр. 144 от \cite{papadimitriou})}
  Нека $L$ e безконтекстен език и $R$ е регулярен език.
  Тогава тяхното сечение $L \cap R$ е безконтекстен език.
\end{thm}
\begin{proof}
  Нека имаме стеков автомат
  \[\M_1 = \PDAn{1}, \text{ където } \L_F(\M_1) = L,\]
  \marginpar{всъщност няма нужда да е детерминиран}
  и краен детерминиран автомат 
  \[\M_2 = \FAn{2}, \text{ където } \L(\M_2) = R.\]
  Ще определим нов стеков автомат $\M = \PDA$, където
  \begin{itemize}
  \item 
    $Q = Q_1 \times Q_2$;
  \item
    $s = \pair{s_1,s_2}$;
  \item
    $F = F_1 \times F_2$;
  \item 
    Функцията на преходите $\Delta$ е дефинирана както следва:
    \begin{itemize}
    \item 
      \marginpar{симулираме едновременно изчислението и на двата автомата}
      Ако $\Delta_1(q_1, a, b) \ni \pair{r_1,c}$
      и $\delta_2(q_2,a) = r_2$, то
      \[\Delta(\pair{q_1,q_2},a,b) \ni \pair{\pair{r_1,r_2}, c}.\]
    \item
      \marginpar{празен ход на автомата $M_2$}
      Ако $\Delta_1(q_1,\varepsilon,b) \ni \pair{r_1,c}$,
      то за всяко $q_2 \in Q_2$,
      \[\Delta(\pair{q_1,q_2},\varepsilon,b) \ni \pair{\pair{r_1,q_2},c}.\]    
    \item
      \marginpar{\writedown Докажете, че $\L(\M) = \L(\M_1) \cap \L(\M_2)$ !}
      $\Delta$ не съдържа други преходи;
    \end{itemize}
  \end{itemize}
\end{proof}

\begin{example}
  Езикът $L = \{w \in \{a,b,c\}^\star \mid n_a(w) = n_b(w) = n_c(w)\}$ не е безконтекстен.
\end{example}
\begin{proof}
  Да допуснем, че $L$ е безконтекстен език.
  Тогава \[L^\prime = L \cap \L(a^\star b^\star c^\star)\] също е безконтекстен език.
  Но $L^\prime = \{a^nb^nc^n \mid n \in \Nat\}$, за който знаем от Пример \ref{example:anbncn}, че {\em не} е безконтекстен.
  Достигнахме до противоречие. Следователно, $L$ не е безконтекстни език.
\end{proof}



% \section*{Библиография}

% Основни източници в тази глава са:
% \begin{itemize}
% \item 
%   глава 4 от \cite{hopcroft1}, глави 5, 6 и 7 от \cite{hopcroft2};
% \item
%   глава 2 от \cite{sipser1};
% \item
%   глава 3 от \cite{papadimitriou}.
% \end{itemize}


% \section{Въпроси}
% Вярно ли е, че:
% \begin{itemize}
% \item
% %  \marginpar{Да} 
%   ако $L$ е безконтекстен език, то езикът $L \cap \{a^{2n}b^{2k}\mid n,k\in\Nat\}$ е безконтекстен ?
% \item
%  % \marginpar{Да}
%   ако $L$ е безкраен безконтекстен език, то съществува безкрайна редица от регулярни езици $L_1,L_2,\dots$,
%   за които $L = \bigcup_{i\in\Nat}L_i$ ?
% \item
%   \marginpar{Не}
%   за всяка безкрайна редица от регулярни езици $L_1,L_2,\dots$, то 
%   езикът $L = \bigcup_{i\in\Nat}L_i$ е безконтекстен ?
% \item
%   %\marginpar{Да}
%   за всеки регулярен език $R$ и всеки безконтекстен език $L$, то $L \cap R$ е безконтекстен ?
% \item
%   за всеки регулярен език $R$ и всеки безконтекстен език $L$, то $L \cup R$ е безконтекстен ?
% \item
%   за всеки регулярен език $R$ и всеки безконтекстен език $L$, то $L \setminus R$ е безконтекстен ?
% \item
%   за всеки регулярен език $R$ и всеки безконтекстен език $L$, то $R \setminus L$ е безконтекстен ?
% \item
%   съществува регулярен език $R$ и безконтекстен език $L$, за които $L \cap R$ не е безконтекстен ?
% \item
%   съществува регулярен език $R$ и нерегулярен, но безконтекстен език $L$, за които $L \cap R$ е регулярен ?
% \item
%   за всеки два нерегулярни, но контекстно-свободни езика $L_1,L_2$, то $L_1\cup L_2$ е регулярен ?
% \item
%   съществуват два нерегулярни, но безконтекстни езика $L_1,L_2$, за които $L_1\setminus L_2$ е регулярен ?
% \item
%   съществуват два нерегулярни, но безконтекстни езика $L_1,L_2$, за които $L_1\cap L_2$ е регулярен ?
% \item
%   съществуват два нерегулярни, но безконтекстни езика $L_1,L_2$, за които $L_1\cup L_2$ е регулярен ?
% \item
%   съществува регулярен език $R$, който може да се представи като $R = L_1 \cup L_2$, където
%   $L_1 \cap L_2 = \emptyset$, $L_1,L_2$ са нерегулярни, но контекстно-свободни ?
% \item
%   езикът $\{a,b\}^\star \setminus \{a^nb^n \mid n\in\Nat\}$ е регулярен ?
% \item
%   езикът $\{a,b\}^\star \setminus \{a^nb^n \mid n\in\Nat\}$ е безконтекстен ?
% \item
%   езикът $\{a,b\}^\star \setminus \{a^nb^{2k+1} \mid n,k\in\Nat\}$ е регулярен ?
% \item
%   езикът $\{a,b\}^\star \setminus \{a^nb^{k} \mid n > k\}$ е регулярен ?
% \item
%   езикът $\{a,b\}^\star \setminus \{a^nbba^{n} \mid n \in \Nat\}$ е регулярен ?
% \item
%   езикът $\{a,b\}^\star \setminus \{a^nb^n \mid n\in\Nat\}$ е безконтекстен ?
% \item
%   езикът $\{a,b,c\}^\star \setminus \{a^nb^mc^k \mid m < n\ \&\ m < k\}$ е безконтекстен ?
% \item
%   \marginpar{Не. $\alpha = b^pa^pbba^p$.}
%   езикът $L = \{uvv^R \mid u,v \in \{a,b\}^\star\ \&\ \abs{u} \leq \abs{v}\}$ е регулярен ?
% \item
%   \marginpar{Да.}
%   езикът $L = \{uvv^R \mid u,v \in \{a,b\}^\star\ \&\ \abs{u} \leq \abs{v}\}$ е безконтекстен ?
% \item
%   съществува алгоритъм, който за даден вход регулярен израз $r$ и безконтекстна граматика $G$
%   проверява дали $\L(r) = \L(G)$?
% \item
%   съществува алгоритъм, който за даден вход регулярен израз $r$ и безконтекстна граматика $G$
%   проверява дали $\L(r) \cap \L(G) = \emptyset$?
% \item
%   съществува алгоритъм, който за даден вход регулярен израз $r$ и безконтекстна граматика $G$
%   проверява дали $\abs{\L(r) \cap \L(G)} < \infty$?
% \item
%   съществува алгоритъм, който за даден вход регулярен израз $r$ и безконтекстна граматика $G$
%   проверява дали $\abs{\L(r) \cap \L(G)} = \infty$?
% \item
%   съществува алгоритъм, който за даден вход регулярен израз $r$, безконтекстна граматика $G$
%   и число $k$, проверява дали $\abs{\L(r) \cap \L(G)} = k$?
% \item
%   съществува алгоритъм, който за даден вход регулярен израз $r$ и безконтекстна граматика $G$
%   проверява дали $\L(r) \setminus \L(G) = \emptyset$?
% \item
%   съществува алгоритъм, който за даден вход регулярен израз $r$ и безконтекстна граматика $G$
%   проверява дали $\L(G) \setminus \L(r) = \emptyset$?
% \item
%   съществува алгоритъм, който за даден вход регулярен израз $r$ и безконтекстна граматика $G$
%   проверява дали $\abs{\L(r) \setminus \L(G)} < \infty$?
% \item
%   съществува алгоритъм, който за даден вход регулярен израз $r$ и безконтекстна граматика $G$
%   проверява дали $\abs{\L(G) \setminus \L(r)} < \infty$?
% \item
%   съществува алгоритъм, който за даден вход регулярен израз $r$ и безконтекстна граматика $G$
%   проверява дали $\abs{\L(r) \setminus \L(G)} = \infty$?
% \item
%   съществува алгоритъм, който за даден вход регулярен израз $r$ и безконтекстна граматика $G$
%   проверява дали $\abs{\L(G) \setminus \L(r)} = \infty$?
% \item
%   съществува алгоритъм, който за даден вход регулярен израз $r$, безконтекстна граматика $G$
%   и число $k$, проверява дали $\abs{\L(r) \setminus \L(G)} = k$?
% \item
%   съществува алгоритъм, който за даден вход регулярен израз $r$, безконтекстна граматика $G$
%   и число $k$, проверява дали $\abs{\L(G) \setminus \L(r)} = k$?
% \end{itemize}

% Нека е дадена безконтекстна граматика $G$ с правила \[S\rightarrow a\vert AB \vert AC, A \rightarrow a, B\rightarrow b, C\rightarrow SB.\]
% Вярно ли е, че ако приложим CYK алгоритъма върху думата $\alpha$, където
% \begin{itemize}
% \item 
%   $\alpha = aabb$, то $N[1,1] = \{S\}$.
% \item 
%   $\alpha = aabb$, то $N[3,3] = \{B\}$.
% \item 
%   $\alpha = aabb$, то $N[1,4] = \{\}$.
% \item
%   $\alpha = baab$, то $N[2,4] = \{\}$.
% \item
%   $\alpha = baab$, то $N[1,3] = \{\}$.
% \end{itemize}



%%% Local Variables: 
%%% mode: latex
%%% TeX-master: "EAI"
%%% End: 


\section{Допълнителни задачи}

\begin{problem}
  Докажете, че следните езици са безконтекстни:
  \begin{enumerate}[a)]
  \item 
    $L = \{\omega_1\sharp\omega_2 \mid \omega_1,\omega_2 \in \{a,b\}^\star\ \&\ \abs{\omega_1} = \abs{\omega_2}\}$;
  \item
    $L = \{\omega_1 \sharp \omega_2 \sharp \cdots \sharp \omega_n \mid n\geq 2\ \&\ \omega_1,\omega_2,\dots,\omega_n \in \{a,b\}^\star\ \&\ \abs{\omega_1} = \abs{\omega_2}\}$;
  \item
    $L = \{\omega_1 \sharp \omega_2 \sharp \cdots \sharp \omega_n \mid n\geq 2\ \&\ \omega_1,\dots,\omega_n \in \{a,b\}^\star\ \&\ (\exists i \neq j)[\abs{\omega_i} = \abs{\omega_j}]\}$;
  \item
    $L = \{\omega_1 \sharp \omega_2 \sharp \cdots \sharp \omega_n \mid n\geq 2\ \&\ (\forall i\in[1,n])[\omega_i \in \{a,b\}^\star\ \&\ \abs{\omega_i} = \abs{\omega_{n+1-i}}]\}$.
  \end{enumerate}
\end{problem}


\begin{problem}
  Проверете дали следните езици са безконтекстни:
  \begin{enumerate}[a)]
  \item
    $\{a^nb^{2n}c^{3n}\ \mid\ n\in\Nat\}$;
  \item
    $\{a^nb^{2n}c^{n}\ \mid\ n\in\Nat\}$;
  \item
    $\{a^mb^n\mid\ m \neq n\}$;
  \item
    $\{a^nb^mc^k\mid n < m < k\}$;
  \item
    $\{a^nb^mc^k\mid k = \min\{n,m\}\}$;
  \item
    $\{a^nb^nc^m\mid m \leq n\}$;
  \item
    $\{a^nb^mc^k\mid k = n\cdot m\}$;
  \item
    $L^\star$, където
    $L = \{\alpha\alpha^R \mid \alpha \in \{a,b\}^\star\}$;
  \item
    $\{www\mid w\in \{a,b\}^\star\}$;
  \item
    $\{ww^R\mid w\in \{a,b\}^\star\}$;
  \item
    $\{a^{n^2}b^n\ \mid n \in \Nat\}$;
  \item
    $\{a^p\ \mid\ p\mbox{ е просто }\}$;
  \item
    $\{\omega \in \{a,b\}^\star \mid \omega = \omega^R\}$;
  \item
    $\{\omega^n \mid \omega \in \{a,b\}^\star\ \&\ n \in \Nat\}$;
  \item
    $\{a^{n^3 + 2n^2} \mid n \in \Nat\}$;
  \item
    % Дефиниция на подниз
    $L = \{w c x\mid w,x\in \{a,b\}^\star\ \&\ w\mbox{ е подниз на }x\}$;
  \item
    $L = \{x_1 c x_2 c \dots c x_k\mid k\geq 2\ \&\ x_i\in\{a,b\}^\star\ \&\ (\exists i,j)[i \neq j\ \&\ x_i = x_j]\}$;
  \item
    $L = \{a^ib^jc^k\mid i,j,k\geq 0\ \&\ (i = j \vee j = k)\}$;
  \item
    % \marginpar{Разгл. $L' = L \cap L(a^*b^*c^*)$.}
    $L = \{\alpha \in \{a,b,c\}^\star\mid n_a(\alpha) > n_b(\alpha) > n_c(\alpha)\}$;
  \item
    $L = \{a,b\}^\star \setminus \{a^nb^n\mid n\in \Nat\}$;
  \item
%    \marginpar{Разгл. $L' = L \cap L(a^*b^*a^*)$.}
    $L = \{\omega \in \{a,b\}^\star \mid n_a(\omega) = 2n_b(\omega)\}$;
  \item
    $L = \{a^nb^mc^ma^n \mid m,n\in\Nat\ \&\ n = m+42\}$;
  \item
    $L = \{babaabaaab\cdots ba^{n-1}ba^nb \mid n \geq 1\}$;
%   \end{enumerate}
% \end{problem}

% \begin{problem}
%   Проверете кои от следните езици са безконтекстен:
%   \begin{enumerate}[a)]
  \item
    $\{a^mb^nc^k\mid m = n \vee n = k \vee m = k\}$;
  \item
    $\{a^mb^nc^k\mid m \neq n \vee n \neq k \vee m \neq k\}$;
  \item
    $\{a^mb^nc^k\mid m = n \wedge n = k \wedge m = k\}$;
  \item
    $\{w \in \{a,b,c\}^\star\mid n_a(w) \neq n_b(w) \vee n_a(w) \neq n_c(w) \vee n_b(w) \neq n_c(w)\}$.
  \end{enumerate}
\end{problem}

\begin{problem}
  Докажете, че ако $L$ е безконтекстен език, то $L^R = \{\omega^R \mid \omega \in L\}$ 
  също е безконтекстен.
\end{problem}

\begin{problem}
  Нека $\Sigma = \{a,b,c,d,f,e\}$.
  Докажете, че езикът $L$ е безконтекстен, където за думите $\omega \in L$ са изпълнени свойствата:
  \begin{itemize}[-]
  \item 
    за всяко $n\in\Nat$, след всяко срещане на $n$ последнователни $a$-та
    следват $n$ последователни $b$-та, и $b$-та не се срещат по друг повод в $\omega$, и
  \item
    за всяко $m\in\Nat$, след всяко срещане на $m$ последнователни $c$-та
    следват $m$ последователни $d$-та, и $d$-та не се срещат по друг повод в $\omega$, и
  \item
    за всяко $k\in\Nat$, след всяко срещане на $k$ последнователни $f$-а
    следват $k$ последователни $e$-та, и $e$-та не се срещат по друг повод в $\omega$.
  \end{itemize}
\end{problem}

\begin{problem}
  Да разгледаме езиците:
  \begin{align*}
    & P = \{\alpha\in\{a,b,c\}^*\,|\, \alpha \text{ е палиндром с четна дължина}\} \\
    & L =  \{\beta b^n\,|\, n\in\mathbb{N}, \beta\in P^n\}.
  \end{align*}
  Да се докаже, че:
  \begin{enumerate}[a)]
  \item 
    $L$ не е регулярен;
  \item 
    $L$ е безконтекстен.
  \end{enumerate}
\end{problem}

\begin{problem}
  Нека $L_1$ е произволен регулярен език над азбуката $\Sigma$, 
  а $L_2$ е езика от всички думи палиндроми над $\Sigma$.
  Докажете, че $L$ е безконтекстен език, където:
  \[L = \{\alpha_1\alpha_2\cdots\alpha_{3n}\beta_1\cdots\beta_m\gamma_1\cdots\gamma_n \mid \alpha_i,\gamma_j \in L_1, \beta_k\in L_2, m,n \in \Nat\}.\]
\end{problem}

\begin{problem}
  Нека $L = \{\omega\in\{a,b\}^\star \mid N_a(\omega) = 2\}$.
  Да се докаже, че езикът $L' = \{\alpha^n \mid \alpha\in L, n \geq 0\}$ не е безконтекстен.
\end{problem}


\begin{problem}
  Нека $\Sigma = \{a,b,c\}$ и $L \subseteq \Sigma^\star$ е безконтестен език. Ако имаме дума 
  $\alpha \in \Sigma^\star$, тогава \emph{L-вариант} на $\alpha$ ще наричаме думата, която се получава като в $\alpha$ всяко едно 
  срещане на символа $a$ заменим с (евентуално различна) дума от $L$.
  Тогава, ако $M \subseteq \Sigma^*$ е произволен безконтестен език, да се докаже че езикът
  \begin{equation*}
    M' = \{\beta\in\Sigma^\star |\ \beta \text{ е $L$-вариант на } \alpha \in M \}
  \end{equation*}
  също е безконтекстен.
\end{problem}

\begin{problem}
  Докажете, че всеки безконтекстен език над азбуката $\Sigma = \{a\}$
  е регулярен.
\end{problem}

\begin{problem}
%  \marginpar{\cite{papadimitriou} стр. 149}
  Да фиксираме азбуката $\Sigma$.
  Нека $L$ е безконтекстен език, а $R$ е регулярен език.
  Докажете, че езикът
  $L/R = \{\omega \in \Sigma^\star \mid (\exists u \in R)[\omega u \in L]\}$
  е безконтекстен.
\end{problem}


\begin{problem}
  Нека е дадена граматиката $G = \pair{\{a,b\}, \{S,A,B,C\},S,R}$.
  Използвайте CYK-алгоритъма, за да проверите дали
  думата $\alpha$ принадлежи на $\L(G)$, където правилата на граматиката $R$ и думата $\alpha$
  са зададени като:
  \begin{enumerate}[a)]
  \item
    $R = \{S\rightarrow BA| CA|a, C\rightarrow BS|SA,A\rightarrow a, B\rightarrow b\}$, $\alpha=bbaaa$;
  \item
    $R =\{S\rightarrow AB|BC, A\rightarrow BA|a,B\rightarrow CC|b, C\rightarrow AB|a\}$, $\alpha=baaba$;
  \item
    $R = \{S\rightarrow AB, A\rightarrow AC|a|b,B\rightarrow CB|a, C\rightarrow a\}$, $\alpha=babaa$;
  \end{enumerate}
\end{problem}

\begin{problem}
  \marginpar{Интересно е също да се направи и безконтекстна граматика за $L$}
  Постройте стеков автомат за езика $L$ над азбуката $\{a,b,\sharp\}$, където
  \[L = \{\omega_1 \sharp \omega_2 \sharp \cdots \sharp \omega_{2n} \mid n\in\Nat\ \&\ \sum^n_{i=1}\abs{\omega_{2i}} = \sum^{n}_{i=1}\abs{\omega_{2i-1}}\}.\]
\end{problem}


%%% Local Variables: 
%%% mode: latex
%%% TeX-master: "EAI"
%%% End: 


% \include{turing}


\bibliographystyle{amsalpha}
\bibliography{EAI}

\printindex

\end{document}

%%% Local Variables: 
%%% mode: latex
%%% TeX-master: t
%%% End: 
